%================================
\section{The Definition of Topological Space}
%================================


%--------------------------------
\begin{definition}
	Let $X$ be a set. A \textit{topological} on $X$ is a collection $\mathcal T \in \mathcal P(X)$ with the following properties.
	\begin{enumerate}[T1.]
		\item $\mathcal T$ is closed under arbitrary union;
		\item $\mathcal T$ is closed under finite intersection;
		\item $X \in \mathcal T$.
	\end{enumerate}
	
	The \textit{Topological Space} $(X, \mathcal T)$ is a set $X$ with a topology $\mathcal T$ on $X$. All $\mathcal T$-sets are said to be \textit{open} in $(X, \mathcal T)$.
\end{definition}
%--------------------------------


%--------------------------------
\begin{lemma}
	$\emptyset \in \mathcal T$.
\end{lemma}
%--------------------------------


%--------------------------------
\begin{proof}
	By T1, given $I$ as any index set, if for all $i \in I$, $U_i \in \mathcal T$, then
	$$
	U = \bigcup_{i \in I} U_i \in \mathcal T.
	$$
	If $I = \emptyset$, then $U = \emptyset$.
\end{proof}
%--------------------------------


%--------------------------------
\begin{note}
	Let $X = \{1,2,3\}$ with topology
	$$
	\mathcal T = \big\{ \{1,2\}, \{3\} \big\}.
	$$
	$\{1,2\} \in \mathcal T$ implies $\{3\} = X \setminus \{1,2\}$ is closed; $\{3\} \in \mathcal T$ implies that $\{1,2\} = X \setminus \{3\}$ is closed. $\{2\} \in \mathcal P(X)$, but $\{2\} \notin \mathcal T$, so $\{2\}$ is not open in $(X, \mathcal T)$, $\{1,3\} = X \setminus \{2\}$ is not closed. For any $U \in \mathcal T$, $\{2\} \ne X \setminus U$, so $\{2\}$ is not open.
\end{note}
%--------------------------------


%--------------------------------
\begin{definition}
	Given $(X, \rho)$ as a metric space, the topology
	$$
	\mathcal T_\rho = \left\{U \in \mathcal P(X) : U = \bigcup_{x \in U} B(x, \delta)\right\},
	$$
	then we call $\mathcal T_\rho$ the topology \textit{induced} by $\rho$, and $(X, \mathcal T_\rho)$ the \textit{underlying topological space} of metric space $(X, \rho)$.
\end{definition}
%--------------------------------


%--------------------------------
\begin{note}
	These topology is induced by metric.
	%--------------------------------
	\begin{enumerate}
		%--------------------------------
		\item In this case, $U$ is open in $(X, \rho)$ iff $U \in \mathcal T_\rho$.
		%--------------------------------
		
		%--------------------------------	
		\item The metric $\rho_p :\mathbb R \times \mathbb R^n \to \mathbb R_{> 0}$ (surjective) induces $\mathcal T_{\rho_p} \subseteq \mathcal P(X)$. And we'll see that for all $p, q \ge 1$, $\mathcal T_{\rho_p} = \mathcal T_{\rho_q}$.
		%--------------------------------
		
		%--------------------------------
		\item The discrete topology $\rho_\mathrm{disc} : X \times X \to \mathbb R_{> 0}$ (non-sujective) induces $\mathcal T_{\rho_\mathrm{disc}} = \mathcal P(X)$. It is the largest topology on $X$, and $\rho[X \times X] \subseteq \{0,1\}$.
		%--------------------------------
		
		%--------------------------------
		\item The metric $\rho_p: C[a,b] \times C[a,b] \to \mathbb R_{>0}$ (surjective) induces $\mathcal T_{\rho_p} \subseteq \mathcal P(X)$. And we'll see that $\mathcal T_{\rho_1} \ne \mathcal {T}_{\rho_\infty}$.
		%--------------------------------
		
		%--------------------------------
		\item Given $X$ as a space, the Hausdorff metric $\rho_H: \mathcal P(X) \times \mathcal P(X) \to \mathbb R_{>0}$ (surjective) induces $\mathcal T_{\rho_H} \subseteq \mathcal P(\mathcal P(X))$.
		%--------------------------------
		
		%--------------------------------
		\item Given $A$ as a set, the hamming metric $\rho: A^n \times A^n \to \mathbb R_{> 0}$ (non-surjective) with $n \in \mathbb N$ induces $\mathcal T_\rho \subseteq \mathcal P(X)$. $\rho[A^n \times A^n] = \mathbb N_{\le n}$.
		%--------------------------------
	\end{enumerate}
	%--------------------------------
	
	These topology is not induced by any metric.
	%--------------------------------
	\begin{enumerate}
		%--------------------------------
		\item The indiscrete topology $\mathcal T = \{\emptyset, X\}$ on $X$ is not induced by any metric space. Suppose it was, then there would be a metric $\rho$ such that for all $x \in X$, for all $\varepsilon > 0$, $B(x, \varepsilon) \in \mathcal T$. But, clearly, for those $\varepsilon \in (0, \phi X)$, $B(x, \varepsilon) \notin \mathcal T$.
		%--------------------------------
		
		%--------------------------------
		\item Let $X = \{1,2,3\}$ with topology
		$$
		\mathcal T = \big\{ \{1,2\}, \{3\} \big\}.
		$$
		These is no such metic $\rho$ induces $\mathcal T$ for same reason.
		%--------------------------------
	\end{enumerate}
	%--------------------------------
\end{note}
%--------------------------------


%--------------------------------
\begin{definition}
	Let $X$ be a set and $\mathcal T, \mathcal T'$ be topologies on $X$. If $\mathcal T \subseteq \mathcal T'$, then we say that $\mathcal T'$ is \textit{finer} than $\mathcal T$, or $\mathcal T$ is \textit{coarser} than $\mathcal T'$.
\end{definition}
%--------------------------------


%--------------------------------
\begin{note} \
	\begin{enumerate}
		%--------------------------------
		\item Given $X$ as a set, for all topology $\mathcal T$ on $X$, $\mathcal T \subseteq \mathcal T_\mathrm{disc}$ and $\mathcal T \supseteq \mathcal T_\mathrm{indisc}$, Thus, $\mathcal T_\mathrm{disc}$ is the finest topology on $X$, and $\mathcal T_\text{indisc}$ is the coarsest.
		%--------------------------------
		
		%--------------------------------
		\item $\rho_p$ and $\rho_\text{disc}$ induced same topology on $\mathbb Z$. But on $\mathbb Q$, $\mathcal T_{\rho_p}$ is coarser than $\mathcal T_{\rho_\text{disc}}$.
		%--------------------------------
	\end{enumerate}
\end{note}
%--------------------------------


%--------------------------------
\begin{definition}
	Given $(X, \mathcal T)$ as a topological space, a set $V \subseteq X$ is said to be \textit{closed} in $(X, \mathcal T)$ iff $X \setminus V \in \mathcal T$.
\end{definition}
%--------------------------------


%--------------------------------
\begin{definition} \
	\begin{enumerate}
		\item In the discrete topology on $X$, all subsets are closed. Because for all $U \in \mathcal T_\text{disc}$, $X \setminus U \in \mathcal T_\text{disc}$.
		\item In the indiscrete topology on $X$, only $\emptyset$ and $X$ is closed.
	\end{enumerate}
\end{definition}
%--------------------------------


%--------------------------------
\begin{lemma}
	Let $X = (X, \mathcal T)$ be a topological space, and let
	$$
	\mathcal C = \left\{ V \subseteq X : V = X \setminus U , \; U \in \mathcal T\right\}.
	$$
	\begin{enumerate}[(i)]
		\item $\mathcal C$ is closed under arbitrary intersection;
		\item $\mathcal C$ is closed under finite intersection;
		\item $\emptyset, X \in \mathcal C$.
	\end{enumerate}
\end{lemma}
%--------------------------------

%--------------------------------
\begin{proof} \
	\begin{enumerate}[(i).]
		\item By De Morgan's laws,
		$$
		\begin{aligned}
			V = X \setminus \bigcup_{i \in I} U_i = \bigcap_{i \in I} (X \setminus U_i).
		\end{aligned}
		$$
		So, if $U_i \in \mathcal T$, then $V \in \mathcal C$.
		
		\item By De Morgan's laws,
			$$
			V = X \setminus \bigcap_{i = 1}^n U_i = \bigcup_{i = 1}^n (X \setminus U_i).
			$$
			
		\item 
			$$
			\emptyset = X \setminus X, \; X = X \setminus \emptyset.
			$$
	\end{enumerate}
\end{proof}
%--------------------------------


%--------------------------------
\begin{definition}
	Let $(X, \mathcal T)$ be a topological space, and let $x \in X$. An \textit{open neighbourhood} of $x$ is a set $N_x \in \mathcal T$ with $x \in N_x$. A \textit{neighbourhood} of $x$ is any $N_x' \supseteq N_x$.
\end{definition}
%--------------------------------


%--------------------------------
\begin{note}
	Given $(X, \mathcal T)$ as a topological space. If $A \in \mathcal T$,
	$$
	A = \bigcup_{x \in A} B, \quad B \in x, \text{ and } B \in \mathcal T.
	$$
	If $\mathcal T = \mathcal T_\rho$ for some metric $\rho$ on $X$, then $A \in \mathcal T$ implies
	$$
	A= \bigcup_{x \in A} B(x, \varepsilon)
	$$
	for some $\varepsilon > 0$.
\end{note}
%--------------------------------


%--------------------------------
\begin{lemma}
	Let $(X, \mathcal T)$ be a topological space and $U \subseteq X$. Then $U \in \mathcal T$ iff for all $x \in U$, there is a neighbourhood $N_x' \subseteq U$.
\end{lemma}
%--------------------------------


%--------------------------------
\begin{proof}
	If $U \ni x$ and $U \in \mathcal T$, then $U$ is an open neighbourhood of $x$, naturally, it is a neighbourhood of $x$.
	
	For only if, clearly, if for all $x \in U$, there is a neighbourhood $N_x' \subseteq U$, then, by definition, there is $N_x \subseteq N_x'$ with $N_x \in \mathcal T$. Now we have $x \in N_x \subseteq N_x' \subseteq U$, then,
	$$
	U = \bigcup_{x \in U} N_x.
	$$
	By definition, $\mathcal T$ is closed under arbitrary union, thus $U$ is open.
\end{proof}
%--------------------------------









































%
