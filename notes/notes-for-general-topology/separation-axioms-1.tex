%================================
\section{Separation Axioms. From $T_0$ to Hausdorff}
%================================

% links

% https://en.wikipedia.org/wiki/Separation_axiom

% https://en.wikipedia.org/wiki/Separated_sets

% https://proofwiki.org/wiki/Definition:Tychonoff_Separation_Axioms


%--------------------------------
\begin{definition}
	[saperated]
	\label{def: seperated}
	In a topological space, two sets are said to be \textit{separated} iff each is disjoint from other's closure.
\end{definition}
%--------------------------------


%--------------------------------
\begin{definition}
	[separated by neighbourhoods]
	\label{def: separated by neighbourhoods}
	In a topological space $(X, \mathcal T)$, two sets $A$ and $B$ are said to be \textit{separated by neighbourhood} iff there are neighbourhoods $N_A$ of $A$ and $N_B$ of $B$ such that $N_A$ and $N_B$ are disjoint.
\end{definition}
%--------------------------------


%--------------------------------
\begin{definition}
	[topologically indistinguishable]
	\label{def: topologically indistinguishable}
	Let $(X, \mathcal T)$ be a topological space. Two points $x,y \in X$ are said to be \textit{topologically indistinguishable} iff they share all their neighbourhoods. That is, let $\mathcal N_x$ be the family of all neighbourhoods of $x$ and let $\mathcal N_y$ be the family of all neighbourhoods of $y$, we have
	$$
	\mathcal N_x = \mathcal N_y.
	$$

	Respectively, $x,y$ are said to be \textit{topologically distinguishable} iff they are not topologically distinguishable; i.e.,
	$$
	\mathcal N_x \ne \mathcal N_y.
	$$
\end{definition}
%--------------------------------


%--------------------------------
\begin{example}
	In an indiscrete topological space, all distinct points are topologically indistinguishable.
\end{example}
%--------------------------------


\centering{
	\subsection*{$T_0$ Spaces}
}

%--------------------------------
\begin{definition}
	[$T_0$ spaces]
	\label{def: T_1 spaces}
	A topological space $(X, \mathcal T)$ is said to be $T_0$ or \textit{Kolmogorov}, iff all distinct points $x,y \in X$ are topologically distinguishable.
\end{definition}
%--------------------------------


%--------------------------------
\begin{example}
	Let $X$ be any set and let $\mathcal T$ be the indiscrete topology on $X$. $(X, \mathcal T)$ is $T_0$ iff $|X| \in \{0, 1\}$.
\end{example}
%--------------------------------


\centering{
	\subsection*{$T_1$ Spaces}
}


%--------------------------------
\begin{definition}
	[$R_0$ spaces]
	\label{def: R_0 spaces}
	A topological space $(X, \mathcal T)$ is said to be $R_0$ iff any two topologically distinguishable points in $X$ are separated.
\end{definition}
%--------------------------------


%--------------------------------
\begin{definition}
	[$T_1$ Spaces]
	\label{def: T_1 spaces}
	A topological space $(X, \mathcal T)$ is said to be $T_1$ or \textit{Fr\'echet} iff it is $T_0$ and $R_0$; i.e., all distinct pionts $x,y \in X$ are separated.
\end{definition}
%-------------------------------

%--------------------------------
\begin{example}
	[$R_0$ but not $T_0$]
	Let $\mathcal T$ be a countable family of disjoint proper intervals on $\mathbb R^n$, and $\bigcup \mathcal T = \mathbb R^n$. $(X, \mathcal T)$ is $R_0$, but not $T_0$.
\end{example}
%--------------------------------


%--------------------------------
\begin{example}
	[$T_0$ but not $R_0$]
	\label{eg: T_0 but not T_1}
	Let $(\mathbb R_{\ge 0}, \mathcal T)$ be a topological space with
	$$
	\mathcal T = \left\{ U \subseteq \mathbb R : \forall i \in \mathbb R_{\ge 0}, \ U_i = [0, i)  \right\},
	$$
	Then for all $x,y \in (\mathbb R_{\ge 0}, \mathcal T)$, if $x \ne y$, then there are $|y - x|$ neighbourhoods $N_x$ of $x$ do not contain $y$. Thus, it is $T_0$.

	On the other hand, it is not $R_0$, because for all $x, y \in (\mathbb R_{\ge 0}, \mathcal T)$ with $x < y$, $x \in \overline{\{y\}} = [0, y]$.
\end{example}
%--------------------------------


%--------------------------------
\begin{example}
	[$R_0$ but not $T_1$]
	\label{eg: R_0 but not T_1}

	Let $X$ be any set with $|X| \ge 3$, let $U \subsetneq X$ with $|U| \ge 2$, let $\mathcal T_{X \setminus U}$ be a $T_1$ topology on $X \setminus U$, and let $\mathcal T$
	$$
	\mathcal T = \mathcal T_{X \setminus U} \cup \{X, U\}.
	$$

	For all $x,y \in X$, if $x \ne y$, then they are separated. Thus, the space is $R_0$.

	But $(X, \mathcal T)$ is not $T_1$, because all $\{u\} \in U$ share the same closure which is $U$ itself.
\end{example}
%--------------------------------


%--------------------------------
\begin{proposition}
	[alternative definitions of $R_0$ spaces]
	\label{prop: alternative definitions of R_0 spaces}
	Let $(X, \mathcal T)$ be $R_0$, then the following conditions are equivalent.
	\begin{enumerate}[(i)]
		\item
		The closure of all singletons in $X$ are not $T_0$ subspace.

		\item
		For any two points $x,y \in X$, $x \in \overline{ \{y\} }$ iff $y \in \overline{ \{x\} }$.

		\item
		Every open set is the union of closed sets.
	\end{enumerate}

	\begin{proof}
		\
		\begin{enumerate}[(i)]
			\item
			By Definition \ref{def: R_0 spaces}, if $y$ and $x$ are topologically distinguishable, by Definition \ref{def: R_0 spaces}, $x$ and $y$ are separated; i.e., $x \notin \overline {\{y\}}$ and $y \notin \overline{\{x\}}$.

			\item
			By Definition \ref{def: R_0 spaces}, for all $x,y \in X$, $x,y$ are not separated only if they are topologically indistinguishable. By Definition \ref{def: topologically indistinguishable}, they share all their neighbourhoods, thus they have the same closure; i.e., $\overline{\{x\}} = \overline{\{y\}}$.

			\item
			For any $U \in \mathcal T$,
			$$
			U = \bigcup_{x \in U} \{x\}.
			$$
			If $(X, \mathcal T)$ is $T_1$, then we are done. Suppose $(X, \mathcal T)$ is not $T_1$, then there exists $A \in \mathcal T$ with $|A| > 1$, and for all $B \subsetneq A$, $B \notin \mathcal T$ (proof omitted). For such $A$, $X \setminus A$ is open, for $X \setminus A = \bigcup (\mathcal T \setminus \{A\})$, thus $A$ is also closed.

			Suppose for any such $A$ with $A \cap U \ne \emptyset$, $A \subseteq U$. Suppose it fails, i.e., $A \cap U \ne A$, then we have $A \cap U \subsetneq A$ and $A \cap U \in \mathcal T$, which is contradicted to the condition of $A$. Now we have
			$$
			U = \bigcup \mathcal A \cup \bigcup_{x \in I} \{x\}
			$$
			where $\mathcal A$ is the family of such $A$, and $I$ is the union of all closed singletons in $U$. Thus $U$ is open.
		\end{enumerate}
	\end{proof}
\end{proposition}
%--------------------------------


%--------------------------------
\begin{proposition}
	[alternative definitions of $T_1$ spaces]
	\label{prop: alternative definitions of T_1 spaces}
	Let $(X, \mathcal T)$ be $T_1$, then the following conditions are equivalent.
	\begin{enumerate}[(i)]
		\item All singletons in $X$ are closed.
		\item Every subset of $X$ is the intersection of all open sets containing it.
		\item Every cofinite subset of $X$ is open.
	\end{enumerate}


	\begin{proof} \
		\begin{enumerate}[(i)]
			\item
			Suppose there exists $\{x\} \subseteq X$ with $\overline{\{x\}} \ne \{x\}$, then there exists $y \in \overline{\{x\}}$ with $x \ne y$. By Definition \ref{def: T_1 spaces}, this is impossible.

			\item
			For any $A \subseteq X$,
			$$
			A = \bigcup_{x \in A} \{x\}.
			$$
			Let $B = X \setminus A$. By De Morgan's law,
			$$
			B = \bigcap_{x \in A} X \setminus \{x\}.
			$$
			$(X, \mathcal T)$ is $T_1$ iff all $\{x\}$ are closed, in which case, $B$ is the intersection of all open sets $X \setminus \{x\} \supseteq B$.

			\item
			Let $A$ be a cofinite subset of $X$. $X \setminus A$ is a finite union of singletons. As $(X, \mathcal T)$ is $T_1$, any singletons in $X$ is closed. By Proposition \ref{prop: dark side of topology}, $X \setminus A$ is closed. By Definition \ref{def: closed sets}, $A$ is open.�
		\end{enumerate}
	\end{proof}
\end{proposition}
%--------------------------------


\centering{
	\subsection*{Hausdorff Spaces}
}


%--------------------------------
\begin{definition}
	[$R_1$ spaces]
	\label{def: R_1 spaces}
	A topological space $(X, \mathcal T)$ is said to be $R_1$ iff any two topological distinguishable points in $X$ are separated by neighbourhoods.
\end{definition}
%--------------------------------


%--------------------------------
\begin{definition}
	[Hausdorff Spaces]
	\label{def: Hausdorff spaces}
	A topological space $(X, \mathcal T)$ is said to be \textit{Hausdorff} or $T_2$ iff it is $T_0$ and $R_1$; i.e., all distinct points $x, y \in X$ are separated by neighbourhoods.
\end{definition}
%--------------------------------


%--------------------------------
\begin{proposition}
	All metrizable spaces are Hausdorff

	\begin{proof}
		Let $(X, \mathcal T)$ be a metrizable space. There exists a metric $\rho$ on $X$ that induces $\mathcal T$. Given distinct points $x,y \in X$, suppose for all $\varepsilon \in \mathbb R_{> 0}$, there exists $z \in B(x, \varepsilon) \cap B(y, \varepsilon)$. Then $\rho(x, z) < \varepsilon$ and $\rho(y, z) < \varepsilon$. Now we have
		$$
		\rho(x, z) + \rho(y, z) < 2\varepsilon.
		$$

		Put $\rho(x,y) > 2\varepsilon$ as $x$ and $y$ are arbitrarily given. Then we have
		$$
		\rho(x, z) + \rho(y,z) < \rho(x,y),
		$$
		which implies that $\rho$ is not a metric on $X$. Hence, $(X, \mathcal T)$ is not metrizable which is contradicted to the condition.
	\end{proof}
\end{proposition}
%--------------------------------


%--------------------------------
\begin{proposition}
	\label{prop: T2 implies T1}
	All singletons in a Hausdorff space are closed.

	\begin{proof}
		Let $(X, \mathcal T)$ be a Hausdorff space, and let $x \in X$. For all $y \in X$ with $x \ne y$, there is a open neighbourhood $U_y$ of $y$ such that $x \notin U_y$. Then, for all such $U_y$, we have
		$$
		\forall y \in X, \; x \in X \setminus U_y = \{x\} \iff x \in \bigcap_{y \in X \setminus \{x\}} X \setminus U_y  = \{x\}.
		$$
		As all $X \setminus U_y$ are closed, their intersection $\{x\}$ is closed.
	\end{proof}
\end{proposition}
%--------------------------------


%--------------------------------
\begin{example}
	[$T_1$ but not Hausdorff]
	\label{eg: T_1 but not T_2}
	Let $X$ be a nonempty set, let $p \in X$, let $\mathcal T'$ be a Hausdorff topology on $X \setminus \{p\}$, and let
	$$
	\mathcal T = \{X\} \cup \mathcal T'.
	$$
	Then, all $x \in (X, \mathcal T)$ are closed, thus $(X, \mathcal T)$ is Fr\'echet. But the only neighbourhood of $p$ is $X$, so its closure is $X$. Then, for any $x \in X \setminus \{p\}$, $x$ and $p$ are not separated, in which case $(X, \mathcal T)$ is not $R_0$. Thus, $(X, \mathcal T)$ is not Hausdorff.
\end{example}
%--------------------------------
