%================================
\section{Separation Axioms}
%================================

% https://en.wikipedia.org/wiki/Separation_axiom

% https://en.wikipedia.org/wiki/Separated_sets
%--------------------------------
\begin{definition}
	[topologically indistinguishable]
	\label{def: topologically indistinguishable}
	Let $(X, \mathcal T)$ be a topological space. Two points $x,y \in X$ are said to be \textit{topologically indistinguishable} iff they share all their neighbourhoods. That is, let $\mathcal N_x$ be the family of all neighbourhoods of $x$ and let $\mathcal N_y$ be the family of all neibourhoods of $y$, we have
	$$
	\mathcal N_x = \mathcal N_y.
	$$
	
	Respectively, $x,y$ are said to be \textit{topologically distinguishable} iff they are not topologically distinguishable; i.e.,
	$$
	\mathcal N_x \ne \mathcal N_y.
	$$
\end{definition}
%--------------------------------


%--------------------------------
\begin{definition}
	[saperated sets]
	Let $(X, \mathcal T)$ be a topological space, and let $A, B \in \mathcal P(X)$.
	
	\begin{enumerate}[(i)]
		\item $A$ and $B$ are said to be \textit{separated} iff each is disjoint from other's closure.
		\item $A$ and $B$ are said to be \textit{separated by neighbourhoods} iff there are neighbourhoods $N_A$ of $A$ and $N_B$ of $B$ such that $N_A$ and $N_B$ are disjoint.
		\item $A$ and $B$ are said to be \textit{separated by closed neighbourhoods} iff there are closed neighbourhoods $\overline N_A$ of $A$ and $\overline N_B$ of $B$ such that $\overline N_A$ and $\overline N_B$ are disjoint.
		\item $A$ and $B$ are said to be \textit{separated by a continuous function} iff there is a continuous function $f: X \to \mathbb R$, such that $f[A] = \{0\}$ and $f[B] = \{1\}$.
		\item $A$ and $B$ are said to be \textit{precisely separated by a continuous function} iff there is a continuous function $f: X \to \mathbb R$, such that $f^{-1}[\{0\}] = A$ and $f^{-1}[\{1\}] = B$
	\end{enumerate}
\end{definition}
%--------------------------------


%--------------------------------
\begin{definition}
	[$T_0$ spaces]
	\label{def: T_1 spaces}
	A topological space $(X, \mathcal T)$ is said to be $T_0$ or \textit{Kolmogorov}, iff all distinct points $x,y \in X$ are \textit{topologically distinguishable}.
\end{definition}
%--------------------------------


%--------------------------------
\begin{example}
	[non-$T_0$ sets]
	\label{eg: non-T_0 sets}
	The a set $X$ with the discrete topology is $T_0$ iff $|X| \in \{0,1\}$ (vacuously true).
\end{example}
%--------------------------------


%--------------------------------
\begin{definition}
	[$R_0$ spaces]
	\label{def: R_0 spaces}
	A topological space $(X, \mathcal T)$ is said to be $R_0$ iff any two topologically distinguishable points in $X$ are separated.
\end{definition}
%--------------------------------


%--------------------------------
\begin{example}
	[$R_0$ but not $T_0$]
	\label{eg: R_0 but not T_0}
	Let $X$ be any set, let $U \subsetneq X$ with $|X \setminus U| > 1$, let $V \subsetneq U$ with $|V| = 1$, and let
	$$
	\mathcal T = \mathcal P(X) \setminus \mathcal P(X \setminus U) \setminus \mathcal P(V) \cup \{ \emptyset, X \}.
	$$
	For any two distinct points $x,y \in U$, the family $\mathcal N_x$ of neighbourhoods of $x$ and the family $\mathcal N_y$ of neighbourhoods of $y$ are different; and for all such $x$ and $y$, $x \notin \overline{ \{y\} } = \{ y \}$ (i.e., they are separated; but, be caution, they are not necessarily be separated by neighbourhoods; for if $y \in V$, the smallest neighbourhood of $y$ is $X$). Thus $(X, \mathcal T)$ is $R_0$. But $X$ is not $T_0$, because for two distinct points $x,y \in X \setminus U \cup V$, the families of their neighbourhoods are the same.
\end{example}
%--------------------------------


%--------------------------------
\begin{example}
	[$T_0$ but not $R_0$]
	\label{eg: T_0 but not T_1}
	Let $(\mathbb R_{\ge 0}, \mathcal T)$ be a topological space with
	$$
	\mathcal T = \left\{ U \subseteq \mathbb R : \forall i \in \mathbb R_{\ge 0}, \ U_i = [0, i)  \right\},
	$$
	Then for all $x,y \in (\mathbb R_{\ge 0}, \mathcal T)$, if $x \ne y$, then there are $|y - x|$ neighbourhoods $N_x$ of $x$ do not contain $y$. Thus, it is $T_0$.
	
	On the other hand, it is not $R_0$, because for all $x, y \in (\mathbb R_{\ge 0}, \mathcal T)$ with $x < y$, $x \in [0,y]$
\end{example}
%--------------------------------


%--------------------------------
\begin{proposition}
	[alternative definitions of $R_0$ spaces]
	\label{prop: alternative definitions of R_0 spaces}
	Let $(X, \mathcal T)$ be $R_0$, then the following conditions are equivalent.
	\begin{enumerate}[(i)]
		\item
		The closure of all singletons in $X$ are not $T_0$ subspace.
		
		\item
		For any two points $x,y \in X$, $x \in \overline{ \{y\} }$ iff $y \in \overline{ \{x\} }$.
		
		\item
		Every open set is the union of closed sets.
	\end{enumerate}
	
	% todo: prove them!
	\begin{proof}
		\
		\begin{enumerate}[(i)]
			\item
			By Definition \ref{def: R_0 spaces}, if $y$ and $x$ are topologically distinguishable, by Definition \ref{def: R_0 spaces}, $x$ and $y$ are separated; i.e., $x \notin \overline {\{y\}}$ and $y \notin \overline{\{x\}}$.
			
			\item
			By Definition \ref{def: R_0 spaces}, for all $x,y \in X$, $x,y$ are not separated only if they are topologically indistinguishable. By Definition \ref{def: topologically indistinguishable}, they share all their neighbourhoods, thus they have the same closure; i.e., $\overline{\{x\}} = \overline{\{y\}}$.
			
			\item
			(Remained as a problem!)
			% todo: Remained as a problem!
		\end{enumerate}
	\end{proof}
\end{proposition}
%--------------------------------



%--------------------------------
\begin{definition}
	[$T_1$ spaces]
	\label{def: T_1 spaces}
	A topological space $(X, \mathcal T)$ is said to be $T_1$ or \textit{Fr\'echet} iff it is $T_0$ and $R_0$.
\end{definition}
%--------------------------------


%--------------------------------
\begin{proposition}
	[alternative definitions of $T_1$ spaces]
	\label{prop: alternative definitions of T_1 spaces}
	Let $(X, \mathcal T)$ be $T_1$, then the following conditions are equivalent.
	\begin{enumerate}[(i)]
		\item All singletons in $X$ are closed.
		\item Every subset of $X$ is the intersection of all open sets containing it.
		\item Every cofinite subset of $X$ is open.
	\end{enumerate}
	
	% todo: prove them!
	\begin{proof}
		
	\end{proof}
\end{proposition}
%--------------------------------


%--------------------------------
\begin{proposition}
	\label{prop: all singletons in a T_1 space are closed}
	
	All singletons in a $T_1$ space are closed, That is, if a topological space $(X, \mathcal T)$ is $T_1$, then
	$$
	\forall x \in (X, \mathcal T) : \exists U \in \mathcal T : \{x\} = X \setminus U.
	$$
\end{proposition}
%--------------------------------


%--------------------------------
\begin{definition}
	[$R_1$ spaces]
	\label{def: R_1 spaces}
	A topological space $(X, \mathcal T)$ is said to be $R_1$ iff any two topological distinguishable points in $X$ are separated by neighbourhoods.
\end{definition}
%--------------------------------


%--------------------------------
\begin{example}
	[$R_0$ but not $R_1$]
	\label{eg: R_0 but not R_1}
	(Remained as a problem!)
	% todo: Remained as a problem
\end{example}
%--------------------------------


%--------------------------------
\begin{definition}
	[$T_2$ spaces]
	\label{def: T_2 spaces}
	A topological space $(X, \mathcal T)$ is said to be $T_2$ or \textit{Hausdorff} or \textit{separated} iff any two distinct points in $(X, \mathcal T)$ are separated by neighbourhoods.
\end{definition}
%--------------------------------

%--------------------------------
\begin{example}
	[$T_2$ but not $T_1$]
	\label{eg: T_2 but not T_1}
	Let $X$ be an nonempty set, and let $\mathcal U = \mathcal P(X \setminus \{x \in X\})$. Then the topological space $(X, \mathcal T)$ with
		$$
		\mathcal T = \mathcal U \cup \{X\}
		$$
		is $T_1$. But it is $T_2$ iff $|X| = 1$ (This is vacuously true). As $|X| > 1$, $\{x\}$ is not open.
\end{example}
%--------------------------------

%--------------------------------
\begin{proposition}
	All metric spaces are Hausdorff.
\end{proposition}
%--------------------------------


