%================================
\section{Continuity}
%================================

%--------------------------------
\begin{definition}
	[continuous maps]
	\label{def: continuous maps}
	Let $(X, \mathcal T_X)$ and $(Y, \mathcal T_Y)$ be topological spaces. A map $f: X \to Y$ is said to be \textit{continuous} iff for any open set $U$ in $Y$, its preimage in $X$ under $f$ is open.
\end{definition}
%--------------------------------


%--------------------------------
\begin{note}
	In Definition \ref{def: continuous maps}, note that even if for any open set $U$ in $X$, $f[X]$ is open in $Y$, $f$ is not necessarily continuous. For example, let $X = (\mathbb R, \mathcal T_X)$ with $\mathcal T_X$ induced by standard Euclidean metric, let $Y=(\mathbb R, \mathcal T_Y)$ with $\mathcal T_Y$ as a indiscrete topology, and define
	$$
	f(x) = [x],
	$$
	where $[x]$ denotes the integer part of $x$. Then for all $U \subseteq X$, $f[U]$ is open in $Y$, but by Definition \ref{def: continuous maps}, $f$ is not continuous.
\end{note}
%--------------------------------


%--------------------------------
\begin{note}
	Let $(X, \mathcal T_X)$ and $(Y, \mathcal T_Y)$ be topological spaces, if $\mathcal T_X$ is the discrete topology on $X$, then any function with domain $X$ is continuous. If $\mathcal T_Y$ is the indiscrete topology on $Y$, then any function with codomain $Y$ is continuous.
\end{note}
%--------------------------------


%--------------------------------
\begin{note}
	A function is continuous bijection does not implies that its inverse is continuous. For example, let $X$ be any set and let $\mathcal T$ and $\mathcal T'$ be its topologies. If $\mathcal T$ is finer than $\mathcal T'$, then any bijection $f: (X, \mathcal T) \to (X, \mathcal T')$ is continuous. In this case, however, if $\mathcal T \ne \mathcal T'$, then $f^{-1}$ is not continuous.
\end{note}
%--------------------------------


%--------------------------------
\begin{proposition}
	\label{prop: alt-def of continuous maps by neighbourhoods}
	Let $(X, \mathcal T_X)$ and $(Y, \mathcal T_Y)$ be topological spaces. A map $f: X \to Y$ is continuous at $x \in X$ iff for any neighbourhood $N_y$ of $f(x)$, there is a neighbourhood $N_x$ of $x$, such that $f[N_x] \subseteq N_y$.
	
	\begin{proof}
		Let $N_y$ be a neighbourhood of $f(x)$. Clearly, there exists an open set $U_y$ contains $y$.
	
		By Definition \ref{def: continuous maps}, $f$ is continuous at $x$ iff $x \in f^{-1}[U_y] \in \mathcal T_X$. Clearly, $f^{-1}[U_y]$ is a neighbourhood of $x$. We have $f[f^{-1}[U_y]] = U_y \subseteq N_y$.
		
		By Proposition \ref{prop: alt-def of open sets by neighbourhoods}, there $U_x$ must contains at least one neighbourhood $N_x$ of $x$, thus, $f[N_x] \subseteq U_y$.
	\end{proof}
\end{proposition}
%--------------------------------


%--------------------------------
\begin{proposition}
	Let $(X, \mathcal T_X)$ and $(Y, \mathcal T_{Y})$ be metrizable spaces. A map $f: X \to Y$ is continuous at $p \in X$ iff for any $\varepsilon > 0$, there is a $\delta > 0$, such that for all $x \in B_X(p, \delta)$, $f(x) \in B_Y(f(p), \varepsilon)$, where $B_X$ is defined by any metrics $\rho_X$ induces $\mathcal T_X$, and $B_Y$ is defined by any metrics $\rho_Y$ induces $\mathcal T_Y$.
	
	\begin{proof}
		Clearly, for all $\varepsilon > 0$, $B_Y(f(x,), \varepsilon)$ is an open neighbourhood of $f(x)$.
		
		$f$ is not necessarily be injective, so $f^{-1}[B_Y(f(x), \varepsilon)] = U \in x$. By Definition \ref{def: continuous maps}, $U$ is open, so for some $\delta > 0$, $B_X(x, \delta) \subseteq U$. Thus, By Proposition \ref{prop: alt-def of continuous maps by neighbourhoods}, $f$ is continuous iff $f[B_X(x, \delta)] \subseteq B_Y(f(x), \varepsilon)$. This satisfies the conditions we have.
	\end{proof}
\end{proposition}
%--------------------------------


%--------------------------------
\begin{proposition}
	Let $(X, \mathcal T_X)$ and $(Y, \mathcal T_Y)$ be topological spaces. A function $f: X \to Y$ is said to be continuous iff for any closed set $V$ in $Y$, its preimage in $X$ under $f$ is closed.
	
	\begin{proof}
		Let $U_Y$ be any open set in $Y$, let $U_X$ be the preimage of $U_Y$ under $f$. By Definition \ref{def: continuous maps}, $U_X$ is open in $X$. Let
		$$
		V_X = f^{-1}[Y \setminus U_Y] = X \setminus U_X,
		$$
		Then $V_X$ is closed.
	\end{proof}
\end{proposition}
%--------------------------------


%--------------------------------
\begin{definition}
	[convergence of sequences]
	\label{def: convergence of sequences}
	Let $(X, \mathcal T)$ be a topological space, and let $\{x_n\}$ be a sequence in $X$. Then $\{x_n\}$ is said to be \textit{converges} in $X$ iff there is an $x \in X$, such that for any open neighbourhood $U_x$ of $x$, it contains a cofinite subset $A \subseteq \{x_n\}$. That is, there exists $N$ in the domain of $\{x_n\}$, for any natural numbers $n \ge N$, $x_n \in U_x$.
\end{definition}
%--------------------------------


%--------------------------------
\begin{example}
	\
	\begin{enumerate}
		\item
		In a discrete topological space, a sequence $\{x_n\}$ converges iff there is an $N$ in the domain of $\{x_n\}$, for any natural numbers $m > N$, $x_N = x_m$.
		
		\item
		In a indiscrete topological space, any sequence $\{x_n\}$ in $X$ converges in $X$. And
		$$
		\lim_{n \to \infty} \{x_n\} = X.
		$$
	\end{enumerate}
\end{example}
%--------------------------------


%--------------------------------
\begin{proposition}
	In a Hausdorff space, any convergent sequence converges to a unique point in the space.
	
	\begin{proof}
		Let $(X, \mathcal T)$ be a Hausdorff space, and let $\{x_n\}$ be a sequence in $X$. Suppose $\{x_n\}$ converges to more than one point, say to $x, y \in X$ with $x \ne y$, then, for all neighbourhoods $N_x$ of $x$ and $N_y$ of $y$, $N_x$ contains a cofinite subset $A \subseteq \{x_n\}$ and $N_y$ contains a cofinite subset $B \subseteq \{x_n\}$. If this were true, $N_x \cap N_y$ should be non-empty, otherwise $N_x$ or $N_y$ should be finite.
		
		Then, $x$ and $y$ are not separated by neighbourhoods, thus $(X, \mathcal T)$ is not Hausdorff. This is a contradiction.
		
		But, as $(X, \mathcal T)$ is Hausdorff, there must be mutually disjoint $N_x$ and $N_y$. Thus, the assumption cause a contradiction.
	\end{proof}
\end{proposition}
%--------------------------------


%--------------------------------
\begin{note}
	As all metrizable spaces are Hausdorff, so any convergent sequence in a metrizable space converges to at most one point.
\end{note}
%--------------------------------


%--------------------------------
\begin{proposition}
	Let $(X, \mathcal T_X)$ and $(Y, \mathcal T_Y)$ be topological space, let $f: X \to Y$ be a map, and let $\{x_n\}$ be a convergent sequence in $X$. If $f$ is continuous, then $f[\{x_n\}]$ is a sequence convergent in $Y$.
	
	\begin{proof}
		Let $U_y$ be any open neighbourhood of $f(x)$. By Definition \ref{def: continuous maps}, $f^{-1}[U_y]$ is also an open neighbourhood of $x$. By Definition \ref{def: convergence of sequences}, $f^{-1}[U_y]$ contains a cofinite subset $A \subseteq \{x_n\}$. Then $f[A]$ is a cofinite subset of $f[\{x_n\}]$. As $f[f^{-1}[U_y]] \supseteq f^{-1}[A]$, $f[\{x_n\}]$ converges in $f[f^{-1}[U_y]] \supseteq f^{-1}[A]$.
	\end{proof}		
\end{proposition}
%--------------------------------


%--------------------------------
\begin{note}
	In this proposition, even if $f[\{x_n\}]$ converges in $Y$, $f$ might be discontinuous. For example, let $X$ any set, let $\mathcal T$ be the indiscrete topology on $X$, let $U$ be another cofinite subset of $X$ with $X \ne U$, and let $\mathcal T' = \{ \emptyset, X, U\}$. Let $f: (X, \mathcal T) \to (X, \mathcal T')$ be defined by
	$$
	f(x) = x.
	$$
	
	By Definition \ref{def: continuous maps}, $f$ is not continuous. But, for any convergent sequence $\{x_n\}$ in $(X, \mathcal T)$, $f[\{x_n\}]$ also convergent in $(X, \mathcal T)$.
\end{note}
%--------------------------------





































%