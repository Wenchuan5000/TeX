%================================
%::::::::::::::::::::::::::::::::
\chapter{Metric Spaces}
%::::::::::::::::::::::::::::::::
%================================



%================================
%::::::::::::::::::::::::::::::::
\chapter{Topological Spaces}
%::::::::::::::::::::::::::::::::
%================================


%================================
\section{Topological Spaces}
%================================



%--------------------------------
\begin{definition}
	[topology]
	\label{def: topology}
	Let $X$ be a set, and let a family $\mathcal T \subseteq \mathcal P(X)$. $\mathcal T$ is called a topology on $X$ iff
	\begin{enumerate}[(i)]
		\item $\emptyset, X \in \mathcal T$;
		\item $\mathcal T$ is closed under arbitrary union;
		\item $\mathcal T$ is closed under finite intersection.
	\end{enumerate}
	The pair $(X, \mathcal T)$ is called a \textit{topological space}. The elements of $\mathcal T$ are called \textit{open sets} in $(X, \mathcal T)$.
\end{definition}
%--------------------------------


%--------------------------------
\begin{definition}
	[topological spaces]
	\label{def: topological spaces}
	Let $X$ be any set, and let $\mathcal T$ be a topology on $X$, then the pair $(X, \mathcal T)$ is called a \textit{topological space}. All subsets of $X$ in $\mathcal T$ are called \textit{open sets} of $(X, \mathcal T)$.
\end{definition}
%--------------------------------


%--------------------------------
\begin{definition}
	[finer and coarser topology]
	\label{def: finer and coarser topology} Let $X$ be any set, and let $\mathcal T, \mathcal T'$ be topologies on $X$. $\mathcal T$ is said to be \textit{finer} than $\mathcal T'$ iff $\mathcal T \supseteq \mathcal T'$; respectively, $\mathcal T$ is said to be \textit{coarser} than $\mathcal T'$ iff $\mathcal T \subseteq \mathcal T'$.
\end{definition}
%--------------------------------


%--------------------------------
\begin{definition}
	[neighbourhood]
	\label{def: neighbourhood}
	Given $(X, \mathcal T)$ as a topological space and a point $x \in X$, a subset $N \subseteq X$ is called a \textit{neighbourhood} iff it contains an open set $U$ containing $x$.
\end{definition}
%--------------------------------


%--------------------------------
\begin{proposition}
	Given $(X, \mathcal T)$ as a topological space and $U \subseteq X$, $U$ is open iff for all $x \in U$, there is a neighbourhood $N$ of $x$ contained in $U$.
	
	\begin{proof}
		If $U$ is open, then $U$ itself is a neighbourhood of $x$ contained in $U$.
		
		Conversely, if for all $x \in U$, there is a neighbourhood $N_x$ of $x$ contained in $U$, then there is a open neighbourhood $U_x \ni x$ contained in $N_x$. Then we have
		$$
		U \supseteq \bigcup_{x \in U} U_x.
		$$
		Suppose $U$ is not open, then $U$ is a proper superset in the relation above. Then there exists $y \in U$ which is not in any $U_x$. This implies that such a $y$ does not have any neighbourhood $N_y$ in $U$, for such an $N_y$ must contains an open $U_y \ni y$. For if it does, then there must be a $U_x$ contains $y$. This is a contradiction. Thus,
		$$
		U = \bigcup_{x \in U} U_x
		$$
		is open.
	\end{proof}
\end{proposition}
%--------------------------------





%================================
\section{Untitled}
%================================


%--------------------------------
\begin{definition}
	[cover]
	\label{def: cover}
	Let $(X, \mathcal T)$ be a topological space, and let $U \subseteq X$, then a family $\mathcal C \subseteq \mathcal P(X)$ is called a \textit{cover} of $U$ iff the union of $\mathcal C$ is a superset of $U$. That is,
	$$
	U \subseteq \bigcup \mathcal C.
	$$
	
	If $\mathcal C \subseteq \mathcal T$, then we call $\mathcal C$ an \textit{open cover} of $U$.
	
	Let $\mathcal C' \subseteq \mathcal C$, iff the union of $\mathcal C'$ is still a superset of $U$, then we call $\mathcal C'$ a subcover of $\mathcal C$.
\end{definition}
%--------------------------------


%--------------------------------
\begin{definition}
	[basis]
	\label{def: basis}
	Let $(X, \mathcal T)$ be a topological space, let $U \subseteq X$, and let $\mathcal B \subseteq \mathcal P(X)$ be a cover of $X$. We call $\mathcal B$ a \textit{base} of $(X, \mathcal T)$ iff $\mathcal B \subseteq \mathcal T$ and the union of $\mathcal B$ is exactly $U$ itself. That is,
	$$
	\mathcal B \subseteq \mathcal T, \text{ and } U = \bigcup \mathcal B.
	$$
\end{definition}
%--------------------------------


%--------------------------------
\begin{definition}
	[subspace topology]
	\label{def: subspace topology}
	Let $(X, \mathcal T)$ be a topological space and let $A \subseteq X$. The \textit{subspace topology} $\mathcal T_A$ on $A$ is defined to be the family of the intersections of open sets in $(X, \mathcal T)$ and $A$. That is,
	$$
	\mathcal T_A = \left\{ U \cap A : \ U \in \mathcal T \right\}.
	$$
\end{definition}
%--------------------------------


%--------------------------------
\begin{definition}
	[quotient topology]
	\label{def: quotient topology}
	Let $(X, \mathcal T)$ be a topological space and let $\sim$ be an equivalence relation on $X$. The \textit{quotient topology} is a topology on $\mathcal P(X/ \sim)$; it is defined as
	$$
	\mathcal T_{X / \sim} = \left\{ U \in \mathcal P(X/\sim) : \{ x \in X: [x] \in U \} \in \mathcal T_X \right\}.
	$$
\end{definition}
%--------------------------------


%--------------------------------
\begin{definition}
	[continuous functions]
	\label{def: continuous functions}
	Let $(X, \mathcal T_X)$ and $(Y, \mathcal T_Y)$ be topological spaces. A function $f: X \to Y$ is said to be \textit{continuous} iff for all open subset $U$ of $Y$, the preimage $f^{-1}[U]$ is open in $X$. That is,
	$$
	\forall U \in \mathcal T_{Y} : f^{-1}[U] \in \mathcal T_X.
	$$
\end{definition}
%--------------------------------


%--------------------------------
\begin{definition}
	[homeomorphisms]
	\label{def: homomorphisms}
	Let $(X, \mathcal T_X)$ and $(Y, \mathcal T_Y)$ be topological spaces. A bijection $f: X \to Y$ is called a \textit{homeomorphism} iff it is continuous and its inverse is also continuous.
\end{definition}
%--------------------------------


%--------------------------------
\begin{definition}
	[homeomorphic]
	\label{def: homomorphisms}
	Two topological spaces $(X, \mathcal T_X)$ and $(Y, \mathcal T_Y)$ are said to be \textit{homeomorphic} or \textit{topologically equivalent}, denoted $X \cong Y$, iff there is an homeomorphism between them.
\end{definition}
%--------------------------------


%--------------------------------
\begin{definition}
	[compactness]
	\label{def: compactness}
	A topological space $(X, \mathcal T)$ is said to be \textit{compact} iff every open cover of $X$ has a finite subcover. That is,
	$$
	\forall \mathcal C \subseteq \mathcal T : \bigcup \mathcal C = X : \exists \mathcal S \subseteq \mathcal C : \bigcup \mathcal S = X : |\mathcal S| < \aleph_0.
	$$
\end{definition}
%--------------------------------


%--------------------------------
\begin{definition}
	[connectedness]
	\label{def: connectedness}
	Let $(X, \mathcal T)$ be a topological space. $(X, \mathcal T)$ is said to be \textit{connected} iff $X$ is not empty and it it not the union of any disjoint open sets. That is,
	$$
	\forall U, V \in \mathcal T : X = U \cup V : U \cap V \ne \emptyset.
	$$
\end{definition}
%--------------------------------


%--------------------------------
\begin{definition}
	[path-connectedness]
	\label{def: path-connectedness}
	Let $(X, \mathcal T)$ be a topological space.
	\begin{enumerate}[(i)]
		\item A map $\gamma: [0,1] \to X$ is called a \textit{path} in $X$ iff it is continuous. If $\gamma(0) = x$ and $\gamma(1)=y$, we say that $\gamma$ is path from $x$ to $y$ in $X$.
		\item $X$ is said to be \textit{path-connected} iff for all $x, y \in X$ there is a path from $x$ to $y$ in $X$.
	\end{enumerate}
\end{definition}
%--------------------------------


%--------------------------------
\begin{definition}
	[topologically indistinguishable]
	\label{def: topologically indistinguishable}
	Let $(X, \mathcal T)$ be a topological space. Two points $x,y \in X$ are said to be \textit{topologically indistinguishable} iff they share all their neighbourhoods. That is, let $\mathcal N_x$ be the family of all neighbourhoods of $x$ and let $\mathcal N_y$ be the family of all neibourhoods of $y$, we have
	$$
	\mathcal N_x = \mathcal N_y.
	$$
	
	Respectively, $x,y$ are said to be \textit{topologically distinguishable} iff they are not topologically distinguishable; i.e.,
	$$
	\mathcal N_x \ne \mathcal N_y.
	$$
\end{definition}
%--------------------------------


%--------------------------------
\begin{definition}
	[saperated sets]
	Let $(X, \mathcal T)$ be a topological space, and let $A, B \in \mathcal P(X)$.
	
	\begin{enumerate}[(i)]
		\item $A$ and $B$ are said to be \textit{separated} iff each is disjoint from other's closure.
		\item $A$ and $B$ are said to be \textit{separated by neighbourhoods} iff there are neighbourhoods $N_A$ of $A$ and $N_B$ of $B$ such that $N_A$ and $N_B$ are disjoint.
		\item $A$ and $B$ are said to be \textit{separated by closed neighbourhoods} iff there are closed neighbourhoods $\overline N_A$ of $A$ and $\overline N_B$ of $B$ such that $\overline N_A$ and $\overline N_B$ are disjoint.
		\item $A$ and $B$ are said to be \textit{separated by a continuous function} iff there is a continuous function $f: X \to \mathbb R$, such that $f[A] = \{0\}$ and $f[B] = \{1\}$.
		\item $A$ and $B$ are said to be \textit{precisely separated by a continuous function} iff there is a continuous function $f: X \to \mathbb R$, such that $f^{-1}[\{0\}] = A$ and $f^{-1}[\{1\}] = B$
	\end{enumerate}
\end{definition}
%--------------------------------


\href{https://en.wikipedia.org/wiki/Separated_sets}{See Wikipedia.org}


%--------------------------------
\begin{definition}
	[$T_0$ spaces]
	\label{def: T_1 spaces}
	A topological space $(X, \mathcal T)$ is said to be $T_0$ or \textit{Kolmogorov}, iff all distinct points $x,y \in X$ are \textit{topologically distinguishable}.
\end{definition}
%--------------------------------


%--------------------------------
\begin{definition}
	[$R_0$ spaces]
	\label{def: R_0 spaces}
	A topological space $(X, \mathcal T)$ is said to be $R_0$ iff any two topologically distinguishable points in $X$ are separated.
\end{definition}
%--------------------------------


%--------------------------------
\begin{definition}
	[$T_1$ spaces]
	\label{def: T_1 spaces}
	A topological space $(X, \mathcal T)$ is said to be $T_1$ or \textit{Fr\'echet} iff any two distinct points in $X$ are separated.
\end{definition}
%--------------------------------


%--------------------------------
\begin{proposition}
	\label{prop: all singletons in a T_1 space are closed}
	
	All singletons in a $T_1$ space are closed, That is, if a topological space $(X, \mathcal T)$ is $T_1$, then
	$$
	\forall x \in (X, \mathcal T) : \exists U \in \mathcal T : \{x\} = X \setminus U.
	$$
\end{proposition}
%--------------------------------


%--------------------------------
\begin{definition}
	[$T_2$ spaces]
	\label{def: T_2 spaces}
	A topological space $(X, \mathcal T)$ is said to be $T_2$ or \textit{Hausdorff} or \textit{separated} iff any two distinct points in $(X, \mathcal T)$ are separated by neighbourhoods.
\end{definition}
%--------------------------------


%--------------------------------
\begin{definition}
	[$T_{2 \nicefrac{1}{2}}$ spaces]
	\label{def: T_2.5 spaces}
	A topological space $(X, \mathcal T)$ is said to be $T_{2 \nicefrac{1}{2}}$ or \textit{Urysohn} iff two distinct points in $X$ are separated by closed neighbourhoods.
\end{definition}
%--------------------------------


%--------------------------------
\begin{definition}
	[$T_3$ spaces]
	\label{def: T_3 spaces}
	A topological space $(X, \mathcal T)$ is said to be $T_3$ or \textit{regular} iff it is $T_0$ and given any point $x \in (X, \mathcal T)$ and closed set $V \subseteq X$ with $x \notin V$ are separated by neighbourhoods.
\end{definition}
%--------------------------------


%--------------------------------
\begin{definition}
	[$T_{3\nicefrac{1}{2}}$ spaces]
	\label{def: T_3.5 spaces}
	A topological space $(X, \mathcal T)$ is said to be $T_{3 \nicefrac{1}{2}}$, or \textit{Tychonoff} or, \textit{completely $T_3$}, or \textit{completely regular}, iff it is $T_0$ and given any point $x$ and closed set $V \subseteq X$ with $x \notin V$, they are separated by a continuous function.
\end{definition}
%--------------------------------


%--------------------------------
\begin{definition}
	[$T_4$ spaces]
	\label{def: T_4 spaces}
	A topological space $(X, \mathcal T)$ is said to be $T_4$ or \textit{normal} iff it is Hausdorff and any tow disjoint closed subsets of $X$ are separated by neighbourhoods.
\end{definition}
%--------------------------------


%--------------------------------
\begin{proposition}
	[Urysohn's lemma]
	\label{prop: urysohn's lemma}
	A topological space is normal iff any two disjoint closed sets are separated by a continuous function.
\end{proposition}
%--------------------------------


%--------------------------------
\begin{definition}
	[$T_5$ spaces]
	\label{def: T_5 spaces}
	A topological space $(X, \mathcal T)$ is said to be $T_5$ or \textit{completely $T_4$} iff it is $T_1$ any two separated sets are separated by neighbourhoods.
\end{definition}
%--------------------------------


%--------------------------------
\begin{proposition}
	Every subspace of a $T_5$ space is normal.
\end{proposition}
%--------------------------------


%--------------------------------
\begin{definition}
	[$T_6$ spaces]
	\label{def: T_6 spaces}
	A topological space $(X, \mathcal T)$ is said to be $T_6$, or \textit{perfectly $T_4$} or \textit{perfectly normal} iff it is $T_1$ and any two disjoint closed sets are precisely separated by a continuous function.
\end{definition}
%--------------------------------



%--------------------------------
\begin{proposition}
	[Tietze extension theorem]
	\label{prop: Tietze extension theorem}
	Let $(X, \mathcal T)$ be normal topological space, and let $f: A \to (\mathbb R, \mathcal T')$ be a continuous map where $A$ is a closed subset of $X$ and $\mathcal T'$ is the standard topology (induced by Euclidean metric). Then there exists a continuous map
	$$
	F: (X, \mathcal T) \to (\mathbb R, \mathcal T'),
	$$
	such that
	$$
	\forall x \in A: f(x) = g(x).
	$$
\end{proposition}
%--------------------------------







































%