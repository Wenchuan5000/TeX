%================================
%::::::::::::::::::::::::::::::::
\chapter{Metric Spaces}
%::::::::::::::::::::::::::::::::
%================================



%================================
%::::::::::::::::::::::::::::::::
\chapter{Topological Spaces}
%::::::::::::::::::::::::::::::::
%================================


%================================
\section{Topological Spaces}
%================================



%--------------------------------
\begin{definition}
	[topology]
	\label{def: topology}
	Let $X$ be a set, and let a family $\mathcal T \subseteq \mathcal P(X)$. $\mathcal T$ is called a topology on $X$ iff
	\begin{enumerate}[(i)]
		\item $\emptyset, X \in \mathcal T$;
		\item $\mathcal T$ is closed under arbitrary union;
		\item $\mathcal T$ is closed under finite intersection.
	\end{enumerate}
\end{definition}
%--------------------------------


%--------------------------------
\begin{definition}
	[topological spaces]
	\label{def: topological spaces}
	Let $X$ be any set, and let $\mathcal T$ be a topology on $X$, then the pair $(X, \mathcal T)$ is called a \textit{topological space}. All subsets of $X$ in $\mathcal T$ are called \textit{open sets} in $(X, \mathcal T)$.
\end{definition}
%--------------------------------


%--------------------------------
\begin{definition}
	[closed sets]
	\label{def: closed sets}
	Let $(X, \mathcal T)$ be a topological space. A subset $V$ of $X$ is said to be \textit{closed} iff there is an open set $U$ in $X$ such that
	$$
	V = X \setminus V.
	$$
\end{definition}
%--------------------------------


%--------------------------------
\begin{proposition}
	\label{prop: dark side of topology}
	Let $X$ be a set, and let $\mathcal C$ be the family of all closed sets in $X$. Then
	\begin{enumerate}[(i)]
		\item $\emptyset, X \in \mathcal C$;
		\item $\mathcal C$ is closed under arbitrary intersection;
		\item $\mathcal C$ is closed under finite union.
	\end{enumerate}
\end{proposition}
%--------------------------------


%--------------------------------
\begin{definition}
	[finer and coarser topology]
	\label{def: finer and coarser topology} Let $X$ be any set, and let $\mathcal T, \mathcal T'$ be topologies on $X$. $\mathcal T$ is said to be \textit{finer} than $\mathcal T'$ iff $\mathcal T \supseteq \mathcal T'$; respectively, $\mathcal T$ is said to be \textit{coarser} than $\mathcal T'$ iff $\mathcal T \subseteq \mathcal T'$.
\end{definition}
%--------------------------------


%--------------------------------
\begin{definition}
	[neighbourhood]
	\label{def: neighbourhood}
	Given $(X, \mathcal T)$ as a topological space and a point $x \in X$, a subset $N \subseteq X$ is called a \textit{neighbourhood} iff it contains an open set $U$ containing $x$.
\end{definition}
%--------------------------------


%--------------------------------
\begin{proposition}
	\label{prop: alt-def of open sets by neighbourhoods}
	Given $(X, \mathcal T)$ as a topological space and $U \subseteq X$, $U$ is open iff for all $x \in U$, there is a neighbourhood $N$ of $x$ contained in $U$.
	
	\begin{proof}
		If $U$ is open, then $U$ itself is a neighbourhood of $x$ contained in $U$.
		
		Conversely, if for all $x \in U$, there is a neighbourhood $N_x$ of $x$ contained in $U$, then there is a open neighbourhood $U_x \ni x$ contained in $N_x$. Then we have
		$$
		U \supseteq \bigcup_{x \in U} U_x.
		$$
		Suppose $U$ is not open, then $U$ is a proper superset in the relation above. Then there exists $y \in U$ which is not in any $U_x$. This implies that such a $y$ does not have any neighbourhood $N_y$ in $U$, for such an $N_y$ must contains an open $U_y \ni y$. For if it does, then there must be a $U_x$ contains $y$. This is a contradiction. Thus,
		$$
		U = \bigcup_{x \in U} U_x
		$$
		is open.
	\end{proof}
\end{proposition}
%--------------------------------


%================================
\section{Metrizable Spaces}
%================================


%================================
\section{Continuity}
%================================


%--------------------------------
\begin{definition}
	[continuous maps]
	\label{def: continuous maps}
	Let $(X, \mathcal T_X)$ and $(Y, \mathcal T_Y)$ be topological spaces. A map $f: X \to Y$ is said to be \textit{continuous} iff for any open set $U$ in $Y$, its preimage in $X$ under $f$ is open.
\end{definition}
%--------------------------------


%--------------------------------
\begin{proposition}
	\label{prop: alt-def of continuous maps by neighbourhoods}
	Let $(X, \mathcal T_X)$ and $(Y, \mathcal T_Y)$ be topological spaces. A map $f: X \to Y$ is continuous at $x \in X$ iff for any neighbourhood $N_y$ of $f(x)$, there is a neighbourhood $N_x$ of $x$, such that $f[N_x] \subseteq N_y$.
	
	\begin{proof}
		Let $N_y$ be a neighbourhood of $f(x)$. Clearly, there exists an open set $U_y$ contains $y$.
	
		By Definition \ref{def: continuous maps}, $f$ is continuous at $x$ iff $x \in f^{-1}[U_y] \in \mathcal T_X$. Clearly, $f^{-1}[U_y]$ is a neighbourhood of $x$. We have $f[f^{-1}[U_y]] = U_y \subseteq N_y$.
		
		By Proposition \ref{prop: alt-def of open sets by neighbourhoods}, there $U_x$ must contains at least one neighbourhood $N_x$ of $x$, thus, $f[N_x] \subseteq U_y$.
	\end{proof}
\end{proposition}
%--------------------------------


%--------------------------------
\begin{proposition}
	Let $(X, \mathcal T_X)$ and $(Y, \mathcal T_{Y})$ be metrizable spaces. A map $f: X \to Y$ is continuous at $p \in X$ iff for any $\varepsilon > 0$, there is a $\delta > 0$, such that for all $x \in B_X(p, \delta)$, $f(x) \in B_Y(f(p), \varepsilon)$, where $B_X$ is defined by any metrics $\rho_X$ induces $\mathcal T_X$, and $B_Y$ is defined by any metrics $\rho_Y$ induces $\mathcal T_Y$.
	
	\begin{proof}
		Clearly, for all $\varepsilon > 0$, $B_Y(f(x,), \varepsilon)$ is an open neighbourhood of $f(x)$.
		
		$f$ is not necessarily be injective, so $f^{-1}[B_Y(f(x), \varepsilon)] = U \in x$. By Definition \ref{def: continuous maps}, $U$ is open, so for some $\delta > 0$, $B_X(x, \delta) \subseteq U$. Thus, By Proposition \ref{prop: alt-def of continuous maps by neighbourhoods}, $f$ is continuous iff $f[B_X(x, \delta)] \subseteq B_Y(f(x), \varepsilon)$. This satisfies the conditions we have.
	\end{proof}
\end{proposition}
%--------------------------------


%--------------------------------
\begin{proposition}
	Let $(X, \mathcal T_X)$ and $(Y, \mathcal T_Y)$ be topological spaces. A function $f: X \to Y$ is said to be continuous iff for any closed set $V$ in $Y$, its preimage in $X$ under $f$ is closed.
	
	\begin{proof}
		Let $U_Y$ be any open set in $Y$, let $U_X$ be the preimage of $U_Y$ under $f$. By Definition \ref{def: continuous maps}, $U_X$ is open in $X$. Let
		$$
		V_X = f^{-1}[Y \setminus U_Y] = X \setminus U_X,
		$$
		Then $V_X$ is closed.
	\end{proof}
\end{proposition}
%--------------------------------



%================================
\section{Cover}
%================================


%--------------------------------
\begin{definition}
	[cover]
	\label{def: cover}
	Let $(X, \mathcal T)$ be a topological space, and let $U \subseteq X$, then a family $\mathcal C \subseteq \mathcal P(X)$ is called a \textit{cover} of $U$ iff the union of all sets in $\mathcal C$ is a superset of $U$. That is,
	$$
	U \subseteq \bigcup \mathcal C.
	$$
	
	If $\mathcal C \subseteq \mathcal T$, then we call $\mathcal C$ an \textit{open cover} of $U$.
	
	Let $\mathcal S \subseteq \mathcal C$, iff the union of $\mathcal S$ is still a superset of $U$, then we call $\mathcal S$ a \textit{subcover} of $\mathcal C$.
\end{definition}
%--------------------------------


%--------------------------------
\begin{definition}
	[basis]
	\label{def: basis}
	Let $(X, \mathcal T)$ be a topological space, let $U \subseteq X$, and let $\mathcal B$ be a open cover of $X$. We call $\mathcal B$ a \textit{base} of $X$ iff the union of $\mathcal B$ is precisely $U$ itself, i.e.,
	$$
	U = \bigcup \mathcal B.
	$$
\end{definition}
%--------------------------------


%--------------------------------
\begin{definition}
	[synthetic basis]
	\label{def: synthetic basis}
	Let $(X, \mathcal T)$ be a topological space, and let $\mathcal B$ be a base of $X$. $\mathcal B$ is said to be \textit{synthetic} iff for any $A, B \in \mathcal B$,
	$$
	A \cap B = \bigcup_{i = 1}^{n} B_i, \quad B_i \in \mathcal B.
	$$
\end{definition}
%--------------------------------


%================================
\section{Untitled}
%================================


%--------------------------------
\begin{definition}
	[subspace topology]
	\label{def: subspace topology}
	Let $(X, \mathcal T)$ be a topological space and let $A \subseteq X$. The \textit{subspace topology} $\mathcal T_A$ on $A$ is defined to be the family of the intersections of open sets in $(X, \mathcal T)$ and $A$. That is,
	$$
	\mathcal T_A = \left\{ U \cap A : \ U \in \mathcal T \right\}.
	$$
\end{definition}
%--------------------------------


%--------------------------------
\begin{definition}
	[quotient topology]
	\label{def: quotient topology}
	Let $(X, \mathcal T)$ be a topological space and let $\sim$ be an equivalence relation on $X$. The \textit{quotient topology} is a topology on $\mathcal P(X/ \sim)$; it is defined as
	$$
	\mathcal T_{X / \sim} = \left\{ U \in \mathcal P(X/\sim) : \{ x \in X: [x] \in U \} \in \mathcal T_X \right\}.
	$$
\end{definition}
%--------------------------------


%--------------------------------
\begin{definition}
	[homeomorphisms]
	\label{def: homomorphisms}
	Let $(X, \mathcal T_X)$ and $(Y, \mathcal T_Y)$ be topological spaces. A bijection $f: X \to Y$ is called a \textit{homeomorphism} iff it is continuous and its inverse is also continuous.
\end{definition}
%--------------------------------


%--------------------------------
\begin{definition}
	[homeomorphic]
	\label{def: homomorphisms}
	Two topological spaces $(X, \mathcal T_X)$ and $(Y, \mathcal T_Y)$ are said to be \textit{homeomorphic} or \textit{topologically equivalent}, denoted $X \cong Y$, iff there is an homeomorphism between them.
\end{definition}
%--------------------------------


%--------------------------------
\begin{definition}
	[compactness]
	\label{def: compactness}
	A topological space $(X, \mathcal T)$ is said to be \textit{compact} iff every open cover of $X$ has a finite subcover. That is,
	$$
	\forall \mathcal C \subseteq \mathcal T : \bigcup \mathcal C = X : \exists \mathcal S \subseteq \mathcal C : \bigcup \mathcal S = X : |\mathcal S| < \aleph_0.
	$$
\end{definition}
%--------------------------------


%--------------------------------
\begin{definition}
	[connectedness]
	\label{def: connectedness}
	Let $(X, \mathcal T)$ be a topological space. $(X, \mathcal T)$ is said to be \textit{connected} iff $X$ is not empty and it it not the union of any disjoint open sets. That is,
	$$
	\forall U, V \in \mathcal T : X = U \cup V : U \cap V \ne \emptyset.
	$$
\end{definition}
%--------------------------------


%--------------------------------
\begin{definition}
	[path-connectedness]
	\label{def: path-connectedness}
	Let $(X, \mathcal T)$ be a topological space.
	\begin{enumerate}[(i)]
		\item A map $\gamma: [0,1] \to X$ is called a \textit{path} in $X$ iff it is continuous. If $\gamma(0) = x$ and $\gamma(1)=y$, we say that $\gamma$ is path from $x$ to $y$ in $X$.
		\item $X$ is said to be \textit{path-connected} iff for all $x, y \in X$ there is a path from $x$ to $y$ in $X$.
	\end{enumerate}
\end{definition}
%--------------------------------


%--------------------------------
\begin{definition}
	[topologically indistinguishable]
	\label{def: topologically indistinguishable}
	Let $(X, \mathcal T)$ be a topological space. Two points $x,y \in X$ are said to be \textit{topologically indistinguishable} iff they share all their neighbourhoods. That is, let $\mathcal N_x$ be the family of all neighbourhoods of $x$ and let $\mathcal N_y$ be the family of all neibourhoods of $y$, we have
	$$
	\mathcal N_x = \mathcal N_y.
	$$
	
	Respectively, $x,y$ are said to be \textit{topologically distinguishable} iff they are not topologically distinguishable; i.e.,
	$$
	\mathcal N_x \ne \mathcal N_y.
	$$
\end{definition}
%--------------------------------


%--------------------------------
\begin{definition}
	[saperated sets]
	Let $(X, \mathcal T)$ be a topological space, and let $A, B \in \mathcal P(X)$.
	
	\begin{enumerate}[(i)]
		\item $A$ and $B$ are said to be \textit{separated} iff each is disjoint from other's closure.
		\item $A$ and $B$ are said to be \textit{separated by neighbourhoods} iff there are neighbourhoods $N_A$ of $A$ and $N_B$ of $B$ such that $N_A$ and $N_B$ are disjoint.
		\item $A$ and $B$ are said to be \textit{separated by closed neighbourhoods} iff there are closed neighbourhoods $\overline N_A$ of $A$ and $\overline N_B$ of $B$ such that $\overline N_A$ and $\overline N_B$ are disjoint.
		\item $A$ and $B$ are said to be \textit{separated by a continuous function} iff there is a continuous function $f: X \to \mathbb R$, such that $f[A] = \{0\}$ and $f[B] = \{1\}$.
		\item $A$ and $B$ are said to be \textit{precisely separated by a continuous function} iff there is a continuous function $f: X \to \mathbb R$, such that $f^{-1}[\{0\}] = A$ and $f^{-1}[\{1\}] = B$
	\end{enumerate}
\end{definition}
%--------------------------------


\href{https://en.wikipedia.org/wiki/Separated_sets}{See Wikipedia.org}


%--------------------------------
\begin{definition}
	[$T_0$ spaces]
	\label{def: T_1 spaces}
	A topological space $(X, \mathcal T)$ is said to be $T_0$ or \textit{Kolmogorov}, iff all distinct points $x,y \in X$ are \textit{topologically distinguishable}.
\end{definition}
%--------------------------------


%--------------------------------
\begin{definition}
	[$R_0$ spaces]
	\label{def: R_0 spaces}
	A topological space $(X, \mathcal T)$ is said to be $R_0$ iff any two topologically distinguishable points in $X$ are separated.
\end{definition}
%--------------------------------


%--------------------------------
\begin{definition}
	[$T_1$ spaces]
	\label{def: T_1 spaces}
	A topological space $(X, \mathcal T)$ is said to be $T_1$ or \textit{Fr\'echet} iff any two distinct points in $X$ are separated.
\end{definition}
%--------------------------------


%--------------------------------
\begin{proposition}
	\label{prop: all singletons in a T_1 space are closed}
	
	All singletons in a $T_1$ space are closed, That is, if a topological space $(X, \mathcal T)$ is $T_1$, then
	$$
	\forall x \in (X, \mathcal T) : \exists U \in \mathcal T : \{x\} = X \setminus U.
	$$
\end{proposition}
%--------------------------------


%--------------------------------
\begin{definition}
	[$T_2$ spaces]
	\label{def: T_2 spaces}
	A topological space $(X, \mathcal T)$ is said to be $T_2$ or \textit{Hausdorff} or \textit{separated} iff any two distinct points in $(X, \mathcal T)$ are separated by neighbourhoods.
\end{definition}
%--------------------------------


%--------------------------------
\begin{definition}
	[$T_{2 \nicefrac{1}{2}}$ spaces]
	\label{def: T_2.5 spaces}
	A topological space $(X, \mathcal T)$ is said to be $T_{2 \nicefrac{1}{2}}$ or \textit{Urysohn} iff two distinct points in $X$ are separated by closed neighbourhoods.
\end{definition}
%--------------------------------


%--------------------------------
\begin{definition}
	[$T_3$ spaces]
	\label{def: T_3 spaces}
	A topological space $(X, \mathcal T)$ is said to be $T_3$ or \textit{regular} iff it is $T_0$ and given any point $x \in (X, \mathcal T)$ and closed set $V \subseteq X$ with $x \notin V$ are separated by neighbourhoods.
\end{definition}
%--------------------------------


%--------------------------------
\begin{definition}
	[$T_{3\nicefrac{1}{2}}$ spaces]
	\label{def: T_3.5 spaces}
	A topological space $(X, \mathcal T)$ is said to be $T_{3 \nicefrac{1}{2}}$, or \textit{Tychonoff} or, \textit{completely $T_3$}, or \textit{completely regular}, iff it is $T_0$ and given any point $x$ and closed set $V \subseteq X$ with $x \notin V$, they are separated by a continuous function.
\end{definition}
%--------------------------------


%--------------------------------
\begin{definition}
	[$T_4$ spaces]
	\label{def: T_4 spaces}
	A topological space $(X, \mathcal T)$ is said to be $T_4$ or \textit{normal} iff it is Hausdorff and any tow disjoint closed subsets of $X$ are separated by neighbourhoods.
\end{definition}
%--------------------------------


%--------------------------------
\begin{proposition}
	[Urysohn's lemma]
	\label{prop: urysohn's lemma}
	A topological space is normal iff any two disjoint closed sets are separated by a continuous function.
\end{proposition}
%--------------------------------


%--------------------------------
\begin{definition}
	[$T_5$ spaces]
	\label{def: T_5 spaces}
	A topological space $(X, \mathcal T)$ is said to be $T_5$ or \textit{completely $T_4$} iff it is $T_1$ any two separated sets are separated by neighbourhoods.
\end{definition}
%--------------------------------


%--------------------------------
\begin{proposition}
	Every subspace of a $T_5$ space is normal.
\end{proposition}
%--------------------------------


%--------------------------------
\begin{definition}
	[$T_6$ spaces]
	\label{def: T_6 spaces}
	A topological space $(X, \mathcal T)$ is said to be $T_6$, or \textit{perfectly $T_4$} or \textit{perfectly normal} iff it is $T_1$ and any two disjoint closed sets are precisely separated by a continuous function.
\end{definition}
%--------------------------------



%--------------------------------
\begin{proposition}
	[Tietze extension theorem]
	\label{prop: Tietze extension theorem}
	Let $(X, \mathcal T)$ be normal topological space, and let $f: A \to (\mathbb R, \mathcal T')$ be a continuous map where $A$ is a closed subset of $X$ and $\mathcal T'$ is the standard topology (induced by Euclidean metric). Then there exists a continuous map
	$$
	F: (X, \mathcal T) \to (\mathbb R, \mathcal T'),
	$$
	such that
	$$
	\forall x \in A: f(x) = g(x).
	$$
\end{proposition}
%--------------------------------


%================================
\section{Boundaries}
%================================


%--------------------------------
\begin{definition}
	[interiors]
	\label{def: interiors}
	The \textit{interior} of a set $A$, denoted $A^\circ$, is defined to be the union of all open subsets of $A$.
\end{definition}
%--------------------------------


%--------------------------------
\begin{definition}
	[closure]
	\label{def: closure}
	The \textit{closure} of a set $A$, denoted $\overline A$, is defined to be the intersection of all closed supersets of $A$.
\end{definition}
%--------------------------------


%--------------------------------
\begin{definition}
	[dense sets]
	\label{def: dense sets}
	Let $(X, \mathcal T)$ be a topological space, and let $A \subseteq X$. $A$ is said to be dense, iff $\overline A = X$.
\end{definition}
%--------------------------------


%--------------------------------
\begin{definition}
	[nowhere dense sets]
	\label{def: nowhere dense sets}
	A set $A$ is said to be \textit{nowhere dense} iff the interior of its closure is empty.
\end{definition}
%--------------------------------


%--------------------------------
\begin{definition}
	[boundaries]
	\label{def: boundaries}
	Let $A$ be any set, the \textit{boundary} of $A$, denoted $\partial A$, is defined to be the complement of the interior of $A$ in the closure of $A$; i.e.,
	$$
	\partial A = \overline A \setminus A^\circ.
	$$
\end{definition}
%--------------------------------


%--------------------------------
\begin{proposition}
	[properties of interiors]
	\label{prop: properties of interiors}
	Let $(X, \mathcal T)$ be any topological space and $A, B \subseteq X$.
	\begin{enumerate}[(i)]
		\item
		(Intensive) $A^\circ \subseteq A$.
		
		\item
		$A$ is open iff $A = A^\circ$.
		
		\item
		(Idempotence) $(A^\circ)^\circ = A^\circ$.
		
		\item
		$(A \cap B)^\circ = A^\circ \cap B^\circ$.
		
		\item
		$A \subseteq B \implies A^\circ \subseteq B^\circ$.
		
		\item
		If $B$ is open, then $B \subseteq A$ iff $B \subseteq A^\circ$.
		
	\end{enumerate}
	
	\begin{proof} \
		\begin{enumerate}[(i)]
			\item
			By Definition \ref{def: interiors}, naturally, $A^\circ \subseteq A$.
			
			\item
			By Definition \ref{def: topological spaces}, $A^\circ$ is the union of open sets hence it is open. $A$ is open iff it is the union of all open subsets of $A$. Thus $A = A^\circ$.
			
			\item
			$A^\circ$ is open, thus $(A^\circ)^\circ = A^\circ$.
			
			\item
			By Definition \ref{def: interiors}, we have
			$$
			\begin{aligned}
				(A \cap B)^\circ &= \left\{ \bigcup U : U \in \mathcal T \land U \subseteq A \cap B \right\} \\
				&= \left\{ \bigcup U: (U \in \mathcal T \land U \subseteq A) \land (U \in \mathcal T \land U \subseteq B) \right\} \\
				&= \left\{ \bigcup U: U \in \mathcal T \land  U \subseteq A \right\} \cap \left\{ \bigcup U : U \in \mathcal T \land U \subseteq B \right\} \\
				&= A^\circ \cap B^\circ.
			\end{aligned}
			$$
			
			\item
			Clearly, $A^\circ \subseteq A$, thus,
			$$
			\begin{aligned}
				A \subseteq B &\implies A^\circ \subseteq B
			\end{aligned}
			$$
			Suppose $A^\circ \not \subseteq B^\circ$, then $A^\circ \setminus B^\circ$ is not empty ($\emptyset$ is the subset of any set, so $A^\circ$ is not empty). 
			
			Then there exists $x \in A^\circ$ with $x \in \partial B$ ($x \in B$ but $x\notin B^\circ$). Then there exists neighbourhood $N_x \ni x$, and $N_x \cap \partial B \ne \emptyset.$ But this is impossible, for $A^\circ \subseteq B$ implies that $A^\circ \cap \partial B = \emptyset$ (This is a straight consequence of $A^\circ \cap \partial A = \emptyset$. See Proposition \ref{prop: properties of boundaries}), so such $N_x$ does not exist. Thus,
			$$
			A^\circ \subseteq B^\circ.
			$$
			
			\item
			If $B$ is open, then $B = B^\circ$. Then $B \subseteq A$ iff $B^\circ \subseteq A^\circ$.
 		\end{enumerate}
	\end{proof}
\end{proposition}
%--------------------------------


%--------------------------------
\begin{proposition}
	[properties of closures]
	\label{prop: properties of closures}
	Let $(X, \mathcal T)$ be a topological space, and let $A, B \subseteq X$.
	\begin{enumerate}[(i)]
		\item
		$\overline A$ is closed.
		
		\item
		$A$ is closed iff $A = \overline A$.
		
		\item
		$A \subseteq B$ implies $\overline A \subseteq \overline B$.
		
		\item
		If $A$ is closed, then $A \supseteq B$ iff $A \supseteq \overline B$
	\end{enumerate}
	
	\begin{proof} \
		\begin{enumerate}[(i)]
			\item
			By Definition \ref{def: closure}, $\overline A$ is the intersection of closed sets. By Proposition \ref{prop: dark side of topology}, $\overline A$ is closed.
			
			\item
			Proposition \ref{prop: dark side of topology} implies that any closed set is the intersection of closed sets, this is precisely what Definition \ref{def: closure} says.
			
			\item
			$$
			\begin{aligned}
				A \subseteq B &\implies X \setminus A \supseteq X \setminus B \\
			\end{aligned}
			$$
			
			$$
			\begin{aligned}
				
			\end{aligned}
			$$
			
		\end{enumerate}
	\end{proof}
\end{proposition}
%--------------------------------

% todo: starts from here.


%--------------------------------
\begin{proposition}
	[properties of boundaries]
	\label{prop: properties of boundaries}
	Let $(X, \mathcal T)$ be a topological space, and let $A \subseteq X$.
	\begin{enumerate}[(i)]
		\item 
		$\partial A$ is closed.
		
		\item
		$A^\circ \cap \partial A = \emptyset$.
		
		\item
		$\overline A = A^\circ \cup \partial A$.
		
		\item
		$A$ is closed iff $\partial A \subseteq A$.
		
		\item
		$\partial A$ is nowhere dense.
		
		\item
		$\partial \overline A \subseteq \partial A \subseteq \partial A^\circ$.
		
		\item
		$\partial A = \partial (X \setminus A)$.
		
		\item
		$A$ is dense iff $\partial A = X \setminus A^\circ$.
		
	\end{enumerate}
	
	\begin{proof} \
		\begin{enumerate}[(i)]
			\item
			$\overline A$ is closed, and $X \setminus A^\circ$ is also closed. Thus
			$$
			\partial A = \overline A \setminus A^\circ = \overline A \cap (X \setminus A)
			$$
			is closed.
			
			\item
			By Definition \ref{def: boundaries}, we have
			$$
			\begin{aligned}
				\partial A = \overline A \setminus A^\circ &\iff \partial A \cap A^\circ = \overline A \setminus A^\circ \cap A^\circ = \overline A \cap \emptyset = \emptyset.
			\end{aligned}
			$$
			
			\item
			We have
			$$
			\begin{aligned}
				\partial A = \overline A \setminus A^\circ &\iff \partial A \cup A^\circ = \overline A \setminus A^\circ \cup A^\circ = \overline A \cap (X \setminus A^\circ \cup A^\circ) \\
				&\iff \partial A \cup A^\circ = \overline A \cap X |_\text{for $A^\circ \subseteq X$} = \overline A.
			\end{aligned}
			$$
			
			\item As $A$ is closed, $A = \overline A$ (this can be straightly proved by Definition \ref{def: closure}). By Definition \ref{def: boundaries}, it is clear that $\partial A \subseteq \overline A$, thus $\partial A \subseteq A$.
			
			\item
			By Definition \ref{def: nowhere dense sets}, $\partial A$ is nowhere dense iff $\overline{\partial A}^\circ$ is empty. We have
			$$
			\begin{aligned}
				\overline{\partial A}^\circ &= \overline{\overline A \setminus A^\circ}^\circ \\
				&= (\overline A \setminus A^\circ) \cup (\overline A \setminus A^\circ) \setminus (\overline A \setminus A^\circ) \\
				&= \emptyset.
			\end{aligned}			
			$$
			
			\item
			$\overline A \supseteq A^\circ$ implies $\overline A^\circ \supseteq (A^\circ)^\circ = A^\circ$, then we have,
			$$
			\begin{aligned}
				\partial \overline A &= \overline{\overline A} \setminus  \overline A^\circ \subseteq \overline A \setminus A^\circ = \partial A.
			\end{aligned}
			$$
			
			$A^\circ \subseteq A$ implies $ \overline{A^\circ} \subseteq \overline A$, then we have,
			$$
			\begin{aligned}
				\partial A^\circ = \overline{A^\circ} \setminus (A^\circ)^\circ \supseteq \overline A \setminus A^\circ.
			\end{aligned}
			$$
			
			\item
			We have
			$$
			\begin{aligned}
				\partial (X \setminus A) &= \overline{X \setminus A} \setminus (X \setminus A)^\circ \\
				&= X \setminus A^\circ \setminus (X \setminus \overline A) \\
				&= X \setminus A^\circ \cap \overline A \\
				&= \overline A \setminus A^\circ \\
				&= \partial A.
			\end{aligned}
			$$
			
			\item
			By Definition \ref{def: dense sets}, $A$ is dense  in $X$ iff $\overline A = X$. Then we have,
			$$
			\begin{aligned}
				\overline A = X &\iff \overline A \setminus A^\circ = X \setminus A^\circ \\
				&\iff \partial A = X \setminus A^\circ.
			\end{aligned}
			$$
		\end{enumerate}
	\end{proof}
\end{proposition}
%--------------------------------


%================================
\section{Limit Points}
%================================


%--------------------------------
\begin{definition}
	[limit points]
	\label{def: limit points}
	Let $(X, \mathcal T_X)$ be a topological space, and let $A \subseteq X$. A point $x \in X$ is called a \textit{limit point} of $A$ iff for all neighbourhood $N_x$ of $x$, $N_x \setminus \{x\}$ intersects $A$.
\end{definition}
%--------------------------------

%--------------------------------
\begin{proposition}
	\label{prop: limit points are always in the closure}
	Let $A$ be any set, and let $x$ be a limit point of $A$, then $x$ is an element of the closure of $A$.
	
	\begin{proof}
		If $A$ is empty, then this is vacuously true. So, suppose $A$ is not empty.
	
		By Definition \ref{def: limit points}, for all neighbourhood $N_x$ of $x$, $N_x \setminus \{x\} \cap A$ is not empty. Naturally, $N_x \cap A$ is not empty.
		
		Assume that $x \notin \overline A$, then $X \setminus \overline A$ is a neighbourhood of $x$, by Definition \ref{def: neighbourhood}, and is disjoint from $A$. This is contradicted to the conditions.
	\end{proof}
\end{proposition}
%--------------------------------


%--------------------------------
\begin{note}
	In this proof, the proposition also holds for $N_x \cap A^\circ = \emptyset$. Because if it is true, then
	$$
	\begin{aligned}
		N_x \cap \partial A \supseteq (N_x \cap A) \setminus (N_x \cap A^\circ) = N_x \cap A.
	\end{aligned}
	$$
	This implies that $A \subseteq \partial A$. In this case, $\overline A = \partial A$, for
	
	Assume that $x\notin \partial A$, then we have the same conclusion.
	
	Then $A^\circ = A \setminus \partial A = \emptyset$. 
\end{note}
%--------------------------------


%--------------------------------
\begin{proposition}
	A set is closed iff it contains all its limit point.
	
	\begin{proof}
		Let $A$ be a set. By proposition \ref{prop: limit points are always in the closure}, for every limit point of $A$, it is also an element of the closure $\overline A$. And $A$ is closed iff $A = \overline A$.
	\end{proof}
\end{proposition}
%--------------------------------


\

\

%--------------------------------
\begin{definition}
	[convergent sequences]
	\label{def: convergent sequences}
	Let $(X, \mathcal T_X)$ be a topological space. A sequence $\{x_n\}$ in $X$ is said to be \textit{convergence} in $X$ iff there is an open set $U$ contains all but finite terms of $\{x_n\}$.
\end{definition}
%--------------------------------







































%