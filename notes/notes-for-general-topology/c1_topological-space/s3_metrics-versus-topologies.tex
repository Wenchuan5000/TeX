%================================
\section{Metrics versus Topologies}
%================================


%--------------------------------
\begin{definition}
	Let $X$ be a set, and let $\rho$ and $\rho'$ be metrics on $X$. We say that $\rho$ and $\rho'$ are \textit{topologically equivalent} if they induce the same topology on $X$.
\end{definition}
%--------------------------------


%--------------------------------
\begin{definition}
	$\rho $ and $\rho'$ are \textit{Lipschitz equivalent} iff there exist $c, C \in \mathbb R_{>0}$ such that for all $x,y \in X$,
	$$
	c \rho(x,y) \le \rho'(x,y) \le C\rho(x,y).
	$$
\end{definition}
%--------------------------------


%--------------------------------
\begin{lemma}
	Lipschitz equivalence implies topological equivalence.
\end{lemma}
%--------------------------------

%--------------------------------
\begin{proof}
	As $\rho$ and $\rho'$ are Lipschitz equivalent, by definition, there exist $c \in \mathbb R_{>0}$ such that for all $x,y \in X$,
	$$
	c \rho(x, y) \le \rho'(x, y).
	$$
	
	Given $r > 0$ and $x \in X$,
	$$
	B_{c\rho}(x, r) = \left\{ y \in X : c\rho(x, r) < r \right\}
	$$
	and
	$$
	B_{\rho'}(x,r) = \{ y \in X : \rho'(x,r) < r \}.
	$$
	
	As $r$ is non-underestimated compared to $\rho'$, then
	$$
	B_{\rho'}(x,r) \supseteq B_{c\rho}(x, r) = B_{\rho}\left(x, \frac{1}{c}r \right)
	$$
	is an open neighbourhood of $x$ in $(X, \rho')$ and is a subset
	
	Let $U \in \mathcal T_{\rho'}$, then for some $\varepsilon > 0$,
	$$
	U \supseteq B_{\rho'} (x, \varepsilon) \supseteq B_{\rho}\left( x, \frac{1}{c}r \right).
	$$
	Thus $U$ is open with respect to $\rho$, i.e., $U \in \mathcal T_{\rho}$.
	
	It is not necessary to prove converse for there always exists $C \in \mathbb R_{>0}$ such that $c = \frac{1}{C}$.
\end{proof}
%--------------------------------


%--------------------------------
\begin{note} \ 
	\begin{enumerate}
		%--------------------------------
		\item For all $p \ge 0$, $\rho_p: \mathbb R^n \times \mathbb R^n \to \mathbb R_{\ge 0}$ are topologically equivalent.
		%--------------------------------
		
		%--------------------------------
		\item On $C[a,b]$, $\rho_1$ and $\rho_\infty$ induce different topologies, hence they are not topologically equivalent, and in particular, they are not Lipschitz equivalent. As Lipschitz equivalence implies topological equivalence, but not vice versa. So Lipschitz in-equivalence do nothing to the proof the topological in-equivalence between $\rho_1$ and $\rho_\infty$. 
		%--------------------------------
		
		%--------------------------------
		\item $\rho_p$ and $\rho_\text{disc}$ on $\mathbb Z$ are topologically equivalent. Firstly, topology $\mathcal T_{\rho_p} = \mathcal P(\mathbb Z)$, because for all $B_{\rho_p} (x, \varepsilon)$ for all $x \in \mathbb Z$ and $\varepsilon \in \mathbb R_{(0,1)}$,  $B_{\rho_p}(x, \varepsilon) = \{x\}$. Thus, for all, $U \in \mathcal P(\mathbb Z)$,
			$$
			U = \bigcup_{x \in U} B_{\rho_p}(x, \varepsilon) = \bigcup_{x \in U} \{x\} \in \mathcal T_{\rho_p}.
			$$
			Thus $\mathcal P(\mathbb Z) \subseteq \mathcal T_{\rho_p}$, but $\mathcal T_{\rho_p} \subseteq \mathcal P(\mathbb Z)$, so $\mathcal P(\mathbb Z) = \mathcal T_{\rho_p}$. Thus $\mathcal T_{\rho_p} = \mathcal T_\text{disc}$.
		%--------------------------------
	\end{enumerate}
\end{note}
%--------------------------------


%--------------------------------
\begin{definition}
	A topological space $(X, \mathcal T)$ is \textit{metrizable} iff $\mathcal T$ is induced by some metric on $X$.
\end{definition}
%--------------------------------


%--------------------------------
\begin{note} \
	\begin{enumerate}
		\item Let $(\mathbb Z, \mathcal T)$ with
		$$
		\mathcal T = \{ U \in \mathcal P(\mathbb Z) : |U| \le 1 \},
		$$
		Then $\mathcal T$ is not induced by any metric. Suppose it were, then all open set $U \in \mathcal T$ should be monotone, and for all $\varepsilon > 0$, and for all $x \in \mathbb Z$, $B(x, \varepsilon)$ should be monotone. But if $\mathcal T$ is induced by some metric, then for all $I \in \mathcal P(X)$ with $|I| > 1$, a set
		$$
		W = \bigcup_{x \in I} B(x, \varepsilon) \in \mathcal T,
		$$
		then $|W| > 1$, which is contradicted to the conditions.
	\end{enumerate}
\end{note}
%--------------------------------


%--------------------------------
\begin{definition} \
	\begin{enumerate}[(i)]
		\item A topological space $(X, \mathcal T)$ is said to be $T_1$ iff every monotone in $\mathcal P(X)$ is closed.
		\item A topological space $(X, \mathcal T)$ is said to be $T_2$ or \textit{Hausdorff} iff
			$$
			\forall x, y \in X \ (x \ne y), \ \exists U, W \in \mathcal T \ (U \cap W = \emptyset), \quad x \in U \land y \in W.
			$$
	\end{enumerate}
\end{definition}
%--------------------------------


%--------------------------------
\begin{note} \ 
	\begin{enumerate}
		\item $(X, \mathcal T_{\rho_\text{disc}})$ is $T_1$, for as any set $U \subseteq X$ is closed for $X \setminus U \in \mathcal T_{\rho_\text{disc}}$ as well. It is also Hausdorff, because for all $x,y \in X$, $\{x\}, \{y\} \in \mathcal T_{\rho_\text{disc}}$ and $\{x\} \cap \{y\} = \emptyset$ if $x \ne y$.
		\item On the other hand, $(X, \{\emptyset, X\})$ is $T_1$ iff $|X| = 1$. And $(X, \{\emptyset, X\})$ is not Hausdorff, because there exist $x,y \in X$ with $x \ne y$, the only open set contains $x$ is $X$, and the only open set contains $y$ is $X$. Clearly, $X \cap X$
	\end{enumerate}
\end{note}
%--------------------------------


%--------------------------------
\begin{lemma} \
	\begin{enumerate}[(i)]
		\item Every metrizable space is Hausdorff.
		\item Every Hausdorff topological space is $T_1$.
	\end{enumerate}
\end{lemma}
%--------------------------------


%--------------------------------
\begin{proof} \
	\begin{enumerate} [(i)]
		\item Let $(X, \rho)$ be metric space, then for all $x, y \in X$, let $r = \frac{\rho(x,y)}{2}$. Suppose $(X, \rho)$ is not Hausdorff, i.e., there is $z \in B(x, r) \cap B(y, r)$. By metric axioms, we have
			$$
			\rho(x, z) + \rho(y, z) \ge \rho (x,y) = 2r.
			$$
			But $z \in B(x, r)$ implies that $\rho(x,z) < r$, and $z \in B(y, r)$ implies that $\rho(y,z) < r$, then we have
			$$
			\rho(x, z) + \rho(y, z) < \rho(x,y),
			$$
			which is contradicted to the metric axioms.
			
		\item (Just an outline...) Let $(X, \mathcal T)$ be Hausdorff. Suppose $X$ is not $T_1$, then there is $\{x\} \subseteq X$ which is not closed. Then there must be a smallest $V \supsetneq \{x\}$ which is closed (Why?). Then there must be a smallest $U \in \mathcal T$ with $U \supseteq V$ (Why?). Then for all $x, y \in U$, there is no disjoint $U_x, U_y$ such that $U_x \ni x$ and $U_y \ni y$.
	\end{enumerate}
\end{proof}
%--------------------------------


%--------------------------------
\begin{definition}
	Let $(X, \mathcal T)$ be a topological space, let $\{x_n\}_{n = 1}^\infty$ be a sequence in $X$, and let $x \in X$. Then $\{x_n\}$ \textit{converges} in $X$ iff there is an $x \in X$, for all $U \in \mathcal T$ with $x \in U$, $U$ contains all but finite elements in $\{x_n\}$.
\end{definition}
%--------------------------------


%--------------------------------
\begin{note} \
	\begin{enumerate}
		\item If $(X, \mathcal T)$ is metrizable, i.e., there is a metric $\rho$ can induce $\mathcal T$. If $\{x_n\} \subseteq X$ converges in $X$, then there exists $x \in X$, for all $\varepsilon > 0$, $B(x, \varepsilon)$ contains all but finite elements in $\{x_n\}$.
		\item If $\mathcal T$ is a discrete topology, a sequence $\{x_n\}$ converges in $(X, \mathcal T)$ iff there is an $N$ such that for all $n \ge N$, $x_n = x_{n + 1}$.
		\item If $\mathcal T$ is an indiscrete topology, then any $\{x_n\} \subseteq X$ converges to any point in $X$, for there is only one non-empty open set which is $X$ itself.
	\end{enumerate}
\end{note}
%--------------------------------


%--------------------------------
\begin{lemma}
	In Hausdorff topological space, any convergent sequence converges to at most one point.
\end{lemma}
%--------------------------------

%--------------------------------
\begin{proof}
	Let $(X, \mathcal T)$ be a Hausdorff topological space. Suppose there is a sequence $\{x_n\}$ converges to $x,y \in X$ with $x \ne y$. By the definition of topological convergence, there are $U_x, U_y \in \mathcal T$ both contains all but finite elements in $\{x_n\}$. $U_x \cap U_y$ must be non-empty (Explain!). $x,y \in U_x \cap U_y$, for if they were not, by Hausdorff property, there must be open $V_x \subseteq U_x$ and $V_y \subseteq U_x$ with $V_x \ni x$ and $V_y \ni y$, and they both contains all but finite elements in $\{x_n\}$, which is not possible. Thus, there is no such open sets $V_x \ni x$ and $V_y \ni y$ with $V_x \cap V_y = \emptyset$, which implies $(X, \mathcal T)$ is not Hausdorff. This is a contradition.
\end{proof}
%--------------------------------


%--------------------------------
\begin{definition} \
	\begin{enumerate}[(i)]
		\item A topological space $(X, \mathcal T)$ is \textit{regular} iff for all closed sets $V \subseteq X$ and $x \in X$ with $x \notin V$, there exist disjoint open sets $U,W \subseteq X$ such that $V \subseteq U$ and $x \in W$.
		\item $(X, \mathcal T)$ is \textit{normal} iff for all disjoint closed sets $V, Z \subseteq X$, there exist disjoint open sets $U, W \subseteq X$ such that $V \subseteq U$ and $Z \subseteq W$.
	\end{enumerate}
\end{definition}
%--------------------------------


%--------------------------------
\begin{note}
	[To do] \
	\begin{enumerate}
		\item Can I find a regular space which is not normal?
		\item Can I find a normal space which is not regular?
		\item Does regular implies normal or normal implies regular?
	\end{enumerate}
\end{note}
%--------------------------------
































%