%================================
\section{Separation Axioms}
%================================


%--------------------------------
\begin{definition}
	[topologically indistinguishable]
	\label{def: topologically indistinguishable}
	Let $(X, \mathcal T)$ be a topological space. Two points $x,y \in X$ are said to be \textit{topologically indistinguishable} iff they share all their neighbourhoods. That is, let $\mathcal N_x$ be the family of all neighbourhoods of $x$ and let $\mathcal N_y$ be the family of all neibourhoods of $y$, we have
	$$
	\mathcal N_x = \mathcal N_y.
	$$
	
	Respectively, $x,y$ are said to be \textit{topologically distinguishable} iff they are not topologically distinguishable; i.e.,
	$$
	\mathcal N_x \ne \mathcal N_y.
	$$
\end{definition}
%--------------------------------


%--------------------------------
\begin{definition}
	[saperated sets]
	Let $(X, \mathcal T)$ be a topological space, and let $A, B \in \mathcal P(X)$.
	
	\begin{enumerate}[(i)]
		\item $A$ and $B$ are said to be \textit{separated} iff each is disjoint from other's closure.
		\item $A$ and $B$ are said to be \textit{separated by neighbourhoods} iff there are neighbourhoods $N_A$ of $A$ and $N_B$ of $B$ such that $N_A$ and $N_B$ are disjoint.
		\item $A$ and $B$ are said to be \textit{separated by closed neighbourhoods} iff there are closed neighbourhoods $\overline N_A$ of $A$ and $\overline N_B$ of $B$ such that $\overline N_A$ and $\overline N_B$ are disjoint.
		\item $A$ and $B$ are said to be \textit{separated by a continuous function} iff there is a continuous function $f: X \to \mathbb R$, such that $f[A] = \{0\}$ and $f[B] = \{1\}$.
		\item $A$ and $B$ are said to be \textit{precisely separated by a continuous function} iff there is a continuous function $f: X \to \mathbb R$, such that $f^{-1}[\{0\}] = A$ and $f^{-1}[\{1\}] = B$
	\end{enumerate}
\end{definition}
%--------------------------------


\href{https://en.wikipedia.org/wiki/Separated_sets}{See Wikipedia.org}


%--------------------------------
\begin{definition}
	[$T_0$ spaces]
	\label{def: T_1 spaces}
	A topological space $(X, \mathcal T)$ is said to be $T_0$ or \textit{Kolmogorov}, iff all distinct points $x,y \in X$ are \textit{topologically distinguishable}.
\end{definition}
%--------------------------------


%--------------------------------
\begin{definition}
	[$R_0$ spaces]
	\label{def: R_0 spaces}
	A topological space $(X, \mathcal T)$ is said to be $R_0$ iff any two topologically distinguishable points in $X$ are separated.
\end{definition}
%--------------------------------


%--------------------------------
\begin{example}
	[$T_0$ but not $R_0$]
	% todo: check if it exists!
\end{example}
%--------------------------------


%--------------------------------
\begin{definition}
	[$T_1$ spaces]
	\label{def: T_1 spaces}
	A topological space $(X, \mathcal T)$ is said to be $T_1$ or \textit{Fr\'echet} iff any two distinct points in $X$ are separated.
\end{definition}
%--------------------------------


%--------------------------------
\begin{proposition}
	[alternative definition of $T_1$ spaces]
	\label{prop: alternative definition of T_1 spaces}
	A topological space $(X, \mathcal T)$ is $T_1$ iff all singletons in $X$ are closed.
\end{proposition}
%--------------------------------


%--------------------------------
\begin{example}
	[$T_0$ but not $T_1$]
	\label{eg: T_0 but not T_1}
	Let $(\mathbb N, \mathcal T)$ be a topological space with
	$$
	\mathcal T = \left\{ U \subseteq \mathbb N : U = \bigcup_{i = 0}^{n} [ 0, i ) \right\},
	$$
	Then for all $x,y \in (\mathbb N, \mathcal T)$, if $x \ne y$, then there are $|y - x|$ neighbourhoods $N_x$ of $x$ do not contain $y$. Thus, it is $T_0$.
	
	On the other hand, it not $T_1$, for all singletons in $(\mathbb N, \mathcal T)$ except $\{0\}$ are not closed. By Proposition \ref{prop: alternative definition of T_1 spaces}, it is not $T_1$.
\end{example}
%--------------------------------


%--------------------------------
\begin{proposition}
	\label{prop: all singletons in a T_1 space are closed}
	
	All singletons in a $T_1$ space are closed, That is, if a topological space $(X, \mathcal T)$ is $T_1$, then
	$$
	\forall x \in (X, \mathcal T) : \exists U \in \mathcal T : \{x\} = X \setminus U.
	$$
\end{proposition}
%--------------------------------


%--------------------------------
\begin{definition}
	[$T_2$ spaces]
	\label{def: T_2 spaces}
	A topological space $(X, \mathcal T)$ is said to be $T_2$ or \textit{Hausdorff} or \textit{separated} iff any two distinct points in $(X, \mathcal T)$ are separated by neighbourhoods.
\end{definition}
%--------------------------------

%--------------------------------
\begin{example}
	[$T_2$ but not $T_1$]
	\label{eg: T_2 but not T_1}
	Let $X$ be an nonempty set, and let $\mathcal U = \mathcal P(X \setminus \{x \in X\})$. Then the topological space $(X, \mathcal T)$ with
		$$
		\mathcal T = \mathcal U \cup \{X\}
		$$
		is $T_1$. But it is $T_2$ iff $|X| = 1$ (This is vacuously true). As $|X| > 1$, $\{x\}$ is not open.
\end{example}
%--------------------------------

%--------------------------------
\begin{proposition}
	All metric spaces are Hausdorff.
\end{proposition}
%--------------------------------


%--------------------------------
\begin{definition}
	[$T_{2 \nicefrac{1}{2}}$ spaces]
	\label{def: T_2.5 spaces}
	A topological space $(X, \mathcal T)$ is said to be $T_{2 \nicefrac{1}{2}}$ or \textit{Urysohn} iff two distinct points in $X$ are separated by closed neighbourhoods.
\end{definition}
%--------------------------------


%--------------------------------
\begin{definition}
	[$T_3$ spaces]
	\label{def: T_3 spaces}
	A topological space $(X, \mathcal T)$ is said to be $T_3$ or \textit{regular} iff it is $T_0$ and given any point $x \in (X, \mathcal T)$ and closed set $V \subseteq X$ with $x \notin V$ are separated by neighbourhoods.
\end{definition}
%--------------------------------


%--------------------------------
\begin{definition}
	[$T_{3\nicefrac{1}{2}}$ spaces]
	\label{def: T_3.5 spaces}
	A topological space $(X, \mathcal T)$ is said to be $T_{3 \nicefrac{1}{2}}$, or \textit{Tychonoff} or, \textit{completely $T_3$}, or \textit{completely regular}, iff it is $T_0$ and given any point $x$ and closed set $V \subseteq X$ with $x \notin V$, they are separated by a continuous function.
\end{definition}
%--------------------------------


%--------------------------------
\begin{definition}
	[$T_4$ spaces]
	\label{def: T_4 spaces}
	A topological space $(X, \mathcal T)$ is said to be $T_4$ or \textit{normal} iff it is Hausdorff and any tow disjoint closed subsets of $X$ are separated by neighbourhoods.
\end{definition}
%--------------------------------


%--------------------------------
\begin{proposition}
	[Urysohn's lemma]
	\label{prop: urysohn's lemma}
	A topological space is normal iff any two disjoint closed sets are separated by a continuous function.
\end{proposition}
%--------------------------------


%--------------------------------
\begin{definition}
	[$T_5$ spaces]
	\label{def: T_5 spaces}
	A topological space $(X, \mathcal T)$ is said to be $T_5$ or \textit{completely $T_4$} iff it is $T_1$ any two separated sets are separated by neighbourhoods.
\end{definition}
%--------------------------------


%--------------------------------
\begin{proposition}
	Every subspace of a $T_5$ space is normal.
\end{proposition}
%--------------------------------


%--------------------------------
\begin{definition}
	[$T_6$ spaces]
	\label{def: T_6 spaces}
	A topological space $(X, \mathcal T)$ is said to be $T_6$, or \textit{perfectly $T_4$} or \textit{perfectly normal} iff it is $T_1$ and any two disjoint closed sets are precisely separated by a continuous function.
\end{definition}
%--------------------------------



%--------------------------------
\begin{proposition}
	[Tietze extension theorem]
	\label{prop: Tietze extension theorem}
	Let $(X, \mathcal T)$ be normal topological space, and let $f: A \to (\mathbb R, \mathcal T')$ be a continuous map where $A$ is a closed subset of $X$ and $\mathcal T'$ is the standard topology (induced by Euclidean metric). Then there exists a continuous map
	$$
	F: (X, \mathcal T) \to (\mathbb R, \mathcal T'),
	$$
	such that
	$$
	\forall x \in A: f(x) = g(x).
	$$
\end{proposition}
%--------------------------------