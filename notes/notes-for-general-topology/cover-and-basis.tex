%================================
\section{Cover and Basis}
%================================


%--------------------------------
\begin{definition}
	[cover]
	\label{def: cover}
	Let $(X, \mathcal T)$ be a topological space, and let $U \subseteq X$, then a family $\mathcal C \subseteq \mathcal P(X)$ is called a \textit{cover} of $U$ iff the union of all sets in $\mathcal C$ is a superset of $U$. That is,
	$$
	U \subseteq \bigcup \mathcal C.
	$$
	
	If $\mathcal C \subseteq \mathcal T$, then we call $\mathcal C$ an \textit{open cover} of $U$.
	
	Let $\mathcal S \subseteq \mathcal C$, iff the union of $\mathcal S$ is still a superset of $U$, then we call $\mathcal S$ a \textit{subcover} of $\mathcal C$.
\end{definition}
%--------------------------------


%--------------------------------
\begin{definition}
	[basis]
	\label{def: basis}
	Let $(X, \mathcal T)$ be a topological space, let $U \subseteq X$, and let $\mathcal B$ be a open cover of $X$. We call $\mathcal B$ a \textit{base} of $X$ iff the union of $\mathcal B$ is precisely $U$ itself, i.e.,
	$$
	U = \bigcup \mathcal B.
	$$
\end{definition}
%--------------------------------


%--------------------------------
\begin{definition}
	[synthetic basis]
	\label{def: synthetic basis}
	Let $(X, \mathcal T)$ be a topological space, and let $\mathcal B$ be a base of $X$. $\mathcal B$ is said to be \textit{synthetic} iff for any $A, B \in \mathcal B$,
	$$
	A \cap B = \bigcup_{i = 1}^{n} B_i, \quad B_i \in \mathcal B.
	$$
\end{definition}
%--------------------------------