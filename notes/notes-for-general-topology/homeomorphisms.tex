%================================
\section{Homeomorphisms}
%================================


%--------------------------------
\begin{definition}
	[homeomorphisms]
	\label{def: homomorphisms}
	Let $(X, \mathcal T_X)$ and $(Y, \mathcal T_Y)$ be topological spaces. A bijection $f: X \to Y$ is called a \textit{homeomorphism} iff it is continuous and its inverse is also continuous.
\end{definition}
%--------------------------------


%--------------------------------
\begin{definition}
	[homeomorphic]
	\label{def: homomorphic}
	Two topological spaces $(X, \mathcal T_X)$ and $(Y, \mathcal T_Y)$ are said to be \textit{homeomorphic} or \textit{topologically equivalent}, denoted $X \cong Y$, iff there is an homeomorphism between them.
\end{definition}
%--------------------------------


%--------------------------------
\begin{proposition}
	Two topological spaces are homeomorphic only if they have the same cardinality.
	
	\begin{proof}
		Let $X$ and $Y$ be two sets with $|X| < |Y|$. There is no surjection from $A$ to $B$.
	\end{proof}
\end{proposition}
%--------------------------------


%--------------------------------
\begin{example}
	$|X| = |Y|$ does not imply $(X, \mathcal T_X)$ and $(Y, \mathcal T_Y)$ are homeomorphic, even if they are finite. For example, let $X = Y = \{1, \ldots, n\}$, and let $\mathcal T_X$ be indiscrete topology on $X$ and $\mathcal T_Y = \mathcal P(X)$. There is no homeomorphism between $X$ and $Y$.
	
	On the other hand, even if $|X| = |Y| \ge \aleph_0$ and $\mathcal T_X$ and $\mathcal T_Y$ are induced by same metric, $X$ and $Y$ might not be homeomorphic. For example, if $\mathcal T_X$ and $\mathcal T_Y$ are both induced by standard Euclidean metric, and $X = [a, b] \subseteq \mathbb R$ and $Y = [c, d) \subseteq \mathbb R$ where $a < b$ and $c < d$. No doubt, $|X| = |Y| = \mathfrak c$, but $X$ and $Y$ are not homeomorphic.
\end{example}
%--------------------------------


%--------------------------------
\begin{example}
	Let $I$ be a proper interval in $\mathbb R^n$, let $\mathcal T$ be standard Euclidean topology on $\mathbb R^n$ and let $\mathcal T_I$ be a subspace topology on $I$. $I \cong \mathbb R^n$ iff $I$ is an open interval.
	
	But if $\mathcal T = \mathcal P(\mathbb R^n)$, then there exists bijection $f: I \to \mathbb R^n$, for $|I| = |\mathbb R^n|$, and such $f$ can be bicontinuous, for any subset $A \subseteq I$ is also open in $\mathbb R^n$, vise versa. In this case, $I \cong \mathbb R^n$ whenever $I$ is a closed, half-close, half-open, or open interval respect to standard Euclidean metric.
\end{example}
%--------------------------------




































%