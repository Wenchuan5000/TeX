
\chapter{Vector Spaces}
%================================================
%::::::::::::::::::::::::::::::::::::::::::::::::


\section{Linear Maps}
%================================================
%------------------------------------------------



\begin{definition}
	\label{def: linear map}	
	A map $f: \mathbb R^n \to \mathbb R^m$ is said to be \textit{linear}, iff for any $\mathbf u, \mathbf v \in \mathbb R^n$, and $a, b \in \mathbb R$,
	$$
	f(a\mathbf u + b\mathbf v) = a f(\mathbf u) + b f(\mathbf v).
	$$
\end{definition}


\begin{note}
	Assume $b = 0$, then we have
	$$
	f(a\mathbf u) = a f(\mathbf u).
	$$

	Assume $a = b = 1$, then Definition \ref{def: linear map} gives
	$$
	f(\mathbf u + \mathbf v) = f(\mathbf u) + f(\mathbf v).
	$$	
\end{note}


\begin{lemma}
	\label{lm: linear map: 0 is 0}
	With the condition above, $f(\mathbf x) = \mathbf 0_{\in \mathbb R^m}$ iff $\mathbf x = \mathbf 0_{\in \mathbb R^n}$.
	\begin{proof}
		$$
		\begin{aligned}
			f(\mathbf x) = \mathbf 0_{\in \mathbb R^m}
			&\iff f(\mathbf x + 0\mathbf x) = \mathbf 0_{\in \mathbb R^m} \\
			&\iff f(\mathbf x) + 0 f(\mathbf x) = \mathbf 0_{\in \mathbb R^m} \\
			&\iff f(\mathbf x) = \mathbf 0_{\in \mathbb R^m}. 
		\end{aligned}
		$$
	\end{proof}
\end{lemma}


\begin{lemma}
	\label{lm: linear map: is injective}
	With the condition above, $f$ is injective.
	\begin{proof}
		For any $\mathbf y \in \mathbb R^m$, there is an $\mathbf x \in \mathbb R^n$ such that
		$$
		\mathbf y = f(\mathbf x).
		$$
		
		Assume there is an $\mathbf x' \in \mathbb R^n$ such that
		$$
		\mathbf y = f(\mathbf x').
		$$
		
		Then we have
		$$
		\begin{aligned}
			f(\mathbf x) - f(\mathbf x') = \mathbf 0
			&\iff f(\mathbf x - \mathbf x') = \mathbf 0.
		\end{aligned}
		$$
		
		By Lemma \ref{lm: linear map: 0 is 0}, we have
		$$
		\mathbf x = \mathbf x'.
		$$
	\end{proof}
\end{lemma}


\begin{lemma}
	\label{lm: linear map: is continuous}
	With the condition above, $f$ is continuous.
	
	\begin{proof}
		Let $\varepsilon \in \mathbb R$, and let $\mathbf p \in \mathbb R^n$. For any $\mathbf q \in \mathbb R^n$ such that $f(\mathbf q) \in B (f(\mathbf p), \varepsilon) \setminus \{f(\mathbf p)\}$,
		$$
		\|f(\mathbf p) - f(\mathbf q) \| = \sqrt{\sum_{i=1}^m (q_i - y_i)^2} < \varepsilon.
		$$
		
		As $f(\mathbf q) \ne f(\mathbf p)$, $\mathbf p \ne \mathbf q$, thus
		$$
		\| \mathbf p - \mathbf q \| > 0.
		$$
	\end{proof}
\end{lemma}


\begin{lemma}
	With the condition above, $f^{-1}$ is linear.
\end{lemma}


%------------------------------------------------
%================================================



\section{Linear Spans and Combinations}
%================================================
%------------------------------------------------


\begin{definition}
	\label{def: span}	
	Let $\langle\mathbf v_i\rangle_{i = 1}^n$ be a sequence of vectors in $\mathbb R^m$.
	
	The \textit{span} of $\langle\mathbf v_i\rangle$ is a subset of $\mathbb R^m$ defined as
	$$
	\mathrm{span}\langle\mathbf v_i\rangle := \left\{ \mathbf a \cdot \langle\mathbf v_i\rangle : \mathbf a \in \mathbb R^n \right\}.
	$$
	
	An element $\mathbf u \in \mathbb R^m$ is a \textit{linear combination} of $\langle\mathbf v_i\rangle$ iff $\mathbf u \in \mathrm{span}\langle \mathbf v_i\rangle$.
\end{definition}



\begin{note}
	By dot product,
	$$
	\mathbf u = \sum_{i = 1}^n a_i \mathbf v_i.
	$$
\end{note}


\begin{note}
	For any vector $\mathbf p \in \mathbb R^m$, the linear combination form of $\mathbf p$ is
	$$
	\mathbf p = \sum_{i = 1}^n p_i \hat e_i,
	$$
	where $\hat e_i$ denotes the $i$-th basis of $\mathbb R^n$; i.e., all terms but the $i$-th term of $\hat e_i$ are $0$.
\end{note}


\begin{lemma}
	\label{lm: span: is linear map}
	With the condition above, let $f: \mathbb R^n \to \mathbb R^m$ be defined as
	$$
	f(\mathbf x) := \mathbf x \cdot \langle \mathbf v_i \rangle.
	$$
	
	Then $f$ is a linear map.
	
	\begin{proof}
		Let $a \in \mathbb R$, and let $\mathbf u, \mathbf v \in \mathbb R^n$ Then we have
		$$
		\begin{aligned}
			f(\mathbf u + \mathbf v) &= (\mathbf u + \mathbf v) \cdot \langle \mathbf v_i \rangle \\
			&= \sum_{i = 1}^n (u_i + v_i) \mathbf v_i \\
			&= \sum_{i = 1}^n u_i \mathbf v_i + \sum_{i = 1}^n v_i \mathbf v_i \\
			&= f(\mathbf u) + f(\mathbf v).
		\end{aligned}
		$$
		
		And
		$$
		\begin{aligned}
			f(a\mathbf u) &= (a \mathbf u) \cdot \langle \mathbf v_i \rangle \\
			&= a (\mathbf u \cdot \langle \mathbf v_i \rangle) \\
			&= af(\mathbf u).
		\end{aligned}
		$$
		
		By Definition \ref{def: linear map}, $f$ is linear.
	\end{proof}
\end{lemma}


%------------------------------------------------
%================================================



\section{Linear Dependency}
%================================================
%------------------------------------------------



\begin{definition}
	\label{def: linear dependency}	
	A sequence $\langle\mathbf v_i\rangle_{i = 1}^{n}$ of vectors in $\mathbb R^m$ is said to be \textit{linearly independent} iff for any $\mathbf a \in \mathbb R^n \setminus \{\mathbf 0\}$,
	$$
	\mathbf a \cdot \langle\mathbf v_i\rangle \ne \mathbf 0.
	$$
	
	$\langle\mathbf v_i\rangle$ is \textit{linearly dependent} iff it is not linearly independent.
\end{definition}



\begin{lemma}
	\label{lm: linear dependency: alt def}
	With the condition above, $\langle\mathbf v_i\rangle_{i = 1}^n$ is linearly dependent iff there exists $k \in \{1, \ldots, n\}$ such that there exists $\mathbf a \in \mathbb R^n \setminus \{\mathbf 0\}$ with $a_k = 0$, such that
	$$
	\mathbf v_k = \mathbf a \cdot \langle \mathbf v_i \rangle.
	$$
	
	\begin{proof}
		$$
		\begin{aligned}
			\mathbf v_k = \mathbf a \cdot \langle \mathbf v_i \rangle &\iff \mathbf 0 = \mathbf a \cdot \langle \mathbf v_i \rangle - \mathbf v_k \\
			&\iff \mathbf 0 = (a_1 \mathbf v_1 + \cdots + 0 \mathbf v_k + \cdots + a_n \mathbf v_n) - \mathbf v_k \\
			&\iff \mathbf 0 = (a_1 \mathbf v_1 + \cdots + (-1) \mathbf v_k + \cdots + a_n \mathbf v_n) \\
			&\iff \mathbf 0 = (a_1, \ldots, a_{k - 1}, -1, a_{k + 1}, \ldots, a_n) \cdot \langle \mathbf v_i \rangle.
		\end{aligned}
		$$
	\end{proof}
\end{lemma}


\begin{note}
	Lemma \ref{lm: linear dependency: alt def} can define Definition \ref{def: linear dependency} in a more geometric way.
\end{note}


\begin{lemma}
	Let $\langle \mathbf v_i \rangle_{i = 1}^n$ be a sequence of vectors in $\mathbb R^m$, where $n < m$.
	
	Then $\mathrm{span}\langle \mathbf v_i \rangle$ is homeomorphic to $\mathbb R^n$.
	
	\begin{proof}
		Let $f: \mathbb R^n \to \mathrm{span}\langle \mathbf v_i \rangle$ be defined as
		$$
		f(\mathbf x) := \mathbf x \cdot \langle \mathbf v_i \rangle.
		$$
		
		By Lemma \ref{lm: span: is linear map}, as $\mathrm{span}\langle \mathbf v_i \rangle \subseteq \mathbb R^m$, $f$ is linear. By Lemma \ref{lm: linear map: is injective}, $f$ is injective.
		
		By Definition \ref{def: span}, for any $\mathbf u \in \mathrm{span} \langle \mathbf v_i \rangle$, there is an $\mathbf a \in \mathbb R^n$, such that
		$$
		\mathbf u = \mathbf a \cdot \langle \mathbf v_i \rangle,
		$$
		Thus $f$ is surjective.
		
		Now, it is proved that $f$ is bijective.
		
		...
	\end{proof}
\end{lemma}


\begin{lemma}
	Let $\langle \mathbf v_i \rangle_{i = 1}^n$ be a sequence of vectors in $\mathbb R^n$ (notice the $n$ here).
	
	Then $\langle \mathbf v_i \rangle$ is linearly independent iff
	$$
	\mathrm{span} \langle \mathbf v_i \rangle = \mathbb R^n.
	$$
	
	\begin{proof}
		Assume $\langle \mathbf v_i \rangle$ is linearly independent.
		
		By Definition \ref{def: span}, for any $\mathbf a \in \mathbb R^n$,
		$$
		\mathbf a \cdot \langle \mathbf v_i \rangle \in \mathrm{span}\langle \mathbf v_i \rangle.
		$$
		So $\mathbb R^n \subseteq \mathrm{span} \langle \mathbf v_i \rangle$. As $\mathrm{span} \langle \mathbf v_i \rangle \subseteq \mathbb R^n$ also, we have
		$$
		\mathrm{span} \langle \mathbf v_i \rangle = \mathbb R^n.
		$$
		
		\qedlm
		
		Conversely, assume $\mathrm{span} \langle \mathbf v_i \rangle = \mathbb R^n$, but $\mathrm{span}\langle \mathbf v_i \rangle$ is not linearly independent.
		
		Let $\langle \mathbf v_{i_k} \rangle_{i = 1}^{n - 1}$ be a subsequence of $\langle \mathbf v_i \rangle$, and assume $\langle \mathbf v_{i_k} \rangle$ is linearly independent. 
	\end{proof}
\end{lemma}




%------------------------------------------------
%================================================


%::::::::::::::::::::::::::::::::::::::::::::::::
%================================================










































%