
\chapter{Set Theory}
%================================================
%::::::::::::::::::::::::::::::::::::::::::::::::



\section{Unused!!!!!!}
%================================================
%------------------------------------------------


\begin{lemma}
	\label{power set of subset}
	
	Let $X$ be any sets.
	
	Then, $\forall X \subseteq Y: X$.
\end{lemma}


\begin{lemma}
	\label{lm: subset of the union of a family and powerset}

	Let $X$ be any set, let $A \subseteq X$, and let $\mathcal F \subseteq 2^X$.
	
	$\forall B \subseteq \bigcup( \mathcal F \cap 2^A): B \subseteq A$.
	
	\begin{proof}
		$$
		\forall B \subseteq \bigcup( \mathcal F \cap 2^A) \implies \exists S \in \mathcal F \cap 2^A: B \subseteq S.
		$$
		$S \in \mathcal F \cap 2^A$ implies $S \in 2^A$, thus $S \subseteq A$. As $B \subseteq S$, we have $B \subseteq A$.
	\end{proof}
\end{lemma}


\begin{lemma}
	\label{lm: intersection of power sets}
	
	Let $X$ be any set, and let $\mathcal A \subseteq 2^X$.
	
	Then, we have
	$$
	\bigcap_{A \in \mathcal A} 2^A = 2^{\bigcap \mathcal A}.
	$$
	
	\begin{proof}
		Let $S \in A$, then we have
		$$
		\begin{aligned}
			S \in 2^{\bigcap \mathcal A} &\iff S \subseteq \bigcap \mathcal A \\
			&\iff \bigwedge_{A \in \mathcal A} S \subseteq A \\
			&\iff \bigwedge_{A \in \mathcal A} S \in 2^A \\
			&\iff S \in \bigcap_{A \in \mathcal A} 2^A.
		\end{aligned}
		$$
	\end{proof}
\end{lemma}


\begin{lemma}
	\label{lm: union of intersections of indexed families}
	Let $X$ be any set, and let
	$$
	\forall i \in I: \mathcal A_i \subseteq 2^X,
	$$
	where $I$ is an indexed set.
	
	Then, we have
	$$
	\bigcup \left( \bigcap_{i \in I} \mathcal A_i \right) \subseteq \bigcap_{i \in I} \left( \bigcup \mathcal A_i \right).
	$$
\end{lemma}



\chapter{Topological Spaces}
%================================================
%::::::::::::::::::::::::::::::::::::::::::::::::


\section{Topological Spaces}
%================================================
%------------------------------------------------


\begin{definition}
	\label{def: topological spaces}
	Let $X$ be any set. A family $\mathcal T \subseteq 2^X$ is a \textit{topology} on $X$, iff it satisfies the \textit{open set axioms}. That is,
	\begin{enumerate}[O1.]
		\item $X \in \mathcal T$;
		\item $\forall \mathcal A \subseteq \mathcal T: \bigcup \mathcal A \in \mathcal T$ ($\mathcal T$ is closed under arbitrary union);
		\item $\forall \mathcal B \subseteq \mathcal T: \#\mathcal B < \aleph_0: \bigcap \mathcal B \in \mathcal T$ ($\mathcal T$ is closed under finite intersection).
	\end{enumerate}
	
	The ordered pair $\mathbb X = (X, \mathcal T)$ is called a \textit{topological space}.
	
	A subset $A \subseteq X$ is said to be \textit{open} iff $A \in \mathcal T$.
\end{definition}


\begin{lemma}
	\label{lm: topological spaces: emptyset in topology}

	For any topology $\mathcal T$, $\emptyset \in \mathcal T$.
	
	\begin{proof}
		As $\emptyset$ is a subset of any set, $\emptyset \subseteq \mathcal T$, and
		$$
		\bigcup \emptyset = \emptyset,
		$$
		by Open Set Axiom O2, $\emptyset \in \mathcal T$.
	\end{proof}
\end{lemma}


\begin{definition}
	Let $X$ be any set and let $\mathcal T$ and $\mathcal T'$ be topologies on $X$.
	
	$\mathcal T$ is said to be \textit{coarser} than $\mathcal T'$, or $\mathcal T'$ is said to be \textit{finer} than $\mathcal T$, iff $\mathcal T \subseteq \mathcal T'$.
\end{definition}


\begin{example}
	Let $X$ be a topological space. $2^X$ itself is a \textit{discrete topology} on $X$. It is the finest topology on $X$. The \textit{indiscrete topology} on $X$ is $\{ \emptyset, \mathcal T \}$. It is the coarsest topology on $X$.
\end{example}


\begin{definition}
	\label{def: closed sets}
	Let $\mathbb X = (X, \mathcal T)$ be a topological space, and let $A \subseteq X$.
	
	$A$ is said to be \textit{closed} in $\mathbb X$ iff
	$$
	\exists U \in \mathcal T: A = X \setminus U.
	$$
\end{definition}


\begin{lemma}
	\label{lm: closed sets axioms}
	
	Let $\mathbb X = (X, \mathcal T)$ be a topological spaces, and let $\mathcal C$ be the family of all closed sets in $\mathbb X$. Then,
	\begin{enumerate}[C1.]
		\item $\emptyset, X \in \mathcal C$;
		\item $\forall \mathcal A \subseteq \mathcal C: \bigcap \mathcal A \in \mathcal C$ ($\mathcal C$ is closed under arbitrary intersection);
		\item $\forall \mathcal B \subseteq \mathcal C: \# \mathcal B < \aleph_0: \bigcup \mathcal B \in \mathcal C$ ($\mathcal C$ is closed under finite union).
	\end{enumerate}
	
	\begin{proof}		
		\begin{enumerate}[C1.]
			\item
			By Definition \ref{def: closed sets}, as $X \in \mathcal T$ and $\emptyset = X \setminus X$, we have $\emptyset \in \mathcal C$. Similarly, as $\emptyset \in \mathcal T$, $X = X \setminus \emptyset$, we have $X \in \mathcal C$.
				\qedlm
				
			\item
			By Definition \ref{def: closed sets}, $\forall \mathcal A \subseteq \mathcal C: \exists \mathcal A' \subseteq \mathcal T:$
			$$
			\mathcal A = \left\{ X \setminus A : A \in \mathcal A' \right\}.
			$$
	
			Then, by De Morgan's Law,
			$$
			\bigcap \mathcal A = X \setminus \bigcup \mathcal A'.
			$$
			
			By Open Set Axiom O2, we have $\bigcup \mathcal A' \in \mathcal T$. Then, by Definition \ref{def: closed sets}, we have $\bigcap \mathcal B \in \mathcal C$.
			\qedlm
			
			\item
			By Definition \ref{def: closed sets}, $\forall \mathcal B \subseteq \mathcal C: \#\mathcal B < \aleph_0: \exists \mathcal B' \subseteq \mathcal T:$
			$$
			\mathcal B = \left\{ X \setminus B : B \in \mathcal B' \right\}.
			$$
			
			Then, by De Morgan's Law
			$$
			\bigcup \mathcal B = X \setminus \bigcap \mathcal B'.
			$$
			
			As $\# \mathcal B' = \# \mathcal B < \aleph_0$, (indeed, $\mathcal B \leftrightarrow \mathcal B'$ here can be considered as a bijection), by Open Set Axioms O2, $\bigcap \mathcal B' \in \mathcal T$. By Definition \ref{def: closed sets}, $\bigcup \mathcal B \in \mathcal C$.
		\end{enumerate}
	\end{proof}
\end{lemma}


\section{Interior}
%================================================
%------------------------------------------------


\begin{definition}
	\label{def: interior}
	
	Let $\mathbb X = (X, \mathcal T)$ be a topological space, and let $A \subseteq X$.
	
	The \textit{interior} of $A$ is defined as
	$$
	A^\circ := \left\{ x \in \bigcup (\mathcal T \cap 2^A) \right\}.
	$$
\end{definition}

\begin{note}
	Sometime, we write $A^\circ_\mathbb X$ or $A^\circ_\mathcal T$ for $A^\circ$ for reminding in which topological space the interior of $A$ is. For example, for topological spaces $\mathbb X_1 = (\mathbb R, \{\emptyset, X\})$ and $\mathbb X_2 = (\mathbb R, 2^X)$,
	$$
	(0,1]^\circ_{\mathbb X_1} = \emptyset, \text{ but } (0,1]^\circ_{\mathbb X_1} = (0,1].
	$$
\end{note}


\begin{lemma}
	\label{lm: interior: subset of the set}
	Let $\mathbb X = (X, \mathcal T)$ be a topological space, and let $A \subseteq X$.
	
	Then $A^\circ \subseteq A$.
	
	\begin{proof}
		By Definition \ref{def: interior}, $\forall U \subseteq A^\circ: U \subseteq \bigcup(\mathcal T \cap 2^A)$. By Lemma \ref{lm: subset of the union of a family and powerset}, $U \subseteq A$. As $A^\circ \subseteq A^\circ$ also, $A^\circ \subseteq A$.
	\end{proof}
\end{lemma}


\begin{lemma}
	Let $\mathbb X = (X, \mathcal T)$ be a topological space, and let $A, B \subseteq X$.
	
	If $A \subseteq B$, then $A^\circ \subseteq B^\circ$.
	
	\begin{proof}
		$$
		\begin{aligned}
			A \subseteq B &\implies 2^A \subseteq 2^B \\
			&\implies \mathcal T \cap 2^A \subseteq \mathcal T \cap 2^B \\
			&\implies \bigcup (\mathcal T \cap 2^A) \subseteq \bigcup (\mathcal T \cap 2^B) \\
			&\implies A^\circ \subseteq B^\circ.
		\end{aligned}
		$$
	\end{proof}
\end{lemma}


\begin{note}
	$A^\circ \subseteq B^\circ$ does not implies $A \subseteq B$.
	
	Let $\mathbb X = (\{1,2\}, \{\emptyset, X, \{2\} \})$, and let $A = \{1\}$ and $B = \{2\}$. Then $A^\circ = \emptyset \subseteq B = \{2\}$, but $A \not \subseteq B$.
\end{note}


\begin{lemma}
	Let $\mathbb X = (X, \mathcal T)$ be a topological space, and let $A \subseteq X$.
	
	If $A \in \mathcal T$, then $A = A^\circ$.
	
	\begin{proof}
		Assume $A \in \mathcal T$.
		
		As $A \in 2^A$ and $A \in \mathcal T$, by Definition \ref{def: interior}, we have
		$$
		A \in \mathcal T \cap 2^A \implies A \subseteq A^\circ.
		$$
		
		By Lemma \ref{lm: interior: subset of the set}, $A^\circ \subseteq A$. Therefore, $A = A^\circ$.
		\qedlm
		
		Conversely, Assume $A = A^\circ$.
		
		By Definition \ref{def: interior}, we have
		$$
		A = \bigcup (\mathcal T \cap 2^A) = \bigcup_{U \in \mathcal T \land U \subseteq A} U
		$$
		
		As $U \in \mathcal T$, by Open Axioms O2, $A \in \mathcal T$.
	\end{proof}
\end{lemma}


\begin{lemma}
	\label{lm: interior of union}
	Let $\mathbb X = (X, \mathcal T)$ be a topological space, and let $\mathcal A \subseteq 2^X$.
	
	Then, we have
	$$
	\left( \bigcap \mathcal A \right)^\circ \subseteq \bigcap_{A \in \mathcal A} A^\circ.
	$$
	
	\begin{proof}
		$$
		\begin{aligned}
			\left( \bigcap \mathcal A \right)^\circ &= \bigcup \left(\mathcal T \cap 2^{\bigcap \mathcal A}\right)
			& \text{(Definition \ref{def: interior})}
			\\
			&= \bigcup \left( \mathcal T \cap \bigcap_{A \in \mathcal A} 2^A \right)
			& \text{(Lemma \ref{lm: intersection of power sets})}
			\\
			&= \bigcup \left( \bigcap_{A \in \mathcal A} (\mathcal T \cap 2^A) \right) \\
			&\subseteq \bigcap_{A \in \mathcal A} \left( \bigcup (\mathcal T \cap 2^A) \right) 
			&\text{(Lemma \ref{lm: union of intersections of indexed families})}
			\\
			&= \bigcap_{A \in \mathcal A} A^\circ.
			&\text{(Definition \ref{def: interior})}
		\end{aligned}
		$$
	\end{proof}
\end{lemma}


\begin{note}
	The equality in Lemma \ref{lm: interior of union} may not hold.
	
	Let $\mathbb X = (\mathbb R, \mathcal T)$ be a topological space with
	$$
	\mathcal T = \{\emptyset, X, (0,2), (1,3)\}.
	$$
	
	Then we have
	$$
	((0,2) \cap (1,3))^\circ = \emptyset \quad \not \subseteq \quad (0, 2)^\circ \cap (1, 3)^\circ = (1,2).
	$$
\end{note}


\begin{lemma}
	Let $\mathbb X = (X, \mathcal T)$ be a topological space, and let $A, B \subseteq X$.
	
	If $A \subseteq B$, then $A^\circ \subseteq B^\circ$.
	
	\begin{proof}
		$$
		\begin{aligned}
			A \subseteq B &\implies 2^A \subseteq 2^B
		\end{aligned}
		$$
	\end{proof}
\end{lemma}


%------------------------------------------------
%================================================



---

---

---

---

---

---

---

---

---

---

---

---





%::::::::::::::::::::::::::::::::::::::::::::::::
%================================================










































%