
\chapter{Differentiation}
%================================================
%::::::::::::::::::::::::::::::::::::::::::::::::


\section{Differentiable Mapping}
%================================================
%------------------------------------------------


\begin{observation}
	Let $f : \mathbb R^m \to \mathbb R^n$, and denote $f_i$ for the $i$-th factor of $f$, i.e., $f = \langle f_i \rangle_i^n$. Assume $f_i$ is smooth at a point $\mathbf p \in \mathbb R^m$.

	Intuitively, $f_i$ is smooth at $\mathbf p$ iff there exists a neighbourhood $N$ of $\mathbf p$ and a plane described by $P_i: \mathbb R^m \to \mathbb R$, such that
	$$
		f_i[N] \approx P_i[N].
	$$
	As $\mathbb R^m$ is considered as a metric space, any open ball $B(\mathbf p, \delta) \subseteq N$ is also required neighbourhood of $\mathbf p$. In this sense, the approximation can be considered as,
	$$
	\lim_{\delta \to 0} f_i[B(\mathbf p, \delta)] = \lim_{\delta \to 0} P_i[B(\mathbf p, \delta)].
	$$
	In the term of elements, that is, there exists $\delta \in \mathbb R_{> 0}$, such that for any $\mathbf t \in \mathbb R^m$ with $\mathbf p + \mathbf t \in B(\mathbf p, \delta)$,	
	$$
	\lim_{\mathbf t \to \mathbf 0_{\mathbb R^m}} f_i(\mathbf p + \mathbf t) = \lim_{\mathbf t \to \mathbf 0_{\mathbb R^m}} P_i(\mathbf p + \mathbf t).
	$$
	
	As $P_i$ describes a plane, it can be considered as a translated linear mapping, and as this plane must be a tangent plane of $f_i$ at $\mathbf p$, there exists a linear mapping $\phi_i: \mathbb R^m \to \mathbb R$ such that
	$$
	P_i(\mathbf p + \mathbf t) = \phi_i(\mathbf p + \mathbf t - \mathbf p) + f_i(\mathbf p).
	$$
	Thus, we have
	$$
	\lim_{\mathbf t \to \mathbf 0_{\mathbb R^m}} f_i(\mathbf p + \mathbf t) = \lim_{\mathbf t \to \mathbf 0_{\mathbb R^m}}\phi_i(\mathbf t) + f_i(\mathbf p).
	$$
	Rearrange the equation, we have
	$$
	\lim_{\mathbf t \to \mathbf 0_{\mathbb R^m}} \frac{f_i(\mathbf p + \mathbf t) - f_i(\mathbf p)}{\| \mathbf t \|_{\mathbb R^m}} = \lim_{\mathbf t \to \mathbf 0_{\mathbb R^m}} \frac{\phi_i(\mathbf t)}{\| \mathbf t \|_{\mathbb R^m}}.
	$$
	As $\phi_i$ is linear, the right hand side of the equation is a constant in $\mathbb R$. Thus, by rearrange the equation again, we have
	$$
	\lim_{\mathbf t \to \mathbf 0_{\mathbb R^m}} \frac{f_i(\mathbf p + \mathbf t) - f_i(\mathbf p) - \phi_i(\mathbf t)}{\|\mathbf t\|_{\mathbb R^m}} = 0
	$$
	
	The last equation describes the smoothness of $f_i$ at $\mathbf p$, in calculus, $f_i$ is said to be differentiable at $\mathbf p$.
	
	Now, assume for any $i \in \{1, \ldots, n\}$, $f_i$ is differentiable at $\mathbf p$. That is, for any $f_i$, there exists a $\phi_i:\mathbb R^m \to \mathbb R$, such that
	$$
	\left\langle\lim_{\mathbf t \to \mathbf 0_{\mathbb R^m}} \frac{f_i(\mathbf p + \mathbf t) - f_i(\mathbf p) - \phi_i(\mathbf t)}{\|\mathbf t\|_{\mathbb R^m}} \right\rangle_{i = 1}^n = \langle 0 \rangle_{i = 1}^n.
	$$
	By vector sum, we have
	$$
	\lim_{\mathbf t \to \mathbf 0_{\mathbb R^m}} \frac{f(\mathbf p + \mathbf t) - f(\mathbf p) - \phi(\mathbf t)}{\|\mathbf t\|_{\mathbb R^m}} = \mathbf 0_{\mathbb R^n},
	$$
	where $\phi = \langle \phi_i \rangle_{i = 1}^n$.
	
	In this sense, Definition \ref{def: differentiable mappings} is introduced as following.
\end{observation}


\begin{definition}
	\label{def: differentiable mappings}
	Let $f:\mathbb R^m \to \mathbb R^n$.
	
	$f$ is said to be \textit{differentiable} at a point $\mathbf p \in \mathbb R^m$ iff for any $\mathbf u \in \mathbb R^m \setminus \{0_{\mathbb R^m}\}$, there exists a linear mapping $\phi: \mathbb R^m \to \mathbb R^n$ such that
	$$
	\lim_{\mathbf t \to \mathbf 0_{\mathbb R^m}}\frac{f(\mathbf p + \mathbf t) - f(\mathbf p) - \phi(\mathbf t)}{\| \mathbf t \|_{\mathbb R^m}} = \mathbf 0_{\mathbb R^n}.
	$$
\end{definition}


\begin{lemma}
	\label{lm: differentiable mappings: uniqueness of phi}
	In Definition \ref{def: differentiable mappings}, $\phi$ is unique.
	
	\begin{proof}
		As Definition holds for $\phi$, there exists an $\alpha: \mathbb R^{m} \to \mathbb R^n$ with
		$$
		\lim_{\mathbf t \to \mathbf 0_{\mathbb R^m}} \alpha(\mathbf t) = \alpha(\mathbf 0_{\mathbb R^m}) = \mathbf 0_{\mathbb R^n},
		$$
		and a neighbourhood $N$ of $\mathbf p$ such that for any $\mathbf t \in \mathbb R^m$ with $\mathbf p + \mathbf t \in N$,
		$$
		\frac{f(\mathbf p + \mathbf t) - f(\mathbf p) - \phi(\mathbf t)}{\|\mathbf t\|_{\mathbb R^m}} = \alpha(\mathbf t).
		$$
	
		Suppose Definition \ref{def: differentiable mappings} also holds for another $\lambda: \mathbb R^m \to \mathbb R^n$, then
		there exists an $\beta: \mathbb R^m \to \mathbb R^n$ with
		$$
		\lim_{\mathbf t \to \mathbf 0_{\mathbb R^m}} \beta(\mathbf t) = \beta(\mathbf 0_{\mathbb R^m}) = \mathbf 0_{\mathbb R^n},
		$$
		and a neighbourhood $N'$ of $\mathbf p$ such that for any $\mathbf t \in \mathbb R^m$ with $\mathbf p + \mathbf t \in N'$,
		$$
		\frac{f(\mathbf p + \mathbf t) - f(\mathbf p) - \lambda(\mathbf t)}{\|\mathbf t\|_{\mathbb R^m}} = \beta(\mathbf t).
		$$
		
		Let $\gamma = \phi - \lambda$. As $\phi$ and $-\lambda$ are both linear, $\gamma$ is also linear. Then, we have
		$$
		\begin{aligned}
		& \ \frac{\gamma(\mathbf t)}{\| \mathbf t \|_{\mathbb R^m}} = \alpha(\mathbf t) - \beta(\mathbf t) \\
		\iff & \lim_{\mathbf t \to \mathbf 0_{\mathbb R^m}} \gamma(\mathbf{\hat t}) = \lim_{\mathbf t \to \mathbf 0_{\mathbb R^m}} (\alpha(\mathbf t) - \beta(\mathbf t)) \\
		\iff& \ \gamma(\mathbf{\hat t}) = \mathbf 0_{\mathbb R^n}.
		\end{aligned}		
		$$
		
		As $\mathbf t$ is arbitrarily picked from $U \cap U'$, and $U \cap U'$ is open in $\mathbb R^m$ as $U$ and $U'$ are open, the set $\left\{ \mathbf{\hat t} : \mathbf t \in U \cap U' - \mathbf p \right\}$ gives all possible directions in $\mathbb R^m$. And, as $\gamma(s \mathbf{\hat t}) = \mathbf 0_{\mathbb R^n}$ for any $\mathbf t \in \mathbb R^m$ and any $s \in \mathbb R$, $\gamma[\mathbb R^m] = \{\mathbf 0_{\mathbb R^m}\}$. Thus, $\phi = \lambda$.
	\end{proof}
\end{lemma}


\begin{lemma}
	\label{lm: differentiable mappings: implies continuity}
	With the condition in Definition \ref{def: differentiable mappings}, if $f$ is differentiable at $\mathbf p$, then $f$ is continuous at $\mathbf p$.
	
	\begin{proof}
		As $f$ is differentiable at $\mathbf p$, there exists an $\alpha: \mathbb R^m \to \mathbb R^n$ with
		$$
		\lim_{\mathbf t \to \mathbf 0_{\mathbb R^m}} \alpha(\mathbf t) = \alpha(\mathbf 0_{\mathbb R^m}) = \mathbf 0_{\mathbb R^m},
		$$
		such that
		$$
		\frac{f(\mathbf p + \mathbf t) - f(\mathbf p) - \phi(\mathbf t)}{\| \mathbf t \|_{\mathbb R^m}} = \alpha(\mathbf t).
		$$
		By rearranging the equation, we observe
		$$
		\begin{aligned}
		& \ \lim_{\mathbf t \to \mathbf 0_{\mathbb R^m}}[f(\mathbf p + \mathbf t) - \phi(\mathbf t)] = \lim_{\mathbf t \to \mathbf 0_{\mathbb R^m}} [\| \mathbf t \|_{\mathbb R^m}\alpha(\mathbf t) + f(\mathbf p)] \\
		\iff & \ \lim_{\mathbf t \to \mathbf 0_{\mathbb R^m}} f(\mathbf p + \mathbf t) = f(\mathbf p).
		\end{aligned}
		$$
		
		Thus, $f$ is continuous at $\mathbf p$.
	\end{proof}
\end{lemma}


%------------------------------------------------
%================================================


\section{Directional Derivatives}
%================================================
%------------------------------------------------


\begin{observation}
	Let $f: \mathbb R^m \to \mathbb R^n$, and let $g:\mathbb R \to \mathbb R^m$ be defined as
	$$
	g(t) := \mathbf p + t \mathbf u,
	$$
	where $\mathbf p , \mathbf u \in \mathbb R^m$ and $\mathbf u \ne \mathbf 0_{\mathbb R^m}$.
	
	Let $h = f \circ g$.
\end{observation}


%------------------------------------------------
%================================================



%::::::::::::::::::::::::::::::::::::::::::::::::
%================================================










































%