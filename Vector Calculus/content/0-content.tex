
\chapter{Differentiation}
%================================================
%::::::::::::::::::::::::::::::::::::::::::::::::


\section{Infintesimal}
%================================================
%------------------------------------------------


\begin{definition}
	\label{def: little-o}
	Let $f, g: \mathbb R^m \to \mathbb R^n$, and let $\mathbf p \in \mathbb R^m$.
	
	Then $f$ is a \textit{little-o} of $g$ as $\mathbf x \to \mathbf p$, i.e.,
	$$
	f(\mathbf x) = o(g(\mathbf x)) \text{ as } \mathbf x \to \mathbf p,
	$$
	iff for any $\varepsilon \in \mathbb R_{> 0}$, there exists a neighbourhood of $U$ of $\mathbf p$ such that for any $\mathbf x \in U$, $\| f(\mathbf x) \| \le \varepsilon \| g(\mathbf x) \|$. Equivalently, that is,
	$$
	\lim_{\mathbf x \to \mathbf p} \frac{f(\mathbf x)}{\| g(\mathbf x) \|_{\mathbb R^n}} = \mathbf 0_{\mathbb R^n}.
	$$
\end{definition}


\begin{note}
	In the case that $f(\mathbf x) = o(g(\mathbf x))$ as $\mathbf x \to \mathbf 0_{\mathbb R^m}$, I will simply write $f(\mathbf x) = o(g(\mathbf x))$.
\end{note}


\begin{lemma}
	$$
	o(f(\mathbf x)) + o(g(\mathbf x)) = o(\| f(\mathbf x) \|_{\mathbb R^n} + \| g(\mathbf x) \|_{\mathbb R^n}).
	$$
	
	\begin{proof}
		By Definition \ref{def: little-o}, for any $\varepsilon \in \mathbb R_{> 0}$, there exists a neighbourhood of $U$ of $\mathbf p$ such that for any $\mathbf x \in U$,
		$$
		\| o(f(\mathbf x)) \|_{\mathbb R^n} \le \varepsilon \| f(\mathbf x) \|.
		$$
		
		Then, there exists some $\mathbf u, \mathbf v \in \mathbb R^n$ such that
		$$
		o(f(\mathbf x)) = \varepsilon\| f(\mathbf x) \| \mathbf u \text{ and } o(g(\mathbf x)) = \varepsilon\| g(\mathbf x) \| \mathbf v.
		$$
		

		....
		
		By Definition \ref{def: little-o}, now we have
		$$
		o(f(\mathbf x)) + o(g(\mathbf x))
		$$
	\end{proof}
\end{lemma}


%------------------------------------------------
%================================================



\section{Differentiable Mapping}
%================================================
%------------------------------------------------


\begin{definition}
	\label{def: differentiable}
	Let $f: \mathbb R^m \to \mathbb R^n$, and let $\mathbf p \in \mathbb R^m$.
	
	Then, $f$ is said to be \textit{differentiable} at $\mathbf p$,  iff there exists a linear map $\phi: \mathbb R^m \to \mathbb R^n$ and an open subset $U \subseteq \mathbb R^m$, such that for any $\mathbf h \in \mathbb R^m \setminus \{\mathbf 0_{\mathbb R^m}\}$ with $\mathbf p + \mathbf h \in U$,
	$$
	f(\mathbf p + \mathbf h) = f(\mathbf p) + \phi(\mathbf h) + o(\phi(\mathbf h)).
	$$
\end{definition}


\begin{lemma}
	The relation in Definition \ref{def: differentiable} holds for a unique $\phi$.

	\begin{proof}
		Aiming for a contradiction, suppose there is another linear map $\lambda: \mathbb R^m \to \mathbb R^n$ such that
		$$
		f(\mathbf p + \mathbf h) = f(\mathbf p) + \lambda(\mathbf h) + o(\lambda(\mathbf h)),
		$$
		then we have
		$$
		\phi(\mathbf h) - \lambda(\mathbf h) = o(\phi(\mathbf h)) - o(\lambda(\mathbf h)).
		$$
		
		Let $g: \mathbb R \to \mathbb R^n$ be defined as $g(t) := \phi(t\mathbf u)$, then
	\end{proof}
\end{lemma}


%------------------------------------------------
%================================================



\section{Derivatives}
%================================================
%------------------------------------------------


\begin{definition}
	\label{def: directional derivatives}
	Let $f: \mathbb R^m \to \mathbb R^n$, let $\mathbf u \in \mathbb R^m \setminus \{\mathbf 0_{\mathbb R^m}\}$, and let $\mathbf p \in \mathbb R^m$.
	
	The \textit{directional derivative} of $f$ along $\mathbf u$ at $\mathbf p$ is defined as
	$$
	\nabla_{\mathbf u} f(\mathbf p) := \lim_{h \to 0} \frac{f(\mathbf p + h\mathbf u) - f(\mathbf p)}{h},
	$$
	if the limit exists in $\mathbb R^n$.
\end{definition}


\begin{lemma}
	\label{lm: direction derivative exists implies relative continuous}
	With the conditions in Definition \ref{def: directional derivatives}, if $\nabla_{\mathbf u}f(\mathbf p)$ exists at $\mathbf p$, then there exists open subset $U \subseteq \mathbb R^m$ such that $f$ is relative continuous on $U \cap \{\mathbf p + h \mathbf u : h\in \mathbb R\}$ for some $U$.
	
	\begin{proof}
		
	\end{proof}
\end{lemma}


\begin{lemma}
	With the conditions in Definition \ref{def: directional derivatives}, let $s \in \mathbb R \setminus \{0\}$, then
	$$
	\nabla_{s\mathbf u}f(\mathbf p) = s \nabla_{\mathbf u} f(\mathbf p)
	$$
	if $\nabla_{\mathbf u}f(\mathbf p)$ exists in $\mathbb R^n$.
	
	\begin{proof}
		Let $g: \mathbb R \to \mathbb R^n$ be defined as
		$$
		g(h) := f(\mathbf p + h\mathbf u).
		$$
		Then, we have
		$$
		\begin{aligned}
			\nabla_{s\mathbf u} f(\mathbf p) &= \lim_{h \to 0} \frac{f(\mathbf p + hs\mathbf u) - f(\mathbf p)}{h} \\
			&= \lim_{h \to 0} \frac{g(sh) - g(0)}{h}.
		\end{aligned}
		$$
		
		As this is a $0/0$ limit, thus, by L'H\'opital's rule, we have
		$$
		\begin{aligned}
			\nabla_{s \mathbf u} f(\mathbf p) &= \lim_{h \to 0} \frac{\mathrm dg(hs)}{\mathrm dh}. \\
		\end{aligned}
		$$
		
		...
	\end{proof}
\end{lemma}


%------------------------------------------------
%================================================


%::::::::::::::::::::::::::::::::::::::::::::::::
%================================================










































%