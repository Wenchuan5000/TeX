
\chapter{Differentiation}
%================================================
%::::::::::::::::::::::::::::::::::::::::::::::::


\section{Differentiable Mapping}
%================================================
%------------------------------------------------


\begin{observation}
	Let $f : \mathbb R^m \to \mathbb R^n$, and denote $f_i$ for the $i$-th factor of $f$, i.e., $f = \langle f_i \rangle_i^n$. Assume $f_i$ is smooth at a point $\mathbf p \in \mathbb R^m$.

	Intuitively, $f_i$ is smooth at $\mathbf p$ iff there exists a neighbourhood $N$ of $\mathbf p$ and a plane described by $P_i: \mathbb R^m \to \mathbb R$, such that
	$$
		f_i[N] \approx P_i[N].
	$$
	As $\mathbb R^m$ is considered as a metric space, any open ball $B(\mathbf p, \delta) \subseteq N$ is also required neighbourhood of $\mathbf p$. In this sense, the approximation can be considered as,
	$$
	\lim_{\delta \to 0} f_i[B(\mathbf p, \delta)] = \lim_{\delta \to 0} P_i[B(\mathbf p, \delta)].
	$$
	In the term of elements, that is, there exists $\delta \in \mathbb R_{> 0}$, such that for any $\mathbf t \in \mathbb R^m$ with $\mathbf p + \mathbf t \in B(\mathbf p, \delta)$,	
	$$
	\lim_{\mathbf t \to \mathbf 0_{\mathbb R^m}} f_i(\mathbf p + \mathbf t) = \lim_{\mathbf t \to \mathbf 0_{\mathbb R^m}} P_i(\mathbf p + \mathbf t).
	$$
	
	As $P_i$ describes a plane, it can be considered as a translated linear mapping, and as this plane must be a tangent plane of $f_i$ at $\mathbf p$, there exists a linear mapping $\phi_i: \mathbb R^m \to \mathbb R$ such that
	$$
	P_i(\mathbf p + \mathbf t) = \phi_i(\mathbf p + \mathbf t - \mathbf p) + f_i(\mathbf p).
	$$
	Thus, we have
	$$
	\lim_{\mathbf t \to \mathbf 0_{\mathbb R^m}} f_i(\mathbf p + \mathbf t) = \lim_{\mathbf t \to \mathbf 0_{\mathbb R^m}}\phi_i(\mathbf t) + f_i(\mathbf p).
	$$
	Rearrange the equation, we have
	$$
	\lim_{\mathbf t \to \mathbf 0_{\mathbb R^m}} \frac{f_i(\mathbf p + \mathbf t) - f_i(\mathbf p)}{\| \mathbf t \|_{\mathbb R^m}} = \lim_{\mathbf t \to \mathbf 0_{\mathbb R^m}} \frac{\phi_i(\mathbf t)}{\| \mathbf t \|_{\mathbb R^m}}.
	$$
	As $\phi_i$ is linear, the right hand side of the equation is a constant in $\mathbb R$. Thus, by rearrange the equation again, we have
	$$
	\lim_{\mathbf t \to \mathbf 0_{\mathbb R^m}} \frac{f_i(\mathbf p + \mathbf t) - f_i(\mathbf p) - \phi_i(\mathbf t)}{\|\mathbf t\|_{\mathbb R^m}} = 0
	$$
	
	The last equation describes the smoothness of $f_i$ at $\mathbf p$, in calculus, $f_i$ is said to be differentiable at $\mathbf p$.
	
	Now, assume for any $i \in \{1, \ldots, n\}$, $f_i$ is differentiable at $\mathbf p$. That is, for any $f_i$, there exists a $\phi_i:\mathbb R^m \to \mathbb R$, such that
	$$
	\left\langle\lim_{\mathbf t \to \mathbf 0_{\mathbb R^m}} \frac{f_i(\mathbf p + \mathbf t) - f_i(\mathbf p) - \phi_i(\mathbf t)}{\|\mathbf t\|_{\mathbb R^m}} \right\rangle_{i = 1}^n = \langle 0 \rangle_{i = 1}^n.
	$$
	By vector sum, we have
	$$
	\lim_{\mathbf t \to \mathbf 0_{\mathbb R^m}} \frac{f(\mathbf p + \mathbf t) - f(\mathbf p) - \phi(\mathbf t)}{\|\mathbf t\|_{\mathbb R^m}} = \mathbf 0_{\mathbb R^n},
	$$
	where $\phi = \langle \phi_i \rangle_{i = 1}^n$.
	
	In this sense, Definition \ref{def: differentiable mappings} is introduced as following.
\end{observation}


\begin{definition}[Differentiable Mappings]
	\
	
	\label{def: differentiable mappings}
	Let $f:\mathbb R^m \to \mathbb R^n$.
	
	$f$ is said to be \textit{differentiable} at a point $\mathbf p \in \mathbb R^m$ iff for any $\mathbf u \in \mathbb R^m \setminus \{0_{\mathbb R^m}\}$, there exists a linear mapping $\phi: \mathbb R^m \to \mathbb R^n$ such that
	$$
	\lim_{\mathbf t \to \mathbf 0_{\mathbb R^m}}\frac{f(\mathbf p + \mathbf t) - f(\mathbf p) - \phi(\mathbf t)}{\| \mathbf t \|_{\mathbb R^m}} = \mathbf 0_{\mathbb R^n}.
	$$
\end{definition}


\begin{theorem}
	\label{thm: differentiable mappings: uniqueness of phi}
	In Definition \ref{def: differentiable mappings}, $\phi$ is unique.
	
	\begin{proof}
		As Definition holds for $\phi$, there exists an $\alpha: \mathbb R^{m} \to \mathbb R^n$ with
		$$
		\lim_{\mathbf t \to \mathbf 0_{\mathbb R^m}} \alpha(\mathbf t) = \alpha(\mathbf 0_{\mathbb R^m}) = \mathbf 0_{\mathbb R^n},
		$$
		and a neighbourhood $N$ of $\mathbf p$ such that for any $\mathbf t \in \mathbb R^m$ with $\mathbf p + \mathbf t \in N$,
		$$
		\frac{f(\mathbf p + \mathbf t) - f(\mathbf p) - \phi(\mathbf t)}{\|\mathbf t\|_{\mathbb R^m}} = \alpha(\mathbf t).
		$$
	
		Suppose Definition \ref{def: differentiable mappings} also holds for another $\lambda: \mathbb R^m \to \mathbb R^n$, then
		there exists an $\beta: \mathbb R^m \to \mathbb R^n$ with
		$$
		\lim_{\mathbf t \to \mathbf 0_{\mathbb R^m}} \beta(\mathbf t) = \beta(\mathbf 0_{\mathbb R^m}) = \mathbf 0_{\mathbb R^n},
		$$
		and a neighbourhood $N'$ of $\mathbf p$ such that for any $\mathbf t \in \mathbb R^m$ with $\mathbf p + \mathbf t \in N'$,
		$$
		\frac{f(\mathbf p + \mathbf t) - f(\mathbf p) - \lambda(\mathbf t)}{\|\mathbf t\|_{\mathbb R^m}} = \beta(\mathbf t).
		$$
		
		Let $\gamma = \phi - \lambda$. As $\phi$ and $-\lambda$ are both linear, $\gamma$ is also linear. Then, we have
		$$
		\begin{aligned}
		& \ \frac{\gamma(\mathbf t)}{\| \mathbf t \|_{\mathbb R^m}} = \alpha(\mathbf t) - \beta(\mathbf t) \\
		\iff & \lim_{\mathbf t \to \mathbf 0_{\mathbb R^m}} \gamma(\mathbf{\hat t}) = \lim_{\mathbf t \to \mathbf 0_{\mathbb R^m}} (\alpha(\mathbf t) - \beta(\mathbf t)) \\
		\iff& \ \gamma(\mathbf{\hat t}) = \mathbf 0_{\mathbb R^n}.
		\end{aligned}		
		$$
		
		As $\mathbf t$ is arbitrarily picked from $U \cap U'$, and $U \cap U'$ is open in $\mathbb R^m$ as $U$ and $U'$ are open, the set $\left\{ \mathbf{\hat t} : \mathbf t \in U \cap U' - \mathbf p \right\}$ gives all possible directions in $\mathbb R^m$. And, as $\gamma(s \mathbf{\hat t}) = \mathbf 0_{\mathbb R^n}$ for any $\mathbf t \in \mathbb R^m$ and any $s \in \mathbb R$, $\gamma[\mathbb R^m] = \{\mathbf 0_{\mathbb R^m}\}$. Thus, $\phi = \lambda$.
	\end{proof}
\end{theorem}


\begin{theorem}
	\label{thm: differentiable mappings: implies continuity}
	With the condition in Definition \ref{def: differentiable mappings}, if $f$ is differentiable at $\mathbf p$, then $f$ is continuous at $\mathbf p$.
	
	\begin{proof}
		As $f$ is differentiable at $\mathbf p$, there exists an $\alpha: \mathbb R^m \to \mathbb R^n$ with
		$$
		\lim_{\mathbf t \to \mathbf 0_{\mathbb R^m}} \alpha(\mathbf t) = \alpha(\mathbf 0_{\mathbb R^m}) = \mathbf 0_{\mathbb R^m},
		$$
		such that
		$$
		\frac{f(\mathbf p + \mathbf t) - f(\mathbf p) - \phi(\mathbf t)}{\| \mathbf t \|_{\mathbb R^m}} = \alpha(\mathbf t).
		$$
		By rearranging the equation, we observe
		$$
		\begin{aligned}
		& \ \lim_{\mathbf t \to \mathbf 0_{\mathbb R^m}}[f(\mathbf p + \mathbf t) - \phi(\mathbf t)] = \lim_{\mathbf t \to \mathbf 0_{\mathbb R^m}} [\| \mathbf t \|_{\mathbb R^m}\alpha(\mathbf t) + f(\mathbf p)] \\
		\iff & \ \lim_{\mathbf t \to \mathbf 0_{\mathbb R^m}} f(\mathbf p + \mathbf t) = f(\mathbf p).
		\end{aligned}
		$$
		
		Thus, $f$ is continuous at $\mathbf p$.
	\end{proof}
\end{theorem}


\begin{theorem}
	\label{thm: differentiable mappings: composed}
	With the condition in Definition \ref{def: differentiable mappings}, let $g: \mathbb R^n \to \mathbb R^k$.
	
	If $f$ is differentiable at $\mathbf p$ and $g$ is differentiable at $f(\mathbf p)$, then $g\circ f$ is differentiable at $\mathbf p$.
	
	\begin{proof}
		As $f$ is differentiable at $\mathbf p$, there exists a linear mapping $\phi: \mathbb R^m \to \mathbb R^n$ and a neighbourhood $N$ of $\mathbf p$ such that for any $\mathbf t\in \mathbb R^m$ with $\mathbf p + \mathbf t \in \mathbb R^m$,
		$$
		f(\mathbf p) + \phi(\mathbf t) = f(\mathbf p + \mathbf t) - \|\mathbf t\|_{\mathbb R^m} \alpha(\mathbf t).
		$$
		
		As $g$ is differentiable at $f(\mathbf p)$, there exists a linear mapping $\lambda: \mathbb R^m \to \mathbb R^n$ such that
		$$
		\lim_{\mathbf t \to \mathbf 0_{\mathbb R^m}} \frac{g(f(\mathbf p) + \| \mathbf t \|_{\mathbb R^m}\phi(\mathbf{\hat t})) - f(\mathbf p) - \lambda(\| \mathbf t \|_{\mathbb R^m}\phi(\mathbf{\hat t}))}{\Big\|\|\mathbf t\|_{\mathbb R^m} \phi(\mathbf{\hat t})\Big\|_{\mathbb R^n}} = \mathbf 0_{\mathbb R^k}.
		$$
		
		As $\phi$ is linear, we have
		$$
		\| \mathbf t \|_{\mathbb R^m} \phi(\mathbf{\hat t}) = \phi(\mathbf t).
		$$
		
		By scalar multiplication, we have
		$$
		\Big\|\|\mathbf t\|_{\mathbb R^m} \phi(\mathbf{\hat t})\Big\|_{\mathbb R^n} = \| \mathbf t \|_{\mathbb R^m} \| \phi (\mathbf{\hat t}) \|_{\mathbb R^n}.
		$$
		
		Now, we have
		$$
		\begin{aligned}
			& \ \lim_{\mathbf t \to \mathbf 0_{\mathbb R^m}} \frac{g(f(\mathbf p) + \phi(\mathbf t)) - g(f(\mathbf p)) - \lambda(\phi(\mathbf t))}{\| \mathbf t \|_{\mathbb R^m}} = \mathbf 0_{\mathbb R^k} \\
			\iff & \ \lim_{\mathbf t \to \mathbf 0_{\mathbb R^m}} \frac{g(f(\mathbf p + \mathbf t)) - g(f(\mathbf p)) - (\lambda \circ \phi) (\mathbf t)}{\| \mathbf t \|_{\mathbb R^m}} = \mathbf 0_{\mathbb R^k}.
		\end{aligned}
		$$
		As $\lambda$ and $\phi$ are both linear, $\lambda \circ \phi$ are also linear.
		
		By Definition \ref{def: differentiable mappings}, $g \circ f$ is differentiable at $\mathbf p$.
	\end{proof}
\end{theorem}


%------------------------------------------------
%================================================


\section{Directional Derivatives}
%================================================
%------------------------------------------------


\begin{observation}
	\label{obs: motivation of directional derivatives}
	Let $f: \mathbb R^m \to \mathbb R^n$, and let $g:\mathbb R \to \mathbb R^m$ be defined as
	$$
	g(t) := \mathbf p + t \mathbf u,
	$$
	where $\mathbf p , \mathbf u \in \mathbb R^m$ and $\mathbf u \ne \mathbf 0_{\mathbb R^m}$.
	
	Let $h = f \circ g$ and define $h': D_{h'} \subseteq \mathbb R \to \mathbb R^n$ as
	$$
	h'(t) :=\lim_{t \to t_0} \frac{h(t) - h(t_0)}{t - t_0},
	$$
	where for any $t \in D_{h'}$, the this limit exists in $\mathbb R^n$. Thus,
	$$
	h'(0) = \lim_{t \to 0} \frac{f(\mathbf p + t \mathbf u) - f(\mathbf p)}{t}
	$$
	describes the instantaneous rate of change of $f$ along the straight line $\{ \mathbf p + t\mathbf u : t \in \mathbb R \}$ with $\|\mathbf u \|_{\mathbb R^m}$ as the unit length. $h'(0)$ is so-called the $\mathbf u$-directional derivative of $f$ at $\mathbf p$ (See Definition \ref{def: directional derivatives}).
\end{observation}


\begin{definition}[Directional Derivatives]
	\label{def: directional derivatives}
	
	Let $f: \mathbb R^m \to \mathbb R^n$, and let $\mathbf u \in \mathbb R^{m} \setminus \{ \mathbf 0_{\mathbb R^m} \}$. The \textit{$\mathbf u$-derived function} of $f$, denoted $\nabla_{\mathbf u}f$ is a function $\nabla_{\mathbf u}f: D \subseteq \mathbb R^m \to \mathbb R^n$ defined as
	$$
	\nabla_{\mathbf u} f(\mathbf x) := \lim_{t \to 0} \frac{f(\mathbf x + t \mathbf u) - f(\mathbf x)}{t},
	$$
	where $D$ is the set of all $\mathbf x \in \mathbb R^m$ such that $\nabla_{\mathbf u}f(\mathbf x)$ exists in $\mathbb R^n$. Let $\mathbf p \in D$, then $\nabla_{\mathbf u} f(\mathbf p)$ is a \textit{$\mathbf u$-directional derivative} of $f$ at $\mathbf p$.
\end{definition}


\begin{note}
	\label{note: derivative}
	
	As $\mathbb R$ is an ordered field, there are only two direction in $\mathbb R$. Thus, for any $u \in \mathbb R \setminus \{0\}$, $u > 0$ or $u < 0$. If $u = 1$, then we write
	$$
	\frac{\mathrm df}{\mathrm d t} \text{ or } f' \text{ for } \nabla_u f,
	$$
	and simply call $f'$ the \textit{derived function} of $f$. If $f$ is differentiable at a point $p \in \mathbb R$, then $f'(p)$ is called the \textit{derivative} of $f$ at $p$.
\end{note}



\begin{theorem}
	\label{thm: directional derivatives: scalar multiplication}
	With the condition in Definition \ref{def: directional derivatives}, for any $s \in \mathbb R \setminus \{ 0 \}$,
	$$
	\nabla_{s\mathbf u}f(\mathbf p) = s\nabla_{\mathbf u} f(\mathbf p).
	$$
	
	\begin{proof}
		Let $\theta = ts^{-1}$, then, by Definition \ref{def: directional derivatives}, we have
		$$
		\begin{aligned}
			s\nabla_{\mathbf u} f(\mathbf p) &= s\lim_{t \to 0} \frac{f(\mathbf p + t\mathbf u) - f(\mathbf p)}{t} \\
			&= \lim_{t \to 0} \frac{f(\mathbf p + t\mathbf u) - f(\mathbf p)}{ts^{-1}} \\
			&= \lim_{t \to 0} \frac{f(\mathbf p + \theta (s \mathbf u)) - f(\mathbf p)}{\theta} \\
			&= \nabla_{s\mathbf u} f(\mathbf p).
		\end{aligned}
		$$
		
	\end{proof}
\end{theorem}


\begin{theorem}
	\label{thm: directional derivatives: relative continuous on a straight line}
	With the condition in Definition \ref{def: directional derivatives}, if $\nabla_{\mathbf u} f(\mathbf p)$ exists, then there exists an open subset $U \subseteq \mathbb R^m$ with $\mathbf p \in U$ such that $f$ is relative continuous on the line described by $\mathbf p + t\mathbf u$ for some $t \in \mathbb R$.
	
	\begin{proof}
		Let $U$ be an open subset of $\mathbb R^m$, and let $g: \mathbb R \to \mathbb R^m$ be defined as
		$$
		g(t) := \mathbf p + t\mathbf u.
		$$
		Then $f$ is relative continuous on the line defined by $\mathbf p + t \mathbf u$ for some $t \in \mathbb R$ iff $f \restriction_{g[\mathbb R]}$ is continuous on $U \cap g[\mathbb R]$.
		
		Let $h = f \circ g$, then
		$$
		\nabla_{\mathbf u} f(\mathbf p) = \lim_{t \to 0} \frac{h(t) - h(0)}{t} = \mathbf v \in \mathbb R^n.
		$$
		Then, there exists an $\alpha: \mathbb R \to \mathbb R^n$ with $\alpha(t) \to \mathbf 0_{\mathbb R^n}$ as $t \to 0$, such that there exists an open subset $I \subseteq \mathbb R$ with $0 \in I$, such that for any $t \in I$,
		$$
		h(t) = t\mathbf v + t\alpha(t) + h(0).
		$$
		Then we have
		$$
		\begin{aligned}
		& \ \lim_{t \to 0} h(t) = \lim_{t \to 0} [t\mathbf v + t\alpha(t) + h(0)] \\
		\iff & \ \lim_{t \to 0} h(t) = h(0).
		\end{aligned}
		$$
		Thus, $h$ is continuous at $0$.
		
		As it is easy to show $g$ is bijective, $g \circ g^{-1}$ is an identity mapping on $g[\mathbb R] \subseteq \mathbb R^m$. As composition of mappings is associative, we have
		$$
		\begin{aligned}
		h = f \circ g &\iff h \circ g^{-1} = f \circ g \circ g^{-1} \\
		&\iff h \circ g^{-1} = f \circ (g \circ g^{-1}) \\
		&\iff h \circ g^{-1} = f \restriction_{g[\mathbb R]}.
		\end{aligned}
		$$
		It is also easy to find that $g^{-1}$ is continuous everywhere, thus, as $h$ is continuous at $0$, $f \restriction_{g[\mathbb R]}$ is continuous on $U \cap g[\mathbb R]$. Thus, $f$ is relative continuous on the line defined by $\mathbf p + t\mathbf u$ for some $t \in \mathbb R$.
	\end{proof}
\end{theorem}


\begin{theorem}
	\label{thm: directional derivatives: differentiable implies continuous directioinal derivative}
	With the condition in Definition \ref{def: directional derivatives}, if $f$ is differentiable at $\mathbf p$, then, for any $\mathbf u \in \mathbb R^m$, $\nabla_{\mathbf u} f$ is continuous at $\mathbf p$.
	
	\begin{proof}
		As $f$ is continuous, it is easy to show that
		$$
		\lim_{t \to 0}\nabla_{\mathbf u} f(\mathbf p + t\mathbf u) = \nabla_{\mathbf u} f(\mathbf p) = \lim_{t \to 0} \frac{f(\mathbf p + t\mathbf u) - f(\mathbf p)}{t}.
		$$
	\end{proof}
\end{theorem}



%================================================
%------------------------------------------------


\section{Mean Value Theorem in Vector Valued Functions}
%================================================
%------------------------------------------------


\begin{lemma}
	\label{lm: non-zero derivative implies monotonic}
	Let $f: \mathbb R \to \mathbb R$, and let $a, b \in \mathbb R$ with $a < b$. Suppose $f$ is continuous on $[a,b]$ and differentiable on $(a,b)$, and $0 \notin f'[(a,b)]$.
	
	Then, $f$ is strictly monotonic on $[a,b]$.
	
	\begin{proof}
		As $f$ is differentiable on $(a,b)$, by Theorem \ref{thm: directional derivatives: differentiable implies continuous directioinal derivative}, $f'$ is continuous on $(a,b)$. This implies, if $0 \notin f'[(a,b)]$, then
		$$
		f'[(a,b)] \subseteq \mathbb R_{> 0} \text{ or } f'[(a,b)] \subseteq \mathbb R_{< 0}.
		$$
	
	
		Let $c \in (a,b)$. As $f$ is differentiable at $c$, for any 
		$$
		f'(c) = \lim_{t \to 0} \frac{f(c + t) - f(c)}{t}.
		$$
		
		Now, Consider $f'(c) > 0$. Then $f(c + t) - f(c) > 0$ as $t \to 0^+$, and $f(c + t) - f(c) < 0$ as $t \to 0^-$. That is, for any $d, e \in (a,b)$,
		$$
		e < c < d \implies f(e) < f(c) < f(d).
		$$
		
		As $f$ is continuous at $a$ and $b$, we have
		$$
		\lim_{e \to a} f(e) = f(a) < f(c) < f(b) = \lim_{d \to b} f(d).
		$$
		
		If $f'(c) < 0$, the proof is similar.
	\end{proof}
\end{lemma}



\begin{lemma}
	[Rolle's Theorem]
	\label{lm: rolle's theorem}
	
	Let $f: \mathbb R^m \to \mathbb R^n$. Let $\mathbf a, \mathbf b \in \mathbb R^m$ with $f(\mathbf a) = f(\mathbf b)$. Suppose $f$ is relative continuous on $\ell[\mathbf a, \mathbf b]$, and relative differentiable on $\ell(\mathbf a, \mathbf b)$. 
	
	Then, there exits $\mathbf c \in \ell(\mathbf a, \mathbf b)$ such that $\nabla_{\mathbf u} f(\mathbf c) = \mathbf 0_{\mathbb R^n}$, where $\mathbf u = \mathbf b - \mathbf a$.
	
	\begin{proof}
		First, consider $f = \langle f_i \rangle_{i = 1}^n$.
		
		Suppose for any $\mathbf c \in \ell(\mathbf a, \mathbf b)$, $\nabla_{\mathbf u} f(\mathbf c) \ne \mathbf 0_{\mathbb R^n}$, then there exists $i \in \{1, \ldots, n\}$ such that $\nabla_{\mathbf u}f_i(\mathbf c) \ne 0$.
		
		Let $g: \mathbb R \to \mathbb R^m$ be defined as
		$$
		g(t) = \mathbf b - t\mathbf a,
		$$
		and let $h_i = f_i \circ g$. Then, for any $t \in (0, 1)$, $h_i'(t) \ne 0$.
		
		As $f_i$ is differentiable on $g[(0,1)]$, and $g$ is differentiable on $(0,1)$, by Theorem \ref{thm: differentiable mappings: composed}, $h_i$ is differentiable on $(0,1)$. In this case, $0 \notin h_i'[(0,1)]$ implies $h_i$ is strictly monotonic (Lemma \ref{lm: non-zero derivative implies monotonic}). This implies
		$$
		h_i(0) = f_i(\mathbf a) \ne f_i(\mathbf b) = h_i(1).
		$$
		As $f(\mathbf a) = \langle f_i(\mathbf a) \rangle_{i = 1}^n$ and $f(\mathbf b) = \langle f_i(\mathbf b) \rangle_{i = 1}^n$, we have $f(\mathbf a) \ne f(\mathbf b)$. This contradicts the assumption that $f(\mathbf a) = f(\mathbf b)$.
		
		Thus, there has to be a $\mathbf c \in \ell(\mathbf a, \mathbf b)$ such that $\nabla_{\mathbf u} f_i(\mathbf c)$.
	\end{proof}
\end{lemma}


\begin{lemma}
	\label{lm: directional derivatives: mean value theorem: real valued function}
	
	Let $f: \mathbb R \to \mathbb R^n$. If $f$ is differentiable on open subset $(a,b)$, and continuous on closed interval $[a,b]$, then there exists a $c \in I$ such that
	$$
	f'(c) = \frac{f(b) - f(a)}{b - a}.
	$$
	
	\begin{proof}
		Let $\phi: \mathbb R \to \mathbb R^n$ be defined as
		$$
		\phi(t) := t \frac{f(b)-f(a)}{b-a}.
		$$
	
		Let $h: \mathbb R \to \mathbb R^n$ be defined as
		$$
		h(t) := f(t) - \phi(t).
		$$
		Then it is easy to find that
		$$
		h(a) = h(b).
		$$
		
		As $f$ and $\phi$ are differentiable on $(a,b)$, so is $h$. (Why?)
		
		As $f$ and $\phi$ are continuous on $[a,b]$, so is $h$. (Why?)
		
		Thus, by Lemma \ref{lm: rolle's theorem}, there exists a $c \in (a,b)$ such that we have
		$$
		\begin{aligned}
			& \ 0 = h'(c) = f'(c) - \frac{f(b) - f(a)}{b - a} \\
			\iff & \ f'(c) = \frac{f(b) - f(a)}{b - a}.
		\end{aligned}
		$$
		
	\end{proof}
\end{lemma}


\begin{theorem}[Mean Value Theorem on $\mathbb R^m \to \mathbb R^n$]
	\label{thm: directional derivatives: mean value theorem: vector valued function}\
	
	Let $f: \mathbb R^m \to \mathbb R^n$. Let $\mathbf p, \mathbf q \in \mathbb R^m$, for convenience, let $g: \mathbb R \to \mathbb R^m$ be defined as
	$$
	g(t) := \mathbf p + t(\mathbf q - \mathbf p).
	$$
	
	If $f \restriction_{g[\mathbb R]}$ is continuous on $g[(0,1)]$, and differentiable on $g[[0,1]]$, then
	$$
	\| f(\mathbf q) - f(\mathbf p) \|_{\mathbb R^n} \le \sup_{\mathbf x \in g[(a,b)]} \| \nabla_{\mathbf u} f(\mathbf x) \|_{\mathbb R^n}.
	$$
	
	\begin{proof}Let $h = f \circ g$. As $f$ is continuous on $g[(0,1)]$ and $g$ is continuous everywhere, $h$ is continuous on $(0,1)$. By Theorem \ref{thm: differentiable mappings: composed}, as $f$ is differentiable on $g[[0,1]]$ and $g$ is differentiable on $[0,1]$, then, by Theorem \ref{thm: differentiable mappings: composed}, $h$ is differentiable on $[0,1]$.
	
		Let $h': D \subseteq \mathbb R \to \mathbb R^n$ be defined as
		$$
		h'(t) := \lim_{t \to 0} \frac{h(c + t) - h(t)}{t},
		$$
		where $D$ is the set of all points in $\mathbb R$ such that the limit exists in $\mathbb R^n$.
		
		By Lemma \ref{lm: directional derivatives: mean value theorem: real valued function}, there exists a $c \in (0,1)$ such that
		$$
		h'(c) = \frac{h(1) - h(0)}{1 - 0}.
		$$
		
		Now, we have
		$$
		\begin{aligned}
		h'(c) &= \lim_{t \to 0} \frac{h(c + t) - h(c)}{t} \\
		&= \lim_{t \to 0} \frac{f(g(c + t)) - f(c)}{t} \\
		&= \left.\lim_{t \to 0} \frac{f(\mathbf p + c\mathbf u + t\mathbf u) - f(\mathbf p + c\mathbf u)}{t}\right|_{\mathbf u = \mathbf q - \mathbf p} \\
		&= \left. \lim_{t \to 0} \frac{f(\mathbf c + t\mathbf u) - f(\mathbf c)}{t} \right|_{\mathbf c = \mathbf p + c \mathbf u} \\
		&= \nabla_{\mathbf u}f(\mathbf c).
		\end{aligned}	
		$$
		
		Thus, there exists a $\mathbf c \in g[(0,1)]$ such that
		$$
		\nabla_{\mathbf u} f(\mathbf c) = h(1) - h(0) =f(\mathbf q) - f(\mathbf p).
		$$
		This implies that there exists some $\mathbf x \in g[(0,1)]$ such that
		$$
		\| \nabla_{\mathbf u}f(\mathbf x) \| \ge \| \nabla_{\mathbf u}f(\mathbf c) \|.
		$$
		Thus,
		$$
		\| f(\mathbf q) - f(\mathbf p) \| \le \sup_{\mathbf x \in g[(0,1)]}\| \nabla_{\mathbf u} f(\mathbf x) \|.
		$$
	\end{proof}
\end{theorem}


%------------------------------------------------
%================================================



\section{Partial Derivatives and Jacobian Matrices}
%================================================
%------------------------------------------------


\begin{definition}
	[Partial Derivatives]
	\label{def: partial derivatives}
	
	Let $f: \mathbb R^m \to \mathbb R^n: \mathbf x \mapsto \mathbf y$. 
	
	The \textit{$i$-th partial derived function} of $f$, denoted $\frac{\partial f}{\partial x_i}$, is the $\mathbf{\hat e}_i$-directional derived function of $f$, where $\mathbf{\hat e}_i$ denotes the $i$-th basis of $\mathbb R^m$. If $\frac{\partial f}{\partial x_i}(\mathbf p)$ exists in $\mathbb R^n$ for a $\mathbf p \in \mathbb R^m$, then this value is called \textit{$i$-th partial derivative} of $f$ at $\mathbf p$.
\end{definition}


\begin{definition}
	[Jacobian Matrices]
	\label{def: jacobian matrices}
	
	With the condition in Definition \ref{def: partial derivatives}, The \textit{Jacobian Matrix} of $f$ is a function $\nabla f: D \subseteq \mathbb R^m \to \mathbb R^{n \times m}$ be defined as
	$$
	\nabla f :=
	\left[
	\begin{matrix}
		\displaystyle \frac{\partial f}{\partial x_1} & \cdots & \displaystyle \frac{\partial f}{\partial x_m}
	\end{matrix}
	\right],
	$$
	where $D$ is the set of all $\mathbf x \in \mathbb R^m$ such that $\frac{\partial f}{\partial x_i}$ exists in $\mathbb R^m$ for any $i \in \{1, \ldots, n\}$.
\end{definition}


\begin{note}
	If $f$ is considered as an $1 \times n$ matrix, then $\nabla$ can be considered as a function from $\mathbb F$ to $\mathbb S$ where the domain $\mathbb F$ is a normed space contains all functions from $\mathbb R^m$ to $\mathbb R^n$, and the codomain $\mathbb S$ is another normed space contains all $n \times m$ matrices. It is defined as
	$$
	\nabla f :=
	\left[
	\begin{matrix}
	\displaystyle \frac{\partial}{\partial x_1} \\
	\vdots \\
	\displaystyle \frac{\partial}{\partial x_m}
	\end{matrix}
	\right]
	\left[
	\begin{matrix}
	f_1 & \cdots & f_n
	\end{matrix}
	\right].
	$$
	In this sense, it is easy to prove that $\nabla$ is linear by matrices multiplication. Also, the $\mathbf u$-directional derived function of $f$ can be considered as
	$$
	\nabla_\mathbf u f =
	\left[
	\begin{matrix}
	u_1 & \cdots & u_m
	\end{matrix}
	\right]
	\left[
	\begin{matrix}
	\displaystyle \frac{\partial}{\partial x_1} \\
	\vdots \\
	\displaystyle \frac{\partial}{\partial x_m}
	\end{matrix}
	\right]
	\left[
	\begin{matrix}
	f_1 & \cdots & f_n
	\end{matrix}
	\right]
	= \mathbf u^\top\nabla f.
	$$
	
	For convenience, we denote
	$$
	(\mathbf u^\top \nabla)^k f = \mathbf u^\top \nabla\Big( \cdots \big(\mathbf u^\top \nabla(\mathbf u^\top \nabla f) \big) \cdots \Big) \quad \text{$k$ times}.
	$$
	In the case $f: \mathbb R^m \to \mathbb R$, as $f (\mathbf p) \in \mathbb R$ for any $\mathbf p \in \mathbb R^m$, $\nabla f(\mathbf p)$ can be considered as an $m$ dimensional vector ($m \times 1$), which is called \textit{gradient} of $f$ at $\mathbf p$. In this case,
	$$
	\mathbf u \cdot \nabla f(\mathbf p) = \nabla f(\mathbf p) \cdot \mathbf u = \nabla_{\mathbf u}f(\mathbf p).
	$$
	where $\cdot$ denotes the inner product.
\end{note}


\begin{theorem}
	[Chain Rule]
	\label{thm: chain rule}
	
	Let $f: \mathbb R^m \to \mathbb R^n: \mathbf x \to \mathbf y$, and let $g: \mathbb R^n \to \mathbb R^k: \mathbf t \to \mathbf x$. For convenience, let $h = g \circ f$.
	
	If $f$ is differentiable at a point $\mathbf p \in \mathbb R^m$ and $g$ is differentiable at $f(\mathbf p) = \mathbf q \in \mathbb R^n$, then
	$$
	\nabla h(\mathbf p) = [\nabla g(f(\mathbf p))]^\top \nabla f(\mathbf p) \in \mathbb R^{k \times m}.
	$$
	
	\begin{proof}
		By Theorem \ref{thm: differentiable mappings: composed}, $h$ is differentiable at $\mathbf p$, and there exists $\phi: \mathbb R^m \to \mathbb R^n$ and $\lambda: \mathbb R^n \to \mathbb R^k$
	\end{proof}
\end{theorem}




%------------------------------------------------
%================================================



---

---

---

---

---

---

---

---

---

---

---

---





%::::::::::::::::::::::::::::::::::::::::::::::::
%================================================










































%