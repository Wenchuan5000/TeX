\chapter{Limit and Continuity}
%================================================
%::::::::::::::::::::::::::::::::::::::::::::::::


\section{Limit}
%================================================
%------------------------------------------------


\begin{theorem}
	\label{thm: limit: exists alpha}
\end{theorem}


%------------------------------------------------
%================================================


%::::::::::::::::::::::::::::::::::::::::::::::::
%================================================



\chapter{Differentiation}
%================================================
%::::::::::::::::::::::::::::::::::::::::::::::::


\section{Differentiable Mapping}
%================================================
%------------------------------------------------


\begin{definition}[Differentiable Mappings]
	\label{def: differentiable mappings}
	\
	
	Let $f:\mathbb R^m \to \mathbb R^n$, and let $\mathbf p \in \mathbb R^m$.
	
	$f$ is said to be \textit{differentiable} at $\mathbf p$ iff for any $\mathbf t \in \mathbb R^m \setminus \{ \mathbf 0_{\mathbb R^m} \}$,
	$$
	\lim_{\mathbf t \to \mathbf 0_{\mathbb R^m}} \frac{f(\mathbf p + \mathbf t) - f(\mathbf p) - \phi(\mathbf t)}{\| \mathbf t \|_{\mathbb R^m}} = \mathbf 0_{\mathbb R^n}.
	$$
\end{definition}


\begin{lemma}[Alternative Definition of Differentiable mapping]
	\label{lm: differentiable mapping: alt-def}
	\
	
	With the condition of Definition \ref{def: differentiable mappings}, $f$ is continuous at $\mathbf p$, iff
	$$
	f(\mathbf p + \mathbf t) = f(\mathbf p) + \phi(\mathbf t) + o(\mathbf t) \quad \text{as $\mathbf t \to \mathbf 0_{\mathbb R^m}$}.
	$$
	
	\begin{proof}
		By Theorem \ref{thm: limit: exists alpha}, the limit in Definition \ref{def: differentiable mappings} is zero, iff there exists a neighbourhood $N$ of $\mathbf p$, and an $\alpha: \mathbb R^m \to \mathbb R^n$ with $\alpha(\mathbf t) \to \mathbf 0_{\mathbb R^n}$ at $\mathbf t \to \mathbf 0_{\mathbb R^m}$ such that for any $\mathbf t \in N \setminus \{ \mathbf p \} - \{ \mathbf p \}$,
		$$
		\begin{aligned}
			\frac{f(\mathbf p + \mathbf t) - f(\mathbf p) - \phi(\mathbf t)}{\| \mathbf t \|_{\mathbb R^m}} = \alpha(\mathbf t) \\
		\end{aligned}
		$$
		
		Then,
		$$
		\begin{aligned}
			& \lim_{\mathbf t \to \mathbf 0_{\mathbb R^m}} \frac{\| \mathbf t \|_{\mathbb R^m} \alpha (\mathbf t)}{\| \mathbf t \|_{\mathbb R^m}} = \mathbf 0_{\mathbb R^m} \\
			\iff & \| \mathbf t \|_{\mathbb R^m} \alpha(\mathbf t) = o(\mathbf t) \quad \text{as $\mathbf t \to \mathbf 0_{\mathbb R^m}$}
		\end{aligned}
		$$
		
		Thus, we have
		$$
		f(\mathbf p + \mathbf t) = f(\mathbf p) + \phi(\mathbf t) + o(\mathbf t) \quad \text{as $\mathbf t \to \mathbf 0_{\mathbb R^m}$}.
		$$
	\end{proof}
\end{lemma}


\begin{theorem}
	\label{thm: differentiable mappings: uniqueness of phi}
	In Definition \ref{def: differentiable mappings}, $\phi$ is unique.
	
	\begin{proof}
		The equation in Definition \ref{def: differentiable mappings} can be considered as: there exists a neighbourhood $N$ of $\mathbf p$ and an $\alpha: \mathbb R^m \to \mathbb R^n$ with $\alpha(\mathbf t) \to \mathbf 0_{\mathbb R^n}$ as $\mathbf t \to \mathbf 0_{\mathbb R^m}$, such that for any $\mathbf t \in \mathbb R^m \setminus \{ \mathbf 0_{\mathbb R^m} \}$ with $\mathbf p + \mathbf t \in N$,
		$$
		f(\mathbf p + \mathbf t) - f(\mathbf p) - \phi(\mathbf t) = \| \mathbf t \|_{\mathbb R^m} \alpha(\mathbf t).
		$$
		
		Suppose there exists another linear mapping $\lambda: \mathbb R^m \to \mathbb R^n$, such that there exists a neighbourhood $N'$ of $\mathbf p$ and a $\beta: \mathbb R^m \to \mathbb R^n$ with $\beta(\mathbf t) \to \mathbf 0_{\mathbb R^n}$ as $\mathbf t \to \mathbf 0_{\mathbb R^m}$, such that for any $\mathbf t \in \mathbb R^m \setminus \{ \mathbf 0_{\mathbb R^m}\}$ with $\mathbf p + \mathbf t \in N'$,
		$$
		f(\mathbf p + \mathbf t) - f(\mathbf p) - \lambda(\mathbf t) = \| \mathbf t \|_{\mathbb R^m} \beta(\mathbf t).
		$$
		
		Then, we have
		$$
		\begin{aligned}
			& \ \lim_{\mathbf t \to \mathbf 0_{\mathbb R^m}} \frac{\phi(\mathbf t) - \lambda(\mathbf t)}{\| \mathbf t \|_{\mathbb R^m}} = \lim_{\mathbf t \to \mathbf 0_{\mathbb R^m}} \left( \beta(\mathbf t) - \alpha(\mathbf t) \right) \\
			\iff & \ \phi (\mathbf{\hat t}) - \lambda (\mathbf{\hat t}) = \mathbf 0_{\mathbb R^m}.
		\end{aligned}
		$$
		
		As $\mathbf t$ is arbitrarily taken from $N \cap N' - \mathbf p$, and there must be an open subset $U \subseteq N \cap N'$, thus,
		$$
		\left\{ \mathbf{\hat t} = \frac{\mathbf t}{\| \mathbf t \|_{\mathbb R^m}} : \mathbf t \in N \cap N' - \mathbf p \right\}
		$$
		contains all possible direction in $\mathbb R^m$. Thus as $\phi$ and $\lambda$ are linear, $\phi(\mathbf{\hat t}) = \lambda(\mathbf{\hat t})$ iff $\phi = \lambda$.
	\end{proof}
\end{theorem}


\begin{theorem}
	With the condition in Definition \ref{def: differentiable mappings}, if $f$ is differentiable at $\mathbf p$, then $f$ is continuous at $\mathbf p$.
	
	\begin{proof}
		As $f$ is differentiable at $\mathbf p$,
	\end{proof}
\end{theorem}



%------------------------------------------------
%================================================



---

---

---

---

---

---

---

---

---

---

---

---





%::::::::::::::::::::::::::::::::::::::::::::::::
%================================================










































%