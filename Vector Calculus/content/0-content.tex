
\chapter{Vector Spaces}
%================================================
%::::::::::::::::::::::::::::::::::::::::::::::::


\section{Linear Combinations}
%================================================
%------------------------------------------------


\begin{definition}
	\label{def: span and linear combination}
	Let $\langle\mathbf v_i\rangle_{i = 1}^n$ be a sequence such that for any $i \in \{1, \ldots, n\}$, $\mathbf v_i \in \mathbb R^n$.
	
	The \textit{linear span of $\langle \mathbf v_i\rangle$}, denoted $\mathrm{span}\langle \mathbf v_i\rangle$ is a subset of $\mathbb R^n$ defined as
	$$
	\mathrm{span}\langle \mathbf v_i\rangle := \left\{ \mathbf a \cdot \langle \mathbf v_i \rangle : \mathbf a \in \mathbb R^n \right\}.
	$$
	
	An element $\mathbf u \in \mathbb R^n$ is a \textit{linear combination of $\langle \mathbf v_i \rangle$} iff
	$$
	\mathbf u \in \mathrm{span} \langle \mathbf v_i \rangle.
	$$
\end{definition}


\begin{definition}
	\label{def: linear dependency}
	With the conditions above, for any $i,j \in \{1, \ldots, n\}$, $\mathbf v_i$ and $\mathbf v_j$ are said to be \textit{linearly dependent} iff there exists $t \in \mathbb R$, such that
	$$
	\mathbf v_i = t \mathbf v_j.
	$$
	
	$\mathbf v_i$ and $\mathbf v_j$ are \textit{linearly independent} iff they are not linearly dependent.
\end{definition}


\begin{note}
	By definition of inner product, that is
	$$
	\mathbf u = \sum_{i = 1}^n a_i \mathbf v_i.
	$$
\end{note}



\begin{note}
	Let $\langle \mathbf{\hat e}_i \rangle_{i = 1}^n$ be a sequence, and for any $i \in \{1, \ldots, n\}$,
	$$
	\mathbf{\hat e}_i := \langle \underset{1}{0}, \ldots, \underset{i}{1}, \ldots, \underset{n}{0} \rangle.
	$$
	
	Then, for any $\mathbf u \in \mathbb R^n$, the linear combination form of $\mathbf u$ is
	$$
	\mathbf u = \sum_{i = 1}^n u_i \mathbf{\hat e}_i.
	$$
\end{note}



%------------------------------------------------
%================================================



\section{Line and Plane}
%================================================
%------------------------------------------------


\begin{definition}
	\label{def: line}
	Let $\mathbf a, \mathbf b \in \mathbb R^n$, where $\mathbf a \ne \mathbf b$. Let $L_{\bf ab}: \mathbb R \to \mathbb R^n$ be a mapping defined as
	$$
	L_{\bf ab}(t) := \mathbf a + t(\mathbf b - \mathbf a).
	$$
	
	The \textit{line $\overline{\bf ab}$ through $\mathbf a$ and $\mathbf b$} is defined as the image of $\mathbb R$ under $L$; i.e.,
	$$
	\overline{\bf ab} := L_{\bf ab}[\mathbb R].
	$$
	
	With $\mathbf a$ and $\mathbf b$ as \textit{end points}, we define
	\begin{enumerate}[(i)]
		\item $L_{\bf ab}[[0,1]]$ as \textit{closed segment},
		\item $L_{\bf ab}[(0,1)]$ as \textit{open segment},
		\item $L_{\bf ab}[(0,1]]$ as \textit{half-open segment},
		\item $L_{\bf ab}[[0,1]]$ as \textit{half-closed segment}.
	\end{enumerate}
\end{definition}


\begin{definition}
	\label{def: plane}
	Let $\mathbf a \in \mathbb R^n$ and $\mathbf u \in \mathbb R^n \setminus \{\mathbf 0\}$.
	
	The \textit{plane through $\mathbf a$ and orthogonal to $\mathbf u$} is defined as
	$$
	P:= \left\{ \mathbf x \in \mathbb R^n : \mathbf u \cdot (\mathbf x - \mathbf a) = \mathbf 0 \right\}.
	$$
\end{definition}



%------------------------------------------------
%================================================



%::::::::::::::::::::::::::::::::::::::::::::::::
%================================================




































%