
\chapter{Particle Motion}
%================================================
%::::::::::::::::::::::::::::::::::::::::::::::::



\section{Particle Motion}
\label{sec: particle motion}
%================================================
%------------------------------------------------


Assume that we have a particle moving in space $\mathbb R^n$, as the motion of the particle is described by $\mathbf r(t)$ with respect to time $t$.

The \textit{displacement} $\Delta \mathbf r$ between $t_1$ and $t_0$  ($t_0 < t_1$) is defined to be the position change over the period $(t_0, t_1)$:
\begin{equation}
	\label{eq: displacement}
	\Delta \mathbf r = \mathbf r(t_1) - \mathbf r(t_0) \quad \mathrm {m}.
\end{equation}

The \textit{average velocity} $\mathbf{\bar v}$ is defined to be the average rate of position change during the period:
\begin{equation}
	\label{eq: ave velocity}
	\mathbf{\bar v} = \frac{\Delta \mathbf r}{\Delta t} \quad \mathrm{m / sec},
\end{equation}
where $\Delta t = t_1 - t_0$.

The \textit{velocity}, or \textit{instantaneous velocity}, $\mathbf v(t)$ at a certain time $t$ is defined to be the rate of change during a very small period between $t$ and $t + \Delta t$ ($|\Delta t| > 0$). That is, the limit of $\mathbf{\bar v}$ over neighbourhood of $t$ as $\Delta t \to 0$:
\begin{equation}
	\label{eq: velocity}
	\mathbf v(t) = \lim_{\Delta t \to 0} \frac{\mathbf r(t + \Delta t) - \mathbf r(t)}{\Delta t} = \frac{\mathrm d \mathbf r}{\mathrm d t}(t) \quad \mathrm{m/sec}.
\end{equation}

The \textit{distance traveled} $s$ of the particle from time $t_0$ to $t_1$ is defined to be the total variation of $\mathbf r(t)$ over $[t_0, t_1]$; as (\ref{eq: velocity}) is given, we have
\begin{equation}
	\label{eq: distance traveled}
	s = \int_{t_0}^{t_1} \| \mathrm d\mathbf r(t) \| = \int_{t_0}^{t_1} \| \mathbf v(t) \| \mathrm d(t) \quad \mathrm{m}.
\end{equation}

The acceleration $\mathbf a(t)$ at a certain time $t$ is defined to be the rate of velocity change over a very small interval between $t$ and $t + \Delta t$:
\begin{equation}
	\label{eq: acceleration}
	\mathbf a(t) = \lim_{\Delta t \to 0} \frac{\mathbf v(t + \Delta t) - \mathbf v(t)}{\Delta t} = \frac{\mathrm d \mathbf v(t)}{\mathrm d t} = \frac{\mathrm d^2 \mathbf r(t)}{\mathrm d t^2} \quad \mathrm{m/sec^2}
\end{equation}


%------------------------------------------------
%================================================


\section{Components of Velocity}

ssss


\section{Components of Acceleration}
%================================================
%------------------------------------------------

Following the condition in Section \ref{sec: particle motion}.

The \textit{tangential acceleration} $\mathbf a_\mathrm{T}(t)$ at a certain time $t$ is defined to be the projection of $\mathbf a(t)$ on any tangent vector of the motion curve at $t$. Thus,
\begin{equation}
	\label{eq: tangential acceleration}
	\mathbf a_\mathrm{T}(t) = \mathbf a(t) \cdot \mathbf{\hat v}(t) \cdot \mathbf{\hat v}(t) \quad \mathrm{m/sec^2}.
\end{equation}

Then \textit{centripetal acceleration} $\mathbf a_\mathrm{C}(t)$ at $t$ is defined to be the projection of $\mathbf a(t)$ at any right vector of the motion curve at $t$. As the right vector is orthogonal to the tangent vector, hence it is orthogonal to $\mathbf a(t)$. By the sum of vectors, we have
\begin{equation}
	\label{eq: centripetal acceleration}
	\mathbf a_{\mathrm C}(t) = \mathbf a(t) - \mathbf a_\mathrm{T}(t) \quad \mathrm{m / sec^2}.
\end{equation}

As any right vector at $t$ can be defined by the.........


%------------------------------------------------
%================================================























%::::::::::::::::::::::::::::::::::::::::::::::::
%================================================
