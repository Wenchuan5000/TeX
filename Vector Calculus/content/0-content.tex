


\chapter{Limit and Continuity}
%================================================
%::::::::::::::::::::::::::::::::::::::::::::::::


In the sections below, the capitals with blackboard bold font, such as $\mathbb A$, denote the normed vector spaces.


\section{O Notations}
%================================================
%------------------------------------------------


\begin{definition}
	\label{def: little-o}
	Let $f: \mathbb X \to \mathbb Y: \mathbf x \mapsto f(\mathbf x)$ and $g: \mathbb X \to \mathbb S: \mathbf x \mapsto g(\mathbf x)$.
	
	$f$ is a \textit{little-o of $g$ as $\mathbf x \to \mathbf p$}, denoted
	$$
	f(\mathbf x) = o(g(\mathbf x)) \quad \text{as $\mathbf x \to \mathbf p$},
	$$
	iff for any $\varepsilon \in \mathbb R_{> 0}$, there exists a neighbourhood $N$ of $\mathbf p$, such that for any $\mathbf x \in N$, $\| f(\mathbf x) \|_{\mathbb Y} \le \varepsilon \| g (\mathbf x)\|_{\mathbb S}$; equivalently, that is,
	$$
	\lim_{\mathbf x \to \mathbf p} \frac{\|f(\mathbf x)\|_{\mathbb Y}}{\| g(\mathbf x) \|_{\mathbb S}} = 0 \text{ or, equivalently, } \lim_{\mathbf x \to \mathbf p} \frac{f(\mathbf x)}{\| g(\mathbf x) \|_{\mathbb S}} = \mathbf 0_{\mathbb Y}
	$$
\end{definition}


\begin{lemma}
	\label{lm: little-o: zero limit}
	With the condition in Definition \ref{def: little-o}, suppose
	$$
	\lim_{\mathbf x \to \mathbf p} \| g(\mathbf x) \| \in \mathbb R,
	$$
	
	Then
	$$
	f(\mathbf x) = o(g(\mathbf x)) \quad \text{as $\mathbf x \to \mathbf p$},
	$$
	implies
	$$
	\lim_{\mathbf x \to \mathbf p} f(\mathbf x) = \mathbf 0_{\mathbb Y}.
	$$
	
	\begin{proof}
		Aiming for a contradiction, suppose there exists $\mathbf r \in \mathbb Y \setminus \{ \mathbf 0_{\mathbb Y} \}$, such that
		$$
		\lim_{\mathbf x \to \mathbf p} f(\mathbf x) = \mathbf r,
		$$
		then, we have
		$$
		\begin{aligned}
			\lim_{\mathbf x \to \mathbf p} \frac{\| f(\mathbf x) \|_{\mathbb Y}}{\| g(\mathbf x) \|_{\mathbb S}} = \lim_{\mathbf x \to \mathbf p} \frac{\| \mathbf r \|_{\mathbb Y}}{\| g(\mathbf x) \|_{\mathbb S}} > 0.
		\end{aligned}
		$$
		
		This contradicts the assumption.
	\end{proof}
\end{lemma}


\begin{lemma}
	\label{lm: little-o: negative}
	With the condition in Definition \ref{def: little-o}, $f$ is a little-o of $g$ iff $-f$ is a little-o of $g$.
	
	\begin{proof}
		$$
		\lim_{\mathbf x \to \mathbf p} \frac{\| - f(\mathbf x) \|_{\mathbb Y}}{\| g(\mathbf x) \|_{\mathbb S}} = \lim_{\mathbf x \to \mathbf p} \frac{\| f(\mathbf x) \|_{\mathbb Y}}{\| g(\mathbf x) \|_{\mathbb S}} = 0.
		$$
	\end{proof}
\end{lemma}


\begin{lemma}
	\label{lm: little-o: finite sum}
	Let $f_1, f_2: \mathbb X \to \mathbb Y: \mathbf x \mapsto f_1(\mathbf x), f_2(\mathbf x)$, and let $g: \mathbb X \to \mathbb S: \mathbf x \mapsto g(\mathbf x)$.
	
	If $f_1$ and $f_2$ are both little-o of $g$ as $\mathbf x \to \mathbf p$, i.e.,
	$$
	f_1(\mathbf x) = o_1(g(\mathbf x)) \text{ and } f_2(\mathbf x) = o_2(g(\mathbf x)) \text{ as $\mathbf x \to \mathbf p$},
	$$
	then $f_1 + f_2$ is also a little-o of $g$ as $\mathbf x \to \mathbf p$, i.e.,
	$$
	f_1(\mathbf x) + f_2(\mathbf x) = o_3(g(\mathbf x))
	$$
	
	\begin{proof}
		By triangle inequality,  we have
		$$
		\begin{aligned}
			\lim_{\mathbf x \to \mathbf p}\frac{\| f_1(\mathbf x) + f_2(\mathbf x) \|_{\mathbb Y}}{\| g(\mathbf x) \|_{\mathbb S} } &\le \lim_{\mathbf x \to \mathbf p} \frac{\| f_1(\mathbf x) \|_{\mathbb Y} + \| f_2(\mathbf x) \|_{\mathbb Y}}{\| g(\mathbf x) \|_{\mathbb S}} \\
			&\le 2 \lim_{\mathbf x \to \mathbf p} \frac{\max\{ \| f_1(\mathbf x) \|_{\mathbb Y} + \| f_2(\mathbf x) \|_{\mathbb Y} \}}{\| g(\mathbf x) \|_{\mathbb S}} \\
			&= 0.
		\end{aligned}
		$$
		
		By Definition \ref{def: little-o}, $f_1 + f_2$ is a little-o of $g$ as $\mathbf x \to \mathbf p$.
	\end{proof}
\end{lemma}



\begin{note}
	In Lemma \ref{lm: little-o: finite sum}, consider $A$ be the set of all mappings being little-o of $g$ as $\mathbf x \to \mathbf p$, then Lemma \ref{lm: little-o: finite sum} tells that $A$ is finitely additive. That is, for any finite $B \subseteq A$,
	$$
	\sum_{o \in B} o(g(\mathbf p)) \in A.
	$$
\end{note}




\chapter{Differentiation}
%================================================
%::::::::::::::::::::::::::::::::::::::::::::::::



\section{Differentiable Mappings}
%================================================
%------------------------------------------------


\begin{definition}
	\label{def: differentiable mappings}
	
	Let $f: \mathbb X \to \mathbb Y$.
	
	$f$ is said to be \textit{differentiable at $\mathbf p \in \mathbb X$} iff there exists a linear mapping $\phi: \mathbb X \to \mathbb Y$ such that for any $\mathbf t \in \mathbb X$,
	$$
	f(\mathbf p + \mathbf t) = f(\mathbf p) + \phi(\mathbf t) + o(\mathbf t) \quad \text{as $\mathbf t \to \mathbf 0_{\mathbb X}$}.
	$$
\end{definition}


\begin{lemma}
	\label{lm: uniqueness of phi}
	
	In Definition \ref{def: differentiable mappings}, the linear mapping $\phi$ is unique.
	
	\begin{proof}
		Suppose there is another linear mapping $\lambda: \mathbb X \to \mathbb Y$, such that for any $\mathbf t \in \mathbb X$,
		$$
		f(\mathbf p + \mathbf t) = f(\mathbf p) + \lambda(\mathbf t) + o_\lambda (\mathbf t) \quad \text{as $\mathbf t \to \mathbf 0_{\mathbb X}$},
		$$
		then we have
		$$
		\begin{aligned}
			& \phi(\mathbf{\hat t}) - \lambda(\mathbf{\hat t}) = \lim_{\mathbf t \to \mathbf 0_{\mathbb X}}\frac{\phi(\mathbf t) - \lambda(\mathbf t)}{\| \mathbf t \|_{\mathbb X}} = \lim_{\mathbf t \to \mathbf 0_{\mathbb X}} \frac{o(\mathbf t) - o_\lambda(\mathbf t)}{\| \mathbf t\|_{\mathbb X}}
		\end{aligned}
		$$
		
		By Lemma \ref{lm: little-o: negative}, $-o_{\lambda}(\mathbf t)$ is also a little-o of $\mathbf t$ as $\mathbf t \to \mathbf 0_{\mathbb X}$, thus, by Lemma \ref{lm: little-o: finite sum}, $o(\mathbf t) - o_\lambda(\mathbf t)$ is a little-o of $\mathbf t$ as $\mathbf t \to \mathbf 0_{\mathbb X}$. By Definition \ref{def: little-o},
		$$
		\phi(\mathbf{\hat t}) - \lambda(\mathbf{\hat t}) = \mathbf 0_{\mathbb X}.
		$$
		
		As $\mathbf t$ is arbitrarily given, $\mathbf{\hat t}$ defines all possible directions in $\mathbb X$. Thus,
		$$
		\phi = \lambda.
		$$
	\end{proof}
\end{lemma}



\begin{lemma}
	With the condition in Definition \ref{def: differentiable mappings}, $f$ is differentiable at $\mathbf p \in \mathbb X$ iff there exists a linear mapping $\phi: \mathbb X \to \mathbb Y$ such that for any $\mathbf t \in \mathbb X$,
	\begin{equation}
		\tag{i}
		\lim_{\mathbf t \to \mathbf 0_{\mathbb X}} \frac{\| f(\mathbf p + \mathbf t) - f(\mathbf p) - \phi(\mathbf t) \|_{\mathbb Y}}{\| \mathbf t \|_{\mathbb X}} = 0.
	\end{equation}
	Equivalently, that is,
	\begin{equation}
		\tag{i'}
		\lim_{\mathbf t \to \mathbf 0_{\mathbb X}} \frac{ f(\mathbf p + \mathbf t) - f(\mathbf p) - \phi(\mathbf t) }{\| \mathbf t \|_{\mathbb X}} = \mathbf 0_{\mathbb Y}.
	\end{equation}
	
	\begin{proof}
		This can be proved from both sides. Consider the equations in this proposition and in Definition \ref{def: differentiable mappings}. We observe that the equation in Definition \ref{def: differentiable mappings} holds iff
		$$
		\begin{aligned}
			& \frac{f(\mathbf p + \mathbf t) - f(\mathbf p) - \phi(\mathbf t)}{\| \mathbf t \|_{\mathbb X}} = \frac{o(\mathbf t)}{\| \mathbf t \|_{\mathbb X}}\\
			\iff & \lim_{\mathbf t \to \mathbf 0_{\mathbb X}} \frac{f(\mathbf p + \mathbf t) - f(\mathbf p) - \phi(\mathbf t)}{\| \mathbf t \|_{\mathbb X}} = \mathbf 0_{\mathbb X} & \text{((i') is proved)} \\
			\iff & \lim_{\mathbf t \to \mathbf 0_{\mathbb X}} \frac{\| f(\mathbf p + \mathbf t) - f(\mathbf p) - \phi(\mathbf t)\|_{\mathbb Y}}{\| \mathbf t \|_{\mathbb X}} = 0. & \text{((i) is proved)}
		\end{aligned}
		$$
	\end{proof}
\end{lemma}



\begin{lemma}
	With the condition in Definition \ref{def: differentiable mappings}, if $f$ is differentiable at $\mathbf p$, then $f$ is continuous at $\mathbf p$.
	
	\begin{proof}
		As $f$ is differentiable at $\mathbf p$, there exists a linear mapping $\phi: \mathbb X \to \mathbb Y$, such that for any $\mathbf t \in \mathbb X$,
		$$
		f(\mathbf p + \mathbf t) = f(\mathbf p) + \phi(\mathbf t) + o(\mathbf t) \quad \text{as $\mathbf t \to \mathbf 0_{\mathbb R^m}$}.
		$$
		
		As
		$$
		\lim_{\mathbf t \to \mathbf 0_{\mathbb X}} \phi(\mathbf t) = \mathbf 0_{\mathbb Y}
		$$
		and, by Lemma \ref{lm: little-o: zero limit},
		$$
		\lim_{\mathbf t \to \mathbf 0_{\mathbb X}}o(\mathbf t) = \mathbf 0_{\mathbb Y},
		$$
		
		we have
		$$
		\lim_{\mathbf t \to \mathbf 0_{\mathbb X}} f(\mathbf p + \mathbf t) = f(\mathbf p),
		$$
		which implies that $f$ is continuous at $\mathbf p$.
	\end{proof}
\end{lemma}



\begin{lemma}
	\label{lm: differentiable: composed}
	
	Let $f: \mathbb X \to \mathbb Y$ and let $g: \mathbb Y \to \mathbb S$. If $f$ is differentiable at a point $\mathbf p \in \mathbb X$, and $g$ is differentiable at $f(\mathbf x)$, then $g \circ f$ is differentiable at $\mathbf p$.
	
	\begin{proof}
		As $g$ is differentiable at $f(\mathbf p)$, there exists $\lambda: \mathbb Y \to \mathbb S$ such that for any $\mathbf s \in \mathbb Y$ with $f(\mathbf p) + \mathbf s \in f[\mathbb X]$,
		$$
		g(f(\mathbf p) + \mathbf s) = g(f(\mathbf p)) + \lambda(\mathbf s) + o(\mathbf s) \quad \text{as $\mathbf s \to \mathbf 0_{\mathbb Y}$}.
		$$
		
		As $f$ is differentiable at $\mathbf p$, $f$ is continuous at $\mathbf p$, thus, there exists $\mathbf t \in \mathbb X$, such that $\displaystyle\lim_{\mathbf t \to \mathbf 0_{\mathbb X}} f(\mathbf p + \mathbf t) = f(\mathbf p) + \mathbf s$. Since $f$ is differentiable at $\mathbf p$, there exists a linear mapping $\phi: \mathbb X \to \mathbb Y$, such that
		$$
		f(\mathbf p + \mathbf t) = f(\mathbf p) + \phi(\mathbf t) + o_1 (\mathbf t) \quad \text{as $\mathbf t \to \mathbf 0_{\mathbb X}$}.
		$$
		
		Then we have
		$$
		g(f(\mathbf p + \mathbf t)) = g(f(\mathbf t)) + \lambda(\Delta f) + o(\Delta f) \quad \text{as $\mathbf t \to \mathbf 0_{\mathbb X}$},
		$$
		where
		$$
		\Delta f = f(\mathbf p + \mathbf t) - f(\mathbf p) = \phi(\mathbf t) + o_1(\mathbf t) \quad \text{as $\mathbf t \to \mathbf 0_{\mathbb X}$}.
		$$
		
		First, find $\lambda(\Delta f)$. As $\lambda$ is linear,
		$$
		\begin{aligned}
			\lambda(\Delta f) &= \lambda(\phi(\mathbf t) + o_1(\mathbf t))
			= \lambda(\phi(\mathbf t)) + \lambda(o_1 (\mathbf t)) \quad \text{as $\mathbf t \to \mathbf 0_{\mathbb X}$}.
		\end{aligned}
		$$
		As $\lambda$ is linear, $\lambda \circ \phi$ is also linear, and $\lambda(o_1(\mathbf t))$ is a little-o of $\mathbf t$, i.e., $o_2(\mathbf t) = \lambda(o_1(\mathbf t))$ as $\mathbf t \to \mathbf 0_{\mathbb X}$, for
		$$
		\begin{aligned}
			\lim_{\mathbf t \to \mathbf 0_{\mathbb X}} \frac{\lambda(o_1(\mathbf t))}{\| \mathbf t \|_{\mathbb X}} &= \lim_{\mathbf t \to \mathbf 0_{\mathbb X}} \lambda\left( \frac{ o_1 (\mathbf t) }{ \| \mathbf t \|_{\mathbb X} } \right) \\
			&= \lambda(\mathbf 0_{\mathbb Y}) \\
			&= \mathbf 0_{\mathbb S}.
		\end{aligned}
		$$
		
		Let $\gamma = \lambda \circ \phi$ for convenience.
		
		Then, find $o(\Delta f)$.
		$$
		\begin{aligned}
			\mathbf 0_{\mathbb S} &= \lim_{\mathbf t \to \mathbf 0_{\mathbb X}} \frac{o(\phi(\mathbf t) + o_1(\mathbf t))}{\| \phi(\mathbf t) + o_1(\mathbf t) \|_{\mathbb Y}}
			&\text{(Definition \ref{def: little-o})} \\
			&= \lim_{\mathbf t \to \mathbf 0_{\mathbb X}} \frac{ o(\phi(\mathbf t) + o_1(\mathbf t)) \| \mathbf t \|_{\mathbb X}^{-1} }{ \| \phi(\mathbf t) + o_1(\mathbf t) \|_{\mathbb Y} \| \mathbf t \|_{\mathbb X}^{-1} } \\
			&= \lim_{\mathbf t \to \mathbf 0_{\mathbb X}} \frac{o(\phi(\mathbf t) + o_1(\mathbf t))}{\| \mathbf t \|_{\mathbb X} \| \phi(\mathbf{\hat t})\| _{\mathbb Y}}
			&\text{(as $\phi$ is linear)} \\
			&= \| \phi(\mathbf{\hat t}) \|_{\mathbb Y}^{-1} \lim_{\mathbf t \to \mathbf 0_{\mathbb X}} \frac{o(\phi(\mathbf t) - o_1(\mathbf t))}{\| \mathbf t \|_{\mathbb X}}.
		\end{aligned}
		$$
		
		Thus, $o(\phi(\mathbf t) - o_1(\mathbf t)) = o_3(\mathbf t)$ as $\mathbf t \to \mathbf 0_{\mathbb X}$.
		
		Now, we have
		$$
		g(f(\mathbf p + \mathbf t)) = g(f(\mathbf t)) + \gamma(\mathbf t) + o_2(\mathbf t) + o_3(\mathbf t) \quad \text{as $\mathbf t \to \mathbf 0_{\mathbb X}$}.
		$$
		
		By Lemma \ref{lm: little-o: finite sum},
		$$
		o_2(\mathbf t) + o_3(\mathbf t) = o_4(\mathbf t) \quad \text{as $\mathbf t \to \mathbf 0_{\mathbb X}$}.
		$$
		
		Finally, we have
		$$
		g(f(\mathbf p + \mathbf t)) = g(f(\mathbf t)) + \gamma(\mathbf t) + o_4(\mathbf t) \quad \text{as $\mathbf t \to \mathbf 0_{\mathbb X}$},
		$$
		which implies $g \circ f$ is differentiable at $\mathbf p$.
	\end{proof}
\end{lemma}


\section{Directional Derivatives}
%================================================
%------------------------------------------------


\begin{definition}
	\label{def: directional derivatives}
	
	Let $f: \mathbb X \to \mathbb Y$, and let $\mathbf u \in \mathbb X \setminus \{ \mathbf 0_{\mathbb X} \}$.
	
	The \textit{$\mathbf u$-directional derived mapping}
\end{definition}



---

---

---

---

---

---

---

---

---

---

---

---
















































%