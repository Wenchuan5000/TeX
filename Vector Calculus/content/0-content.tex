
\chapter{Directional and Partial Derivatives}
%================================================
%::::::::::::::::::::::::::::::::::::::::::::::::



\section{Directional Derivatives}
%================================================
%------------------------------------------------


\begin{definition}
	\label{def: directional derivatives}
	Let $U$ be an open set of $\mathbb R^n$, and let $f: U \to \mathbb R^m$. Let $\vec u \in U \setminus \{\vec 0\}$ and $\vec x \in U$.
	
	Then, the \textit{$\vec u$-directional derivative of $f$ at $\vec x$} is defined as
	$$
	\nabla_{\vec u} f(\vec x) := \lim_{t \to 0} \frac{f(\vec x + t \vec u) - f(\vec x)}{t},
	$$
	if the limit exists in $\mathbb R^m$.
\end{definition}



\begin{note}
	By Definition \ref{def: directional derivatives}, if we consider $\nabla_{\vec u} f$ as a function, the mapping between elements is
	$$
	\vec x \in U \mapsto \vec y \in \mathbb R^m.
	$$
	
	Thus, $\nabla_{\vec u} f: U \to \mathbb R^m$, and it can considered as the \textit{$\vec u$-directional derived function of $f$} defined as
	$$
	\nabla_{\vec u} f(\vec x) :=
	\begin{cases}
		\displaystyle \lim_{t \to 0} \frac{f(\vec x + t \vec u) - f(\vec x)}{t} & \text{, if the limit exists in $\mathbb R^m$}; \\
		0 & \text{, otherwise}.
	\end{cases}
	$$
\end{note}



\begin{lemma}
	\label{lemma: directional derivatives: dg/dt}
	With the condition above, let $g: \mathbb R^1 \to \mathbb R^m$ be defined as
	$$
	g(t) := f(\vec x + t\vec u),
	$$
	then,
	$$
	\nabla f_{\vec u} = \frac{\mathrm d g}{\mathrm d t}.
	$$
	
	\begin{proof}
		$$
		\begin{aligned}
			\nabla_{\vec u} f (\vec x) &= \lim_{t \to 0} \frac{f(\vec x + t \vec u) - f(\vec x)}{t} &\text{(by Definition \ref{def: directional derivatives})} \\
			&= \lim_{t \to 0} \frac{g(t) - g(0)}{t} &\text{(by assumption)} \\
			&= \frac{\mathrm d g(t)}{\mathrm dt}.
		\end{aligned}
		$$
		
		Generalized it, we have
		$$
		\nabla_{\vec u} f = \frac{\mathrm d g}{\mathrm dt}.
		$$
	\end{proof}
\end{lemma}



\begin{note}
	As $\vec x + t\vec u$ defines a subset of $\mathbb R^n$, $g(t)$ can be considered as a function defined on a new one-dimensional axis with $\vec x$ as the new origin $0'$ and $\vec u$ as the new unit $1'$.

	Let
	$$
	L : =  \{ \vec x + t \vec u \in \mathbb R^n : t \in \mathbb R \},
	$$
	then we have
	$$
	g[\mathbb R] = f[L] \in \mathbb R^m.
	$$
\end{note}



\begin{lemma}
	\label{lemma: directional derivatives: su}
	With the condition above, let $s \in \mathbb R^1 \setminus \{0\}$, we have
	$$
	s\nabla_{\vec u} f = \nabla_{s\vec u} f.
	$$
	
	\begin{proof}
		Let $g: \mathbb R^1 \to \mathbb R^m$ be defined as
		$$
		g(t) := f(\vec x + t \vec u),
		$$
		then we have
		$$
		\begin{aligned}
			s \nabla_{\vec u} f(\vec x) &= s \frac{\mathrm d g(t)}{\mathrm d t} &\text{(by lemma \ref{lemma: directional derivatives: dg/dt})} \\
			&= \frac{\mathrm d g(t)}{\mathrm dt} \cdot \frac{\mathrm d st}{\mathrm d t} \\
			&= \frac{\mathrm d g(st)}{\mathrm d(t)} &\text{(by chain rule)} \\
			&= \lim_{t \to 0} \frac{g(st) - g(t)}{t} \\
			&= \lim_{t \to 0} \frac{f(\vec x + t s \vec u) - f(\vec x)}{t} &\text{(by assumption)} \\
			&= \nabla_{s \vec u} f(\vec x). & \text{(by Definition \ref{def: directional derivatives})}
		\end{aligned}
		$$
		
		Generalized it, we have
		$$
		s\nabla_{\vec u} f = \nabla_{s\vec u} f.
		$$
	\end{proof}
\end{lemma}



%------------------------------------------------
%================================================


\section{Partial Derivatives}
%================================================
%------------------------------------------------



\begin{definition}
	\label{def: partial derivatives}

	Let $U$ be an open set of $\mathbb R^n$, and let $f: U \to \mathbb R^m$. Let $\vec x \in U$.
	
	The \textit{$i$-th partial derivative of $f$ at $\vec x$} is defined to be the $\hat e_i$-directional derivative of $f$ at $\vec x$.
\end{definition}



\begin{note}
	Explicitly, by Definition \ref{def: directional derivatives}, that is,
	$$
	\nabla_{i} f(\vec x) = \lim_{t \to 0}\frac{f(\vec x + t \hat e_i) - f(\vec x)}{t},
	$$
	if the limit exists in $\mathbb R^m$. Here, we write $\nabla_i$ for $\nabla_{\hat e_i}$ for convince.
	
	As $\hat e_i$ is the $i$-th basis of $\mathbb R^n$, we can let $\delta = t \hat e_i \in \mathbb R_i$, then we have
	$$
	\nabla_i f(\vec x) = \lim_{\delta \to 0 \in \mathbb R_i} \frac{f(x_1, \ldots, x_i + \delta, \ldots , x_n) - f(x_1, \ldots, x_i, \ldots, x_n)}{\delta}.
	$$
	
	Now, let $g: \mathbb R_i \to \mathbb R^m$ be defined as
	$$
	g(x_i) := f(x_1, \ldots, x_i, \ldots, x_n),
	$$
	then we have
	$$
	\nabla_i f(\vec x) = \lim_{\delta \to 0} \frac{g(x_i + \delta) - g(x_i)}{\delta} = \frac{\mathrm d g(x_i)}{\mathrm d x_i}.
	$$
	
	In classical notation, we write
	$$
	\frac{\partial f(\vec x)}{\partial x_i} \text{ for } \frac{\mathrm d g(x_i)}{\mathrm d x_i} \text{, and } \frac{\partial f}{\partial x_i} \text{ for } \frac{\mathrm d g}{\mathrm d x_i}.
	$$
\end{note}



%------------------------------------------------
%================================================



\section{Gradient}
%================================================
%------------------------------------------------



\begin{definition}
	\label{def: gradient}
	Let $U$ be an open set of $\mathbb R^n$, and let $f: U \to \mathbb R^m$. Let $\vec x \in U$.
	
	The \textit{gradient of $f$ at $\vec x$} is defined as
	$$
	\nabla f(\vec x) := (\nabla_1 f(\vec x), \ldots, \nabla_n f(\vec x)).
	$$
\end{definition}


\begin{note}
	$$
	\nabla f: U \to \mathbb R^m \times \cdots \times \mathbb R^m \text{ ($n$ times)}
	$$
	(Note the $m$ and $n$ here.)
\end{note}



\begin{lemma}
	Following the conditions in Definition \ref{def: gradient}, we have
	$$
	\nabla f = \frac{\partial f}{\partial \vec x},
	$$
	where, in classical notation, $\frac{\partial f}{\partial \vec x} = \frac{\mathrm d f}{\mathrm d \vec x}$.
	
	\begin{proof}
		For any $i \in \{1, \ldots, n\}$, let $g_i : U_i \to \mathbb R^m$ be defined as
		$$
		g_i(x_i) := f(x_1, \ldots, x_i, \ldots, x_n).
		$$
	
		Then we have
		$$
		\begin{aligned}
			\nabla f(\vec x) &= \left( \frac{\mathrm d g_1 (x_1)}{\mathrm d x_1}
			\ldots , \frac{\mathrm d g_n(x_n)}{\mathrm d x_n} \right) \\
			&= \left( \mathrm d g_1 (x_1), \ldots, \mathrm d g_n (x_n) \right) \cdot \left(\frac{1}{\mathrm d x_1}, \ldots, \frac{1}{\mathrm d x_n} \right) \\
			&= \frac{\mathrm d f(\vec x)}{\mathrm d \vec x}
		\end{aligned}
		$$
		
		(Missing Details.)
	\end{proof}
\end{lemma}



\begin{definition}
	$$
	\mathrm d f := \nabla f \cdot \mathrm d\vec x.
	$$
\end{definition}



\begin{lemma}
	Following the condition in Definition \ref{def: gradient}, let $g: T \to U$, where $T$ is an open subset of $\mathbb R$. Then we have
	$$
	\frac{\mathrm d f \circ g}{\mathrm d t} = \nabla f \cdot \frac{\mathrm d g}{\mathrm dt}.
	$$

	\begin{proof}
		$$
		\begin{aligned}
			\frac{\mathrm d f(r(t))}{\mathrm dt} &= \frac{\mathrm d f(r(t))}{\mathrm dr(t)} \cdot \frac{\mathrm dr(t)}{\mathrm d t} \\
			&= \nabla f(\vec x) \cdot \frac{\mathrm dr(t)}{\mathrm dt}.
		\end{aligned}
		$$
		
		Generalize it, we have
		$$
		\frac{\mathrm d f \circ g}{\mathrm d t} = \nabla f \cdot \frac{\mathrm d g}{\mathrm dt}.
		$$
	\end{proof}
\end{lemma}



\begin{lemma}
	$$
	\nabla_{\vec u} f = \nabla f \cdot \vec u.
	$$
	
	\begin{proof}
		By Definition \ref{def: directional derivatives},
		$$
		\nabla_{\vec u} f(\vec x) = \lim_{t \to 0} \frac{f(\vec x + t\vec u) - f(\vec x)}{t}.
		$$
		
		Let
		$$
		g(t) := \vec x + t\vec u,
		$$
		then we have
		$$
		\begin{aligned}
			\nabla_{\vec u} f(\vec x) &= \lim_{t \to 0} \frac{f(g(t)) - f(g(0))}{t} \\
			&= \left. \frac{\mathrm d f(g(t))}{\mathrm dt} \right|_{t = 0} \\
			&= \nabla f(\vec x) \cdot \vec u.
		\end{aligned}
		$$
	\end{proof}
\end{lemma}



%------------------------------------------------
%================================================














%::::::::::::::::::::::::::::::::::::::::::::::::
%================================================
