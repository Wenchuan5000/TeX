
\chapter{Particle Motion}
%================================================
%::::::::::::::::::::::::::::::::::::::::::::::::



\section{Particle Motion}
\label{sec: particle motion}
%================================================
%------------------------------------------------


Assume that we have a particle moving in space $\mathbb R^n$, as the motion of the particle is described by $\mathbf r(t)$ with respect to time $t$.

The \textit{displacement} $\Delta \mathbf r$ between $t_1$ and $t_0$  ($t_0 < t_1$) is defined to be the position change over the period $(t_0, t_1)$:
\begin{equation}
	\label{eq: displacement}
	\Delta \mathbf r = \mathbf r(t_1) - \mathbf r(t_0) \quad \mathrm {m}.
\end{equation}

The \textit{average velocity} $\mathbf{\bar v}$ is defined to be the average rate of position change during the period:
\begin{equation}
	\label{eq: ave velocity}
	\mathbf{\bar v} = \frac{\Delta \mathbf r}{\Delta t} \quad \mathrm{m / sec},
\end{equation}
where $\Delta t = t_1 - t_0$.

The \textit{velocity}, or \textit{instantaneous velocity}, $\mathbf v(t)$ at a certain time $t$ is defined to be the rate of change during a very small period between $t$ and $t + \Delta t$ ($|\Delta t| > 0$). That is, the limit of $\mathbf{\bar v}$ over neighbourhood of $t$ as $\Delta t \to 0$:
\begin{equation}
	\label{eq: velocity}
	\mathbf v(t) = \lim_{\Delta t \to 0} \frac{\mathbf r(t + \Delta t) - \mathbf r(t)}{\Delta t} = \frac{\mathrm d \mathbf r}{\mathrm d t}(t) \quad \mathrm{m/sec}.
\end{equation}

The \textit{distance traveled} $s$ of the particle from time $t_0$ to $t_1$ is defined to be the total variation of $\mathbf r(t)$ over $[t_0, t_1]$; as (\ref{eq: velocity}) is given, we have
\begin{equation}
	\label{eq: distance traveled}
	s = \int_{t_0}^{t_1} | \mathrm d\mathbf r(t) | = \int_{t_0}^{t_1} | \mathbf v(t) | \mathrm d(t) \quad \mathrm{m}.
\end{equation}

The acceleration $\mathbf a(t)$ at a certain time $t$ is defined to be the rate of velocity change over a very small interval between $t$ and $t + \Delta t$:
\begin{equation}
	\label{eq: acceleration}
	\mathbf a(t) = \lim_{\Delta t \to 0} \frac{\mathbf v(t + \Delta t) - \mathbf v(t)}{\Delta t} = \frac{\mathrm d \mathbf v(t)}{\mathrm d t} = \frac{\mathrm d^2 \mathbf r(t)}{\mathrm d t^2} \quad \mathrm{m/sec^2}
\end{equation}


%------------------------------------------------
%================================================


\section{Angular Velocity}
%================================================
%------------------------------------------------


Following the condition in Section \ref{sec: particle motion}.

The \textit{direction of velocity} of the particle at time $t$ is given by the unit vector of velocity $\mathbf v(t)$:
\begin{equation}
	\label{eq: direction of velocity}
	\mathbf{\hat v}(t) = \frac{\mathbf v(t)}{| \mathbf v(t) |}.
\end{equation}

As the unit $\mathrm{m/sec}$ is canceled by the quotient between $\mathbf v(t)$ and $| \mathbf v(t) |$, $\mathbf{\hat v}(t)$ only remains the direction of $\mathbf v(t)$, with length $1$.

The \textit{angular velocity} $\boldsymbol{\omega}(t)$ at time $t$ of the particle is defined to be the angular change of the particle during a very small time interval. That is,
\begin{equation}
	\label{eq: angular velocity}
	\boldsymbol{\omega}(t) = \lim_{\Delta t \to 0} \frac{\mathbf{\hat v}(t + \Delta t) - \mathbf{\hat v}(t)}{\Delta t} = \frac{\mathrm d \mathbf{\hat v}(t)}{\mathrm dt} \quad \mathrm{sec^{-1}}.
\end{equation}


\begin{problem}
	Explain circular motion and uniform circular motion.
\end{problem}


%------------------------------------------------
%================================================


\section{Curvature}
%================================================
%------------------------------------------------
Following the condition in Section \ref{sec: particle motion}.

The \textit{curvature} $\kappa(t)$ of the particle at time $t$ is defined to be
\begin{equation}
	\label{eq: curvature}
	\kappa(t) = \left| \frac{\mathrm d \mathbf{\hat v}(t)}{\mathrm d s(t)} \right| \quad \mathrm{sec^{-1}},
\end{equation}
where $\mathrm d s(t)$ is the differential of distance traveled at $t$.

By (\ref{eq: distance traveled}), we have
$$
\begin{aligned}
	\mathrm d s(t) &= \frac{\mathrm d s(t)}{\mathrm dt} \mathrm dt \\
	&= \frac{\mathrm d}{\mathrm d t} \int_{t_0}^t | \mathbf v(\tilde t) | \mathrm d \tilde t \cdot \mathrm dt \\
	&= | \mathbf v(t) | \mathrm dt.
\end{aligned}
$$

As $\mathrm d \mathbf{\hat v}(t) = \boldsymbol{\omega}(t) \mathrm d t$, we have
\begin{equation}
	\label{eq: curvature 2}
	\tag{\ref{eq: curvature}'}
	\kappa(t) = \frac{| \boldsymbol{\omega}(t) |}{| \mathbf v(t) |} \quad \mathrm{sec^{-1}}.
\end{equation}


\begin{problem}
	Prove that
	\begin{equation}
		\label{eq: curvature 3}
		\tag{\ref{eq: curvature}''}
		\kappa(t) = \frac{|\mathbf v(t) \times \mathbf a(t)|}{|\mathbf v(t)|^3} \quad \mathrm{sec^{-1}}.
	\end{equation}
\end{problem}


%------------------------------------------------
%================================================




\section{Components of Acceleration}
%================================================
%------------------------------------------------

Following the condition in Section \ref{sec: particle motion}.

The \textit{tangential acceleration} $\mathbf a_\mathrm{T}(t)$ at a certain time $t$ is defined to be the projection of $\mathbf a(t)$ on any tangent vector of the motion curve at $t$. Thus,
\begin{equation}
	\label{eq: tangential acceleration}
	\mathbf a_\mathrm{T}(t) = \mathbf a(t) \cdot \mathbf{\hat v}(t) \cdot \mathbf{\hat v}(t) \quad \mathrm{m/sec^2}.
\end{equation}

Then \textit{centripetal acceleration} $\mathbf a_\mathrm{C}(t)$ at $t$ is defined to be the projection of $\mathbf a(t)$ at any normal vector of the motion curve at $t$. That is,
\begin{equation}
	\mathbf a_{\mathrm C}(t) = \mathbf a(t) \cdot \mathbf{\hat n}(t) \cdot \mathbf{\hat n}(t) \quad \mathrm{m / sec^2},
\end{equation}
where $\mathbf{\hat n}(t)$ is the unit normal vector given by
$$
\mathbf{\hat n}(t) = \frac{\mathrm d \mathbf{\hat v}(t)}{\mathrm d t} \cdot \left|\frac{\mathrm d \mathbf{\hat v}(t)}{\mathrm d t} \right|^{-1}.
$$

As the normal vector is orthogonal to the tangent vector, hence it is orthogonal to $\mathbf a(t)$. By the sum of vectors, we have
\begin{equation}
	\label{eq: centripetal acceleration}
	\mathbf a_{\mathrm C}(t) = \mathbf a(t) - \mathbf a_\mathrm{T}(t) \quad \mathrm{m / sec^2}.
\end{equation}




%------------------------------------------------
%================================================



%::::::::::::::::::::::::::::::::::::::::::::::::
%================================================




\chapter{Force}
%================================================
%::::::::::::::::::::::::::::::::::::::::::::::::



\section{Newton's Laws of Motion}
%================================================
%------------------------------------------------


\paragraph{First Law.}

\begin{quote}
	An object at rest remains at rest, and an object in motion remains in motion at constant speed and in a straight line unless acted on by an unbalanced force.
\end{quote}


\paragraph{Second Law}

\begin{quote}
	The acceleration of an object depends on the mass of the object and the amount of force applied.
\end{quote}


\paragraph{Third Law}

\begin{quote}
	Whenever one object exerts a force on another object, the second object exerts an equal and opposite on the first.
\end{quote}


%------------------------------------------------
%================================================




\section{Explanation for the Laws}
%================================================
%------------------------------------------------


% todo: missing conditions


The Second Law can be expressed as the equation
\begin{equation}
	\label{eq: Newton's Second Law}
	\mathbf F(t) = m \mathbf a(t) \quad \mathrm{N},
\end{equation}
where $m$ is the mass of the object and the unit Newton is defined as
$$
\rm N = kg \cdot m/sec.
$$

In the First Law, the force is $\mathbf 0$, thus it is the case as $\mathbf a(t) = \mathbf 0$. In this case,
$$
\mathbf v(t) = \left. \int \mathbf a(t) \mathrm d t \right|_{\mathbf a(t) = \mathbf 0} = \mathbf v_0 \quad \mathrm{m/s},
$$
where $\mathbf v_0$ is the initial velocity. Thus, the First Law can be considered as
\begin{equation}
	\label{eq: Newton's First Law}
	\mathbf a(t) = \mathbf 0 \iff \mathbf F(t) = \mathbf 0.
\end{equation}

Denote $\mathbf F_{\mathbf a}(t)$ for the force exerted on the object, then the object also exerts a force
\begin{equation}
	\mathbf F_{- \mathbf a}(t) = - \mathbf F_{\mathbf a}(t) \quad \mathrm{N}.
\end{equation}


%------------------------------------------------
%================================================



\section{Momentum}
%================================================
%------------------------------------------------


\begin{definition}
	\label{def: momentum}
	Given particle motion $\mathbf r(t)$, the momentum of the particle at a certain time $t$ is defined to be the integral of fors
	$$
	\mathbf p(t) := \int \mathbf F(t) \mathrm dt \quad \mathrm{kg \cdot m/s},
	$$
	where $\mathbf F(t)$ is the force at $t$, $m$ is the mass of, and $\mathbf v_0$ is the initial velocity of the particle.
\end{definition}


\begin{proposition}
	\label{prop: momentum: zero}
	Let $\mathbf a(t)$ describe the acceleration of a particle motion with respect to time $t$, and let $\mathbf p(t)$ describe the momentum of the particle motion. $\mathbf p(t) = \mathbf 0$ iff
	$$
	\mathbf 0 = \int_{0}^{t} \mathbf a(t) \mathrm d t + m \mathbf v_0 \quad \mathrm{m/s},
	$$
	where $\mathbf v_0$ is the initial velocity of the particle.
	
	\begin{proof}
		By Definition \ref{def: momentum},
		$$
		\mathbf 0 = m \left. \int \mathbf a(t_*) \mathrm d t_* \right|_{t_* = t} = m \int_{0}^{t} \mathbf a(t_*) \mathrm d t_* + m \mathbf v_0 \quad \mathrm{m/s}.
		$$
		That is, the velocity $\mathbf v(t) = \mathbf 0$.
	\end{proof}
\end{proposition}


\begin{proposition}
	...

	\begin{proof}
		$$
		\begin{aligned}
			m \left( \mathbf v(t) + \left. \int \mathbf a_2 \mathrm d t_* \right|_{t_* = t} \right) = \mathbf 0 &\iff \mathbf v(t) + \left. \int \mathbf a_2 \mathrm d t_* \right|_{t_* = t} = \mathbf 0 \\
			&\iff \int \mathbf a(t) \mathrm d t + \int \mathbf a_2 \mathrm d t = \mathbf 0 \\
			&\iff 
		\end{aligned}
		$$
	\end{proof}
\end{proposition}



%------------------------------------------------
%================================================



%::::::::::::::::::::::::::::::::::::::::::::::::
%================================================




































%