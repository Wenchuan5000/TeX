%================================================
%::::::::::::::::::::::::::::::::::::::::::::::::
\chapter{Topological Spaces}


%================================================
\section{Basic Definitions}


%------------------------------------------------
\begin{definition}
	\label{def: topological space}

	Let $X$ be any set, and let $\mathcal T \subseteq 2^X$.
	
	Then $\mathcal T$ is called a \textbf{topology on $X$} iff it satisfies the \textbf{open set axioms}. That is,
	\begin{enumerate}[O1.]
		\item $\emptyset, X \in \mathcal T$
		\item For any $\mathcal U \subseteq \mathcal T$, $\bigcup \mathcal U \in \mathcal T$; i.e., $\mathcal T$ is closed under arbitrary union.
		\item For any finite $\mathcal V \subseteq \mathcal T$, $\bigcap \mathcal V \in \mathcal T$; i.e., $\mathcal T$ is closed under finite intersection.
	\end{enumerate}
	
	The ordered pair $\mathbb X = (X, \mathcal T)$ is called a \textbf{topological space}.
	
	A subset $U \subseteq X$ is said to be \textbf{open} iff it is an element of $\mathcal T$.
\end{definition}
%------------------------------------------------


%------------------------------------------------
%\begin{note}
	Rigorously, $\emptyset \in \mathcal T$ is not necessary for O1 in Definition \ref{def: topological space}, because it can be proved in a simple way.
	
	As empty set is an element of any set, it is also an element of $\mathcal T$. Therefore,
	$$
	\emptyset = \bigcup \emptyset \in \mathcal T.
	$$
%\end{note}
%------------------------------------------------










%================================================


%::::::::::::::::::::::::::::::::::::::::::::::::
%================================================
