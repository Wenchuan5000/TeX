%================================
\section{Continuous Maps}
%================================

Let $(X, \mathcal T_X)$ and $(Y, \mathcal T_Y)$ be topological spaces, and let $f$ be any map from $(X, \mathcal T_X)$ to $(Y, \mathcal T_Y)$.


%--------------------------------
\begin{definition}
	$f$ is said to be \textit{continuous on $U_x \in \mathcal T_X \cap \mathcal P(X)$} if and only if for any $U_y \in \mathcal T_Y$, the preimage $f^{-1}[U_y] \cap U_x \in \mathcal T_X$.
\end{definition}
%--------------------------------


%--------------------------------
\begin{definition}
	$f$ is said to be \textit{continuous} if and only if for any $U \in \mathcal T_Y$, the preimage $f^{-1}[U] \in \mathcal T_X$.
\end{definition}
%--------------------------------


%--------------------------------
\begin{example}
	Even if for any $U \in \mathcal T_X$ such that $f[X] \in \mathcal T_Y$, $f$ is not necessarily continuous. For example, let $X = \mathbb R$ and let $\mathcal T_X$ be induced by Euclidean metric, let $Y = \mathbb R$ and $\mathcal T_Y = \mathcal P(Y)$, and let $f$ defined by
	$$
	f(x) = x.
	$$
	For any $U \subseteq X$ (even unnecessary to be open), $f[U] \in \mathcal T$. But $f$ is not continuous, for there exists $U \in \mathcal T_Y$ such that $f^{-1}[U] \notin \mathcal T_X$; for example, for any $U \subseteq \mathbb Q$, or for any $\{y\} \in \mathcal T_Y$, etc.
	
	This example also shows that being bijective dose not implies continuity.  
\end{example}
%--------------------------------


%--------------------------------
\begin{example}
	If $\mathcal T_X = \mathcal P(X)$, then any map from $\mathcal T_X$ to $\mathcal T_Y$ is continuous.
\end{example}
%--------------------------------


%--------------------------------
\begin{definition}
	$f$ is said to be \textit{continuous at a point} $x \in X$ if and only if there is an open neighbourhood $U_x$ of $x$ such that for any open neighbourhood $U_y$ of $y$, the preimage $f^{-1}[U_y] \cap U_x \in \mathcal T_X$.
\end{definition}
%--------------------------------


%--------------------------------
\begin{proposition}
	$f$ is continuous at $x \in X$ if and only if for any neighbourhood $N_{y}$ of $f(x)$, there exists neighbourhood $N_x$ of $x$ such that $f(N_x) \subseteq N_y$.
	
	\begin{proof}
		Let $U_y$ be an open neighbourhood of $f(x)$.
		
		If $f$ is continuous at $x$, then there must be an open neighbourhood $U_x$ of $x$ such that $f^{-1}[U_y] \cap U_x \in \mathcal T_X$. $U_y \ni f(x)$, so $f^{-1}[U_y] \cap U_x \ni x$. As $f$ is continuous, $U_y \in \mathcal T_Y$ implies $f^{-1}[U_y] \in \mathcal T$, then, as $U_x \in \mathcal T$, $f^{-1}[U_y] \cap U_x$ is an open neighbourhood of $x$. And we have
		$$
		f[f^{-1}[U_y] \cap U_x] \subseteq U_y.
		$$
		
		On the other hand, if there is an open neighbourhood $N_x$ of $x$ such that $f[N_x] \subseteq U_y$, then we have
		$$
		\begin{aligned}
			f^{-1}[f[N_x]] \subseteq f^{-1}[U_y] &\implies N_x \subseteq f^{-1}[U_y] \\
			&\implies N_x = f^{-1}[U_y] \cap N_x.
		\end{aligned}
		$$
		As $N_x \in \mathcal T$, it is also true that $f^{-1}[U_y] \cap N_x \in \mathcal T$. Thus $f$ is continuous on $x$.
	\end{proof}
\end{proposition}
%--------------------------------







































%