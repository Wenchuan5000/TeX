%================================
\section{Metrizability}
%================================


%--------------------------------
\begin{definition}
	[metric spaces]
	\label{def: metric spaces}
	Let $X$ be any set. A \textit{metric} $\rho$ on $X$ is a function $\rho: X \times X \to \mathbb R$ satisfying the following conditions: for all $x,y,z \in X$
	\begin{enumerate}[(i)]
		\item $\rho(x,y) \ge 0$, and $\rho(x,y) = 0$ iff $x = y$;
		\item $\rho(x,y) = \rho(y,z)$;
		\item $\rho(x,z) + \rho(z,y) \ge \rho(x,y)$.
	\end{enumerate}
\end{definition}
%--------------------------------


%--------------------------------
\begin{definition}
	[balls]
	\label{def: balls}
	Let $(X, \rho)$ be a metric space, let $x \in X$, and let $\varepsilon \in \mathbb R_{> 0}$. The \textit{open $\varepsilon$-ball about $x$} or just \textit{$\varepsilon$-ball about $x$} is defined to be
	$$
	B(x, \varepsilon) = \left\{ y \in X : \rho(x,y) < \varepsilon \right\}.
	$$
	The \textit{closed $\varepsilon$-ball about $x$} is defined to be
	$$
	\overline B (x, \varepsilon) = \{ y \in X : \rho(x,y) \le \varepsilon \}.
	$$
\end{definition}
%--------------------------------


%--------------------------------
\begin{example}
	\label{eg: Euclidean metrics}
	Let $X$ be any set, and let metric $\rho_p$ on $X^n$ ($n \in \mathbb Z_{>0}$) defined by
	$$
	\rho_p (x,y) = \left( \sum_{i = 1}^n |x_i - y_i|^p \right)^\frac{1}{p},
	$$
	where $p \in \mathbb R_{\ge 1}$. $\rho_2$ is so called the \textit{standard Euclidean metric}. If $X = \mathbb R$, then the metric space $(\mathbb R^n, \rho_2)$ is so-called \textit{Euclidean $n$-space}.
	
	For all $p,q \in \mathbb R_{\ge 1}$, if $p < q$, then for all $\varepsilon \in \mathbb R_{> 0}$ and for all $x,y \in X$, $\rho_p(x,y) \ge \rho_q(x,y)$; in particular, $\rho_p = \rho_q$ iff there is a unique $k \in \{1, \ldots, n\}$, such that for all $i \in \{1, \ldots, n\} \setminus \{k\}$, $x_i = 0$. As $\rho_p(x,y)$ is always ``overestimated'' than $\rho_q(x,y)$, we have $B_{\rho_p}(x, \varepsilon) \supseteq B_{\rho_q}(x,\varepsilon)$.
\end{example}
%--------------------------------


%--------------------------------
\begin{example}
	\label{eg: discrete metric}
	Let $X$ be any set. The \textit{discrete metric} $\rho$ on $X$ is defined to be
	$$
	\rho(x,y) =
	\begin{cases}
		0 & \text{if $x = y$}, \\
		1 & \text{otherwise}.
	\end{cases}
	$$
\end{example}
%--------------------------------


%--------------------------------
\begin{example}
	Let $a,b \in \mathbb R$ with $a< b$, and let metric $\rho_p$ on $C[a,b]$ defined by
	$$
	\rho_p (f, g) = \left( \int_a^b |f(t) - g(t)|^p dt \right)^\frac{1}{p},
	$$
	where $p \ge 1$. In particular,
	$$
	\rho_\infty (f,g) = \sup_{t \in [a,b]} |f(t) - g(t)|.
	$$
\end{example}
%--------------------------------

% todo: more examples
% 1. Hamming metric
% 2. Great circle metric
% 3. Hausdorff metric


%--------------------------------
\begin{theorem}
	Let $(X, \rho)$ be a metric space, and let
	$$
	\mathcal T_\rho = \left\{ \bigcup_{x \in U} B_\rho (x, \varepsilon) : U \subseteq X \land \varepsilon \in \mathbb R_{> 0} \right\}.
	$$
	$(X, \mathcal T_\rho)$ is a topological space; i.e., $\mathcal T_\rho$ satisfies the open set axioms (Definition \ref{def: topology}).
	
	\begin{proof}
		For O1. Apparently, there exists (for any) $\varepsilon \in \mathbb R_{> 0}$,
		$$
		X = \bigcup_{x \in X} B(x, \varepsilon).
		$$
		
		For O2. Let $I$ be an index set, and let $U_i \in T$ for all $i \in I$. There exists $\varepsilon \in \mathbb R_{> 0}$ we can define
		$$
		U = \bigcup_{i \in I} U_i = \bigcup_{i \in I} \bigcup_{x \in U_i} B(x, \varepsilon) = \bigcup_{x \in U} B(x, \varepsilon).
		$$
		Then, for all $x \in U$, there exists $B(x, \varepsilon) \ni x$. Thus $U \in \mathcal T$.
		
		For O3. Let $U, V \in \mathcal T$. If $U \cap V = \emptyset$, then the proof is done. Suppose $U \cap V \ne \emptyset$, and let $x \in U \cap V$. $U$ is open and $x \in U$, so there exists $r_1 \in \mathbb R_{> 0}$ such that $B(x, r_1) \subseteq U$; $V$ is open and $x \in V$, so there exists $r_2 \in \mathbb R_{> 0}$ such that $B(x, r_2) \subseteq V$.
		
		If $r_1 = r_2$, then $B(x, r_1) = B(x, r_2)$. If $r_1 \ne r_2$, say $r_1 < r_2$, then $B(x, r_1) \subseteq B(x, r_2) \subseteq V$. Above all, for any $x \in U \cap V$, there exits $r \in \mathbb R_{>0}$ such that $B(x, r) \subseteq U \cap V$. Thus, there exists $\varepsilon \in \mathbb R_{> 0}$ such that
		$$
		U \cap V = \bigcup_{x \in U \cap V} B(x, \varepsilon),
		$$
		i.e., $U \cap V \in \mathcal T_\rho$.
	\end{proof}
\end{theorem}
%--------------------------------



%--------------------------------
\begin{proposition}
	\label{prop: metric spaces are Hausdorff}
	Let $(X, \rho)$ be a metric space, then for all $x, y \in X$ ($x \ne y$), there is an $\varepsilon > 0$ such that $B(x, \varepsilon) \cap B (y, \varepsilon) = \emptyset$.
	
	\begin{proof}
		Suppose for all $\varepsilon > 0$, $B(x, \varepsilon) \cap B(y, \varepsilon) \ne \emptyset$, then there must be a $z \in X$ such that $z \in B(x, \varepsilon) \cap B(y, \varepsilon)$. $z \in B(x, \varepsilon)$ only if $\rho(x,z) < \varepsilon$, and $z \in B(y, \varepsilon)$ only if $\rho (z,y) < \varepsilon$. Thus
		$$
		\rho(x, z) + \rho(y,z) < 2\varepsilon.
		$$
		As the assumption holds for all $\varepsilon > 0$, we may put
		$$
		\varepsilon = \frac{\rho(x,y)}{2}.
		$$
		Then, we have
		$$
		\rho(x,z) + \rho(y,z) < \rho(x,y),
		$$
		which is impossible.
	\end{proof}
\end{proposition}
%--------------------------------


%--------------------------------
\begin{definition}
	[induced topologies]
	\label{def: induced topologies}
	Let $(X, \rho)$ be a metric space. A topology $\mathcal T$ on $X$ is said to be \textit{induced} by $\rho$ iff for all $\varepsilon > 0$, any $U \in \mathcal T$ is the union of ball(s) in $X$; i.e.,
	$$
	\mathcal T = \left\{ U \subseteq X :  U = \bigcup_{x \in X} B(x, \varepsilon) \right\}.
	$$
	
	In this case, $\mathcal T$ is called the \textit{underlying topology} of $\rho$.
\end{definition}
%--------------------------------


%--------------------------------
\begin{definition}
	[metrizable spaces]
	\label{def: metrizable spaces}
	Let $(X, \mathcal T)$ be a topological space. If there is any $\rho$ induce $\mathcal T$, then $(X, \mathcal T)$ is said to be \textit{metrizable}.
\end{definition}
%--------------------------------


%--------------------------------
\begin{definition}
	[Lipschitz equivalence]
	\label{def: Lipschitz equivalence}
	Let $X$ be any set, and let $\rho$ and $\rho'$ be metrics on $X$. $\rho$ and $\rho'$ are said to be \textit{Lipschitz equivalent} iff there exist $c, C > 0$, such that for all $x,y \in X$,
	$$
	c \rho(x,y) \le \rho'(x,y) \le C \rho(x,y).
	$$
\end{definition}
%--------------------------------


%--------------------------------
\begin{proposition}
	\label{prop: Lipschitz equivalence is a equivalence relation}
	Lipschitz equivalence is an equivalence relation.
	
	\begin{proof}
		Clearly, Definition \ref{def: Lipschitz equivalence} also holds for $\rho = \rho'$. So, Lipschitz equivalence is reflexive. In Definition \ref{def: Lipschitz equivalence}, the relation also holds for $\frac{1}{C} \rho' \le \rho \le \frac{1}{c} \rho'$. So Lipschitz equivalence is symmetric.
		
		If there is another $\rho''$ be Lipschitz equivalent to $\rho'$, then there is $r, R > 0$, such that for all $x,y \in X$,
		$$
		r\rho''(x,y) \le \rho'(x,y) \le R\rho''(x,y).
		$$
		By the conditions in Definition \ref{def: Lipschitz equivalence}, we have
		$$
		\frac{c}{r} \rho(x,y) \le \rho''(x,y) \le  \frac{C}{R} \rho(x,y),
		$$
		i.e., $\rho$ and $\rho''$ are also Lipschitz equivalent. So Lipschitz equivalence is transitive.
		
		Above all, Lipschitz equivalence is an equivalence relation.
	\end{proof}
\end{proposition}
%--------------------------------


%--------------------------------
\begin{proposition}
	\label{prop: Lipschitz equivalent metrics induces the same topology}
	Let $X$ be any set, and let $\rho$ and $\rho'$ be metrics on $X$. If $\rho$ and $\rho'$ are Lipschitz equivalent, then $\rho$ and $\rho'$ induce the same topology.
	
	\begin{proof}
		As $\rho$ and $\rho'$ are Lipschitz equivalent, by Definition \ref{def: Lipschitz equivalence}, there is a $c > 0$ such that for all $x,y \in X$,
		$$
		c \rho(x,y) \le \rho'(x,y).
		$$
		
		Given $r \in \mathbb R_{> 0}$ and for all $x \in X$, we have
		$$
		B_{\rho'}(x, cr) \subseteq B_{c \rho}(x, r) = B_\rho \left( x, \frac{1}{c} r \right).
		$$
		For all $U \in \mathcal T_\rho$, for all $x \in U$, there is an $\varepsilon \in \mathbb R_{> 0}$, such that
		$$
		B_{\rho'} (x, \varepsilon) \subseteq B_{\rho}(x, \varepsilon) \subseteq U.
		$$
		So $U \in \mathcal T_\rho'$. Then we have $\mathcal T_{\rho} \subseteq \mathcal T_{\rho'}$.
		
		Similarly, $U \in \mathcal T_{\rho'}$ only if $U \in \mathcal T_\rho$. Then we have $\mathcal T_{\rho'} \subseteq \mathcal T_{\rho}$.
		
		Above all, $\mathcal T_{\rho} = \mathcal T_{\rho'}$.
	\end{proof}
\end{proposition}
%--------------------------------


%--------------------------------
\begin{note}
	In this proposition, $\mathcal T_\rho$ and $\mathcal T_{\rho'}$ are said to be homeomorphic or topologically equivalent (see Definition \ref{def: homomorphic}). And $\rho$ and $\rho'$ are also said to be topologically equivalent.
\end{note}
%--------------------------------


%--------------------------------
\begin{example}
	In Example \ref{eg: Euclidean metrics}, for all $p,q \ge 1$, all $\rho_p$ and $\rho_q$ induce the same topology. Let $X$ be any subset of $\mathbb R^n$, then for all $x,y \in X$, if $p < q$, then
	$$
	\rho_p (x,y) \ge \rho_q (x,y).
	$$
	Thus, if $\rho_1$ and $\rho_\infty$ are Lipschitz equivalent, then any other $\rho_p$ and $\rho_q$ are Lipschitz equivalent. We have
	$$
	\rho_1 (x,y) = \sum_{i = 1}^n |x_i - y_i| \ge \max_{i \in \{1, \ldots, n\}} |x_i - y_i| = \rho_\infty (x,y).
 	$$
 	Clearly,
 	$$
 	\rho_\infty (x,y) \le \rho_1 (x,y) \le n \rho_\infty (x,y).
 	$$
 	By Definition \ref{def: Lipschitz equivalence}, $\rho_1$ and $\rho_\infty$ are Lipschitz equivalent, hence for all $p, q \ge 1$, $\rho_p$ and $\rho_q$ are Lipschitz equivalent. Thus, by Proposition \ref{prop: Lipschitz equivalent metrics induces the same topology}, they induce the same topology.
\end{example}
%--------------------------------






























