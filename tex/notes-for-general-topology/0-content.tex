\tableofcontents



%================================
%::::::::::::::::::::::::::::::::
\chapter{Metric Spaces}
%::::::::::::::::::::::::::::::::
%================================


%================================
\section{Metric Spaces}
%================================


%--------------------------------
\begin{definition}
	\label{definition: metric space}
	Let $X$ be any set. A mapping $d: X \times X \to \mathbb R_{\ge 0}$ is \textit{metric on $X$} iff it satisfies the \textit{metric axioms}. That is, for any $x, y, z \in X$:
	\begin{enumerate}[\bf M1. ]
		\item $d(x,y) = 0$ iff $x = y$;
		\item $d(x,y) = d(y,x)$;
		\item $d(x, z) \le d(x,y) + d(y,z)$.
	\end{enumerate}
	
	In this case, the pair $M = (X, d)$ is called a \textit{metric space}.
\end{definition}
%--------------------------------


%--------------------------------
\begin{definition}
	\label{definition: ball}
	A $M = (X, d)$ be a metric space, let $x \in X$ and let $\varepsilon \in \mathbb R_{> 0}$. An \textit{open $\varepsilon$-ball}, or just $\varepsilon$-ball, about $x$ is defined to be the set
	$$
	B_\varepsilon (x; d) = \{ y \in X : d(x,y) < \varepsilon \}.
	$$
	
	A \textit{closed ball} is defined to be the set
	$$
	\overline{B}_\varepsilon (x; d) = \{ y \in X : d(x,y) \le \varepsilon \}.
	$$
\end{definition}
%--------------------------------


%--------------------------------
\begin{note}
	As
	$$
	M = (X, d), \ M' = (X, d'), \ M'' = (X, d''), \ \ldots
	$$
	are different although they share the same set $X$, for any $x \in X$ and any $\varepsilon \in \mathbb R_{> 0}$,
	$$
	B_\varepsilon(x; d),\ B_\varepsilon (x; d'), \ B(x; d''), \ \ldots
	$$
	are also different. However, if confusion is unlikely, we simply write ``$B_\varepsilon(x)$'' for ``$B_\varepsilon(x; d)$''.
\end{note}
%--------------------------------


%--------------------------------
\begin{example}
	The \textit{Euclidean metric space} $M = (X, d)$ is an $n$-dimensional set $X$ equipped with the \textit{Euclidean metric} $d$ defined as
	$$
	d(x,y) = \left( \sum_{i = 1}^n |x_i - y_i|^2 \right)^\frac{1}{2}.
	$$
	
	This is also called \textit{standard Euclidean metric}, in contrast to the \textit{non-standard Euclidean metrics}
	$$
	d_p(x,y) = \left( \sum_{i = 1}^n |x_i - y_i|^p \right)^\frac{1}{p}, \quad p \ge 1.
	$$
	
	In particular,
	$$
	d_\infty (x,y) = \max_{1 \le i \le n} |x_i - y_i|.
	$$
\end{example}
%--------------------------------


%--------------------------------
\begin{example}
	A \textit{discrete metric space} $M = (X, d)$ is a set $X$ equiped with the \textit{discrete metric} $d$ defined as
	$$
	d(x,y) =
	\begin{cases}
		0, & \text{if $x = y$}; \\
		1, & \text{else}.
	\end{cases}
	$$
	
	This is an equivalent definition of the discrete metric:
	$$
	d(x, y) = (\mathrm{sgn}(d'(x,y)))^2,
	$$
	where $\mathrm{sgn}(\cdot)$ is a \href{https://en.wikipedia.org/wiki/Sign_function}{sign function}, and $d'$ is any metric on $X$.
\end{example}
%--------------------------------


%--------------------------------
\begin{example}
	\footnote{
		See \href{https://en.wikipedia.org/wiki/Minkowski_inequality}{Minkowski inequality}.
	}
	Denote $C[a,b]$ for the set of all continuous mapping $\mathbb R_{[a,b]} \to \mathbb R$. On $C[a,b]$, we can define a metric $d$ as
	$$
	d_p(f, g) = \left( \int_{a}^{b} |f(t) - g(t)|^p \mathrm{d} t \right)^\frac{1}{p}, \quad p \ge 1.
	$$
	
	In particular,
	$$
	d_\infty (f,g) = \sup_{t \in \mathbb R_{[a,b]}} |f(t) - g(t)|.
	$$
\end{example}
%--------------------------------


%--------------------------------
\begin{example}
	\footnote{
		See \href{https://en.wikipedia.org/wiki/Hausdorff_distance}{Hausdorff distance}.
	}
	Let $M = (X, d)$ be a metric space. The \textit{Hausdorff metric} $d_H$ on $2^X \setminus \{\emptyset\}$ is defined as
	$$
	d_H = \max \left\{ \sup_{x \in X}d(x,Y), \sup_{y \in Y} d(y, X)\right\},
	$$
	where
	$$
	\begin{aligned}
		d(x,Y) = \inf_{y \in Y}(x,y), \text{ and } d(y, X) = \inf_{x \in X} (y, x).
	\end{aligned}
	$$
\end{example}
%--------------------------------


%================================
\section{Open sets in Metric Spaces}
%================================


%--------------------------------
\begin{definition}
	\label{definition: open set in metric space}
	Let $M = (X, d)$ be a metric space, and let $U \subseteq X$. $U$ is said to be \textit{open in $M$}, iff for any $y \in U$, there exists $\varepsilon \in \mathbb R_{> 0}$, such that $B_\varepsilon(y) \subseteq U$.
\end{definition}
%--------------------------------


%--------------------------------
\begin{lemma}
	\label{lemma: open balls of point inside open ball}
	Let $M = (X, d)$ be a metric space, let $x \in A$ and let $\varepsilon \in \mathbb R_{> 0}$. For any $y \in B_\varepsilon (x)$, there is a $\delta \in \mathbb R_{> 0}$ such that $B_\delta (y) \subseteq B_\varepsilon(x)$.
	
	\begin{proof}
		For any $y \in B_\varepsilon (x)$, by the definition of open balls (Definition \ref{definition: ball}), we have $d(x,y) < \varepsilon$.
		
		Let $\delta \in \mathbb R_{> 0}$ such that $\delta + d(x,y) = \varepsilon$.
		
		By M3 in metric axioms (Definition \ref{definition: metric space}), for any $z \in A$ with $d(y,z) < \delta$, we have
		$$
		d(x, z) \le d(y, z) + d(x, y) < \varepsilon.
		$$
		
		Thus, again, by the definition of open balls, we have $B_\delta(y) \subseteq B_\varepsilon(x)$.
		
		\qedlemma
	\end{proof}
\end{lemma}
%--------------------------------


%--------------------------------
\begin{theorem}
	\footnote{
		Shared on \href{https://proofwiki.org/wiki/Set_is_Open_iff_Union_of_Open_Balls}{ProofWiki}.
	}
	Let $M = (X, d)$ be a metric space, and let $U \subseteq X$. $U$ is open in $M$ iff it is a union of open balls.
	
	\begin{proof}
		First, prove $\Rightarrow$.
		
		As $U$ is open, for any $y \in U$, there exists $\varepsilon_y \in \mathbb R_{> 0}$ such that $B_{\varepsilon_y}(y) \subseteq U$.
		
		Therefore,
		$$
		U = \bigcup_{y \in U} B_{\varepsilon_y} (y).
		$$
		
		$\qedlemma$
		
		Now, prove $\Leftarrow$.
		
		Aiming for a contradiction, suppose $U$ is a union of open balls but not open.
		
		As $U$ is not open, there is a $y \in U$ such that for any $\varepsilon \in \mathbb R_{> 0}$, $B_\varepsilon (y) \not \subseteq U$.
		
		As $U$ is a union of open balls, there is an $x \in U$ and $r \in \mathbb R_{> 0}$ such that $y \in B_r (x)$.
		
		By Lemma \ref{lemma: open balls of point inside open ball}, there exists a $\delta \in \mathbb R_{> 0}$ such that $B_\delta (y) \subseteq B_r (x)$.
		
		This is a contradiction by the assumption.
		
		Thus, $U$ has to be open.
		
		\qed
	\end{proof}
\end{theorem}
%--------------------------------


%--------------------------------
\begin{theorem}
	\label{theorem: metric space is hausdorff}

	Let $M = (X, d)$ be any metric space. $M$ is \textit{Hausdorff}. That is, For any distinct points $x,y \in X$, we can always find an $\varepsilon \in \mathbb R_{> 0}$ such that
	$$
	B_\varepsilon(x) \cap B_\varepsilon(y) = \emptyset.
	$$
	
	\begin{proof}
		Aiming for a contradiction, suppose there are $x,y \in X$ with $x \ne y$, such that for any $\varepsilon \in \mathbb R_{> 0}$, we can always find a $z \in X$ such that
		$$
		z \in B_\varepsilon(x) \cap B_\varepsilon(y).
		$$
		
		Let $r = d(x,y)/2$, and let $z \in B_r(x) \cap B_r(y)$.
		
		As $z \in B_r(x)$, by the definition of open balls (Definition \ref{definition: ball}), $d(x,z) < r$; as $z \in B_r(y)$, similarly, $d(y,z)< r$. Then we have
		$$
		d(x, z) + d(y, z) < 2r = d(x,y).
		$$
		
		By M3 in metric axioms (Definition \ref{definition: metric space}), this is impossible.
		
		\qed
	\end{proof}
\end{theorem}
%--------------------------------


%--------------------------------
\begin{definition}
	\label{definition: closed set in metric space}
	Let $M = (X, d)$ be any metric space, and let $V \subseteq X$. $V$ is said to be \textit{closed} in $M$, iff there is an open set $U$ satisfies $X \setminus U = V$.
\end{definition}
%--------------------------------


%--------------------------------
\begin{lemma}
	In a metric space, any singleton is closed.
	
	\begin{proof}
		Let $M=(X, d)$ be a metric space, let $x \in X$, and let $y \in X \setminus \{x\}$.
		
		As $M$ is Hausdorff (Theorem \ref{theorem: metric space is hausdorff}), there is an $\varepsilon \in \mathbb R_{> 0}$ such that
		$$
		0 < \varepsilon < d(x,y),
		$$
		thus $X \setminus \{x\}$ is open, hence, by Definition \ref{definition: metric space}, its complement $\{x\}$ is open.
		
		\qedlemma
	\end{proof}
\end{lemma}
%--------------------------------


%--------------------------------
\begin{theorem}
	Let $M = (X, d)$ be a metric space, denote $\mathcal T$ for the family of open subsets of $X$. Then $\mathcal T$ satisfies the following conditions:
	
	\begin{enumerate}[\bf O1.]
		\item $X, \emptyset \in \mathcal T$;
		\item For any $\mathcal U \subseteq \mathcal T$, $\bigcup \mathcal U \in \mathcal T$; in words, $\mathcal T$ is closed under arbitrary union;
		\item For any finite $\mathcal V \subseteq \mathcal T$, $\bigcap \mathcal V \in \mathcal T$; in words, $\mathcal T$ is closed under finite intersection.
	\end{enumerate}
	
	\begin{proof}
		First, prove O1.
		
		As $\emptyset$ is the subset of any set, $\emptyset \in \mathcal T$. $\bigcup \emptyset = \emptyset \in \mathcal T$.
		
		By Definition \ref{definition: closed set in metric space}, $X = X \setminus \emptyset$.
		
		\qedlemma
		
		Then, prove O2.
		
		Let $\mathcal U \subseteq \mathcal T$, and denote $\mathcal O$ for the open balls in $M$.
		
		For any $U \in \mathcal U$, there is an $\mathcal O_U \subseteq \mathcal O$ such that $U = \bigcup \mathcal O_U$. 
		
		Let $\mathcal O: \mathcal T \to $
		
		Then we have
		$$
		\bigcup \mathcal U = \bigcup_{U \in \mathcal U} \left( \bigcup \mathcal O(U) \right).
		$$
		
		Thus $\mathcal T$ is closed under arbitrary union.
	\end{proof}
\end{theorem}
%--------------------------------





































%