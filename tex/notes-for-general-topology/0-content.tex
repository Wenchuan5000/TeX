\tableofcontents



%================================
%::::::::::::::::::::::::::::::::
\chapter{Topological Spaces}
%::::::::::::::::::::::::::::::::
%================================


%================================
\section{Metric Spaces}
%================================


%--------------------------------
\begin{definition}
	\label{definition: metric space}
	Let $X$ be any set. A mapping $d: X \times X \to \mathbb R_{\ge 0}$ is \textit{metric on $X$} if and only if it satisfies the \textit{metric axioms}. That is, for any $x, y, z \in X$:
	\begin{enumerate}[\bf M1. ]
		\item $d(x,y) = 0$ if and only if $x = y$;
		\item $d(x,y) = d(y,x)$;
		\item $d(x, z) \le d(x,y) + d(y,z)$.
	\end{enumerate}
	
	In this case, the pair $M = (X, d)$ is called a \textit{metric space}.
\end{definition}
%--------------------------------


%--------------------------------
\begin{definition}
	A $M = (X, d)$ be a metric space, let $x \in X$ and let $\varepsilon \in \mathbb R_{> 0}$. An \textit{open $\varepsilon$-ball}, or just $\varepsilon$-ball, about $x$ is defined to be the set
	$$
	B_\varepsilon (x; d) = \{ y \in X : d(x,y) < \varepsilon \}.
	$$
	
	A \textit{closed ball} is defined to be the set
	$$
	\overline{B}_\varepsilon (x; d) = \{ y \in X : d(x,y) \le \varepsilon \}.
	$$
\end{definition}
%--------------------------------


%--------------------------------
\begin{note}
	As
	$$
	M = (X, d), \ M' = (X, d'), \ M'' = (X, d''), \ \ldots
	$$
	are different although they share the same set $X$, for any $x \in X$ and any $\varepsilon \in \mathbb R_{> 0}$,
	$$
	B_\varepsilon(x; d),\ B_\varepsilon (x; d'), \ B(x; d''), \ \ldots
	$$
	are also different. However, if confusion is unlikely, we simply write ``$B_\varepsilon(x)$'' for ``$B_\varepsilon(x; d)$''.
\end{note}
%--------------------------------


%--------------------------------
\begin{example}
	The \textit{Euclidean metric space} $M = (X, d)$ is an $n$-dimensional set $X$ equipped with the \textit{Euclidean metric} $d$ defined as
	$$
	d(x,y) = \left( \sum_{i = 1}^n |x_i - y_i|^2 \right)^\frac{1}{2}.
	$$
	
	This is also called \textit{standard Euclidean metric}, in contrast to the \textit{non-standard Euclidean metrics}
	$$
	d_p(x,y) = \left( \sum_{i = 1}^n |x_i - y_i|^p \right)^\frac{1}{p}, \quad p \ge 1.
	$$
	
	In particular,
	$$
	d_\infty (x,y) = \max_{1 \le i \le n} |x_i - y_i|.
	$$
\end{example}
%--------------------------------


%--------------------------------
\begin{example}
	A \textit{discrete metric space} $M = (X, d)$ is a set $X$ equiped with the \textit{discrete metric} $d$ defined as
	$$
	d(x,y) =
	\begin{cases}
		0, & \text{if $x = y$}; \\
		1, & \text{else}.
	\end{cases}
	$$
	
	This is an equivalent definition of the discrete metric:
	$$
	d(x, y) = (\mathrm{sgn}(d'(x,y)))^2,
	$$
	where $\mathrm{sgn}(\cdot)$ is a \href{https://en.wikipedia.org/wiki/Sign_function}{sign function}, and $d'$ is any metric on $X$.
\end{example}
%--------------------------------


%--------------------------------
\begin{example}
	\footnote{
		See \href{https://en.wikipedia.org/wiki/Minkowski_inequality}{Minkowski inequality}.
	}
	Denote $C[a,b]$ for the set of all continuous mapping $\mathbb R_{[a,b]} \to \mathbb R$. On $C[a,b]$, we can define a metric $d$ as
	$$
	d_p(f, g) = \left( \int_{a}^{b} |f(t) - g(t)|^p \mathrm{d} t \right)^\frac{1}{p}, \quad p \ge 1.
	$$
	
	In particular,
	$$
	d_\infty (f,g) = \sup_{t \in \mathbb R_{[a,b]}} |f(t) - g(t)|.
	$$
\end{example}
%--------------------------------


%--------------------------------
\begin{example}
	\footnote{
		See \href{https://en.wikipedia.org/wiki/Hausdorff_distance}{Hausdorff distance}.
	}
	Let $M = (X, d)$ be a metric space. The \textit{Hausdorff metric} $d_H$ on $2^X \setminus \{\emptyset\}$ is defined as
	$$
	d_H = \max \left\{ \sup_{x \in X}d(x,Y), \sup_{y \in Y} d(y, X)\right\},
	$$
	where
	$$
	\begin{aligned}
		d(x,Y) = \inf_{y \in Y}(x,y), \text{ and } d(y, X) = \inf_{x \in X} (y, x).
	\end{aligned}
	$$
\end{example}
%--------------------------------


%================================
\section{Open sets in Metric Spaces}
%================================


%--------------------------------
\begin{definition}
	Let $M = (X, d)$ be a metric space, and let $U \subseteq X$. $U$ is said to be \textit{open in $M$} if and only if for any $y \in U$, there exists $\varepsilon \in \mathbb R_{> 0}$, such that $B_\varepsilon(y) \subseteq U$.
\end{definition}
%--------------------------------

%--------------------------------
\begin{lemma}
	Let $M = (X, d)$ be a metric space, let $x \in A$ and let $\varepsilon \in \mathbb R_{> 0}$. For any $y \in B_\varepsilon (x)$, there is a $\delta \in \mathbb R_{> 0}$ such that $B_\delta (y)$
\end{lemma}
%--------------------------------












































%