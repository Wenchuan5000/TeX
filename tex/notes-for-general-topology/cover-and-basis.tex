%================================
\section{Cover and Basis}
%================================


%--------------------------------
\begin{definition}
	[cover]
	\label{def: cover}
	Let $(X, \mathcal T)$ be a topological space, and let $U \subseteq X$, then a family $\mathcal C \subseteq \mathcal P(X)$ is called a \textit{cover} of $U$ iff the union of all sets in $\mathcal C$ is a superset of $U$. That is,
	$$
	U \subseteq \bigcup \mathcal C.
	$$
	
	If $\mathcal C \subseteq \mathcal T$, then we call $\mathcal C$ an \textit{open cover} of $U$.
	
	Let $\mathcal S \subseteq \mathcal C$, iff the union of $\mathcal S$ is still a superset of $U$, then we call $\mathcal S$ a \textit{subcover} of $\mathcal C$.
\end{definition}
%--------------------------------


%--------------------------------
\begin{definition}
	[basis]
	\label{def: basis}
	\label{theorem: alt def of basis}
	Let $(X, \mathcal T)$ be a topological space, and let $\mathcal B\subseteq \mathcal T$. $\mathcal B$ is a \textit{base} for $\mathcal T$ iff it satisfies the following properties.
	\begin{enumerate}[(i)]
		\item For each $x \in X$, there exists $B \in \mathcal B$ such that $x \in B$, i.e., $\bigcup \mathcal B = X$.
		\item For each $x \in B_1 \cap B_2$ where $B_1, B_2 \in \mathcal B$, there exists $B_3 \ni x$ such that $B_3 \subseteq B_1 \cap B_2$.
	\end{enumerate}
\end{definition}
%--------------------------------


%--------------------------------
\begin{note}
	A base for a topology is not necessarily closed under finite intersections. For example, let
	$$
	\mathcal I = \left\{ (a,b) : 0 < b - a < 1 \right\} \in \mathcal P(\mathbb R),
	$$
	and let
	$$
	\mathcal J = \left\{ (-\infty, 2), (-2, \infty) \right\} \in \mathcal P(\mathbb R).
	$$
	$\mathcal I \cup \mathcal J$ is a basis for the Euclidean topology on $\mathbb R$, but $\mathcal I \cup \mathcal J$ is not closed under finite intersection, for $\bigcap \mathcal J = (-2, 2)$ is not a member of $\mathcal I \cup \mathcal J$.
\end{note}
%--------------------------------


%--------------------------------
\begin{corollary}
	\label{coro: alt def of basis}
	Let $(X, \mathcal T)$ be a topological space and let $\mathcal B$ be a basis for $\mathcal T$. For all $U \in \mathcal T$, there exists $\mathcal I, \mathcal J \subseteq \mathcal B$ ($\mathcal J$ finite) such that
	$$
	U = \bigcup \mathcal I = \bigcap \mathcal J.
	$$
	% todo: POOF
\end{corollary}
%--------------------------------


%--------------------------------
\begin{example}
	Let $(\mathbb R, \mathcal T)$ be a topological space where $\mathcal T$ is induce from the standard Euclidean metric, and let $\mathcal B$ be the family of all open proper intervals in $\mathbb R$. $\mathcal B$ is a base for $\mathcal T$.
	
	However, if $\mathcal T$ is a discrete topology on $\mathbb R$, $\mathcal B$ is not a base for $\mathcal T$, for some open subsets of $\mathbb R$, such as $\{0\}$, is neither any arbitrary union nor any finite intersection of $\mathcal B$-sets.
\end{example}
%--------------------------------


%--------------------------------
\begin{theorem}
	Let $(X, \mathcal T)$ and $(X, \mathcal T')$ be topological spaces, and let $\mathcal B$ and $\mathcal B'$ be the basis for $\mathcal T$ and $\mathcal T'$ respectively. $\mathcal T'$ is finer than $\mathcal T$ iff for any $x \in X$ and for any $B \in \mathcal B'$ with $B \ni x$, there exists $B' \in \mathcal B'$ such that $x \in B' \subseteq B$.
	
	\begin{proof}	
		For $\implies$. Let $B \in \mathcal B$. Clearly, $B \in \mathcal T$. Also, $B \in \mathcal T'$ for $\mathcal T \subseteq \mathcal T'$. By Corollary \ref{coro: alt def of basis}, there exists $\mathcal I' \subseteq \mathcal B'$ such that $B = \bigcup \mathcal I'$. Clearly, all $\mathcal I'$-sets are subsets of $B$.
		
		For $\impliedby$. For all $B \in \mathcal B$ and for all $x \in B$, there exists $B' \in \mathcal B'$ with $x \in B' \subseteq B$. Let $\mathcal J'$ denote all such $B'$, then we have $B = \bigcup \mathcal J'$. By Definition \ref{def: basis}, we have $B \in \mathcal B \subseteq \mathcal T'$. Above all, $\mathcal T \subseteq \mathcal T$.
	\end{proof}
\end{theorem}
%--------------------------------





%--------------------------------
\begin{note}
	In this theorem, it is not necessary that $\mathcal B \subseteq \mathcal B'$. For example, let 
	$$
	\begin{aligned}
		& \mathcal B = \left\{ (a, b) \subseteq \mathbb R : b - a = 1 \right\}, \\
		& \mathcal B' = \left\{ (a, b) \subseteq \mathbb R : b - a = 2 \right\}.
	\end{aligned}
	$$
	Obviously, $\mathcal B \not \subseteq \mathcal B'$ and $\mathcal B' \not \subseteq \mathcal B$, but they generate exactly the same topologies on $\mathbb R$.

	For another example, let
	$$
	\begin{aligned}
		& \mathcal B = \left\{ \text{all open intervals in $\mathbb R$}\right\}, \text{ and} \\
		& \mathcal B' = \left\{ \text{all singletons in $\mathbb R$} \right\}.
	\end{aligned}
	$$
	In this case, also, $\mathcal B \not \subseteq \mathcal B'$ and $\mathcal B' \not \subseteq \mathcal B$, but $\mathcal B'$ generates the discrete topology which is the finest topology on $\mathbb R$.
\end{note}
%--------------------------------


%%--------------------------------
%\begin{theorem}
%	Let $(X, \mathcal T)$ be a topological space be genrated by a base $\mathcal B$. For all $U \in \mathcal T$, there is a $B \in \mathcal B$ such that $U \subseteq \mathcal B$.
%	
%	\begin{proof}
%		By Definition \ref{def: generated by basis}, if $\mathcal T$ is generated by $\mathcal B$, then for all $U \in \mathcal T$, there is an finite set $I$, such that
%		$$
%		U = \bigcap_{i \in I} B_i, \quad B_i \in \mathcal B.
%		$$
%		Thus, for at least one $k \in I$, $U \subseteq B_k$.
%	\end{proof}
%\end{theorem}
%%--------------------------------
%
%
%%--------------------------------
%\begin{proposition}
%	Let $X$ be any set, and let $\mathcal T$ and $\mathcal T'$ be its topologies generated by basis $\mathcal B$ and $\mathcal B'$ respectively. Then $\mathcal T'$ is finer than $\mathcal T$ iff for any $B \in \mathcal B$, there is a $B' \in \mathcal B'$ such that $B' \subseteq B$.
%	
%	\begin{proof}
%		If $\mathcal T$ is generated by $\mathcal B$, then for all $U' \in \mathcal T'$,
%		$$
%		U' = \bigcup_{j \in J} B_j',
%		$$
%		where $B_j \in \mathcal B$.
%		
%		As $\mathcal T$ is generated by $\mathcal B$, then, certainly, $\mathcal B \subseteq \mathcal T$.
%
%		By the conditions we have, $\mathcal T \subseteq \mathcal T'$ iff for all $B \in \mathcal B$, there is $W' \in \mathcal T$ such that
%		$$
%		B = W' = \bigcup_{i \in I} B_i',
%		$$
%		where $B_i' \in \mathcal B'$. Certainly, all such $B_i'$ are contained in $B$.
%	\end{proof}
%\end{proposition}
%%--------------------------------
%
%
%%--------------------------------
%\begin{proposition}
%	Let $X$ be any set, and let $\mathcal T \subseteq \mathcal P(X)$. $\mathcal T$ is a topology on $X$ iff it generates itself.
%	
%	\begin{proof}
%		If $\mathcal T$ is a topology on $X$, then, by Definition \ref{def: generated by basis}, any open set generated by $\mathcal T$ is still a member of $\mathcal T$. On the other hand, if $\mathcal T$ generates itself, then, $\emptyset$ and $X$ must be members of $\mathcal T$, and, by Definition \ref{def: generated by basis}, $\mathcal T$ is a topology on $X$.
%	\end{proof}
%\end{proposition}
%%--------------------------------


% todo: check this link to fill the propositions.
% 
% https://en.wikipedia.org/wiki/Base_(topology)#Theorems
% 
% check the "local base" theorem in this link.




























%