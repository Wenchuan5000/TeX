%================================
\section{Interiors}
%================================


Let $(X, \mathcal T)$ be a topological space, and let $A, B \subseteq X$.


%--------------------------------
\begin{definition}
	The \textit{interior} of $A$, denoted $A^\circ$, is defined to be the union of all open subsets of $A$; i.e.,
	$$
	A^\circ = \bigcup \mathcal U, \quad \mathcal U = \mathcal P(X) \cap \mathcal T.
	$$
\end{definition}
%--------------------------------


%--------------------------------
\begin{proposition}
	$A^\circ \subseteq A$.
	
	\begin{proof}
		Let $\mathcal U = \mathcal P(A) \cap \mathcal T$. Clearly
		$$
		A^\circ = \bigcup \mathcal U \subseteq A.
		$$
	\end{proof}
\end{proposition}
%--------------------------------


%--------------------------------
\begin{proposition}
	$A \in \mathcal T$ if and only if $A = A^\circ$.
	
	\begin{proof}
		Let $\mathcal U = \mathcal P(A) \cap \mathcal T$.
		
		$\mathcal U$ is closed under arbitrary union, so if $A = A^\circ = \bigcup \mathcal U$, then $A \in \mathcal T$.
		
		On the other hand, suppose $A \in \mathcal T$ but $A \ne A^\circ$, then $A^\circ \subsetneq A$. As $A \in \mathcal T$, there exists $U \in \mathcal P(A) \cap \mathcal T$ with $U \ni x$. But as $x \in X \setminus A^\circ$, $U$ could not be a subset of $A^\circ$. Then we have $U \in \mathcal T$ but $U \not \subseteq \bigcup \mathcal U$, which is contradicted to the assumption.
	\end{proof}
\end{proposition}
%--------------------------------


%--------------------------------
\begin{proposition}
	$(A \cap B)^\circ = A^\circ \cap B^\circ$.
	
	\begin{proof}
		$(A \cap B)^\circ \in \mathcal T$, so there exists $\mathcal U \subseteq \mathcal T$ such that 
		$$
		\bigcup \mathcal U = (A \cap B)^\circ.
		$$
		Clearly, $\mathcal U = \mathcal P(A \cap B) \cap \mathcal T$.
		
		Let $\mathcal I = \mathcal P(A) \cap \mathcal T$ and $\mathcal J = \mathcal P(B) \cap \mathcal T$, then $\mathcal I = A^\circ$ and $\mathcal J = B^\circ$. 
		
		$$
		\begin{aligned}
			\mathcal I \cap \mathcal J &= \mathcal P(A) \cap \mathcal T \cap \mathcal P(B) \cap \mathcal T \\
			&= \mathcal P(A \cap B) \cap \mathcal T \\
			&= \mathcal U.
		\end{aligned}
		$$
		Then we have
		$$
		(A \cap B)^\circ = \bigcup \mathcal U = \bigcup \mathcal I \cup \bigcup \mathcal J = A^\circ \cap B^\circ.
		$$
	\end{proof}
\end{proposition}
%--------------------------------


%--------------------------------
\begin{proposition}
	If $A \subseteq B$, then $A^\circ \subseteq B^\circ$.
	
	\begin{proof}
		Let $\mathcal U = \mathcal P(A) \cap \mathcal T$, and let $\mathcal V = \mathcal P(B) \cap \mathcal T$
	
		$A^\circ \subseteq A$, so $A \subseteq B$ implies $A^\circ \subseteq B$, so $\mathcal U \subseteq \mathcal V$. Thus,
		$$
		A^\circ = \bigcup \mathcal{U} \subseteq \bigcup \mathcal V = B^\circ.
		$$
	\end{proof}
\end{proposition}
%--------------------------------


%--------------------------------
\begin{note}
	$A^\circ \subseteq B^\circ$ dose not implies $A \subseteq B$. For example, let $X = \mathbb R^n$, let $\mathcal T$ be induce by Euclidean metric $\rho$ on $X$, and let
	$$
	\begin{aligned}
		& B = B_\rho (\vec 0, 1), \text{ and} \\
		& A = \{ -\hat e_1 + (- \hat e_1 - \hat e)t : t \in (0, 1] \},
	\end{aligned}
	$$
	where $\hat e_i$ denotes the $i$-th unit vector in $X$.
	
	Clearly $A^\circ = \emptyset \subseteq B^\circ$, but $A \not \subseteq B$ for $\hat e_1 \in A \setminus B$.
\end{note}
%--------------------------------



%--------------------------------
\begin{proposition}
	Let $\mathcal T$ be induced by a metric $\rho$ on $X$. The interior of $A^\circ$ is the union of all open balls in $A$; i.e., there exists $\varepsilon \in \mathbb R_{>0}$ such that
	$$
	A^\circ = \bigcup_{x \in A} B(x, \varepsilon).
	$$
	
	\begin{proof}
		Let $\mathcal U = \mathcal P(A) \cap \mathcal T$, then any $U \in \mathcal U$,
		$$
		U = \bigcup_{x \in U} B(x, \varepsilon).
		$$
	
		For any open subsets $U \in \mathcal U$ and for any $x \in U$, there exists $\varepsilon \in \mathbb R_{> 0}$ such that
		$$
		A^\circ = \bigcup \mathcal U = \bigcup_{U \in \mathcal U} \bigcup_{x \in U} B(x, \varepsilon) = \bigcup_{x \in A} B(x, \varepsilon).
		$$
	\end{proof}
\end{proposition}
%--------------------------------


%================================
\section{Closures}
%================================

Let $(X, \mathcal T)$ be a topological space, and let $A, B \subseteq X$.

%--------------------------------
\begin{definition}
	The \textit{closure} of $A$, denoted $\overline A$, is defined to be the intersection of all closed supersets of $A$.
\end{definition}
%--------------------------------


%--------------------------------
\begin{proposition}
	$X \setminus A^\circ = \overline{X \setminus A}$.
	
	\begin{proof}
		Let $\mathcal U = \mathcal P(A) \cap \mathcal T$, then $A^\circ = \bigcup \mathcal U$. By De Morgan's Law,
		$$
		X \setminus A^\circ = \bigcap_{U \in \mathcal U} (X \setminus U).
		$$
		As $\mathcal U$ is the family of all open subsets of $A$, the set of all $X \setminus U$ is the family of all closed superset of $X \setminus A$. Thus
		$$
		\bigcap_{U \in \mathcal U}(X \setminus U) = \overline{X \setminus A}.
		$$
	\end{proof}
\end{proposition}
%--------------------------------


%--------------------------------
\begin{proposition}
	$\overline A$ is closed.
	
	\begin{proof}
		As it is the intersection of some closed sets, $\overline A$ is closed.
	\end{proof}
\end{proposition}
%--------------------------------


%--------------------------------
\begin{proposition}
	$\overline A$ is closed if and only if $A = \overline A$.
	
	\begin{proof}
		Let $\mathcal U = \mathcal P(X \setminus A) \cap \mathcal T$, then we have
		$$
		\begin{aligned}
			X \setminus \bigcup \mathcal U &= X \setminus (X \setminus A)^\circ \\
			&= \overline{X \setminus (X \setminus A)} \\
			&= \overline A.
		\end{aligned}
		$$
	\end{proof}
\end{proposition}
%--------------------------------


%--------------------------------
\begin{proposition}
	If $A \subseteq B$, then $\overline A \subseteq \overline B$.
	
	\begin{proof}
		$$
		\begin{aligned}
			A \subseteq B &\iff (X \setminus A) \supseteq (X \setminus B) \\
			&\implies (X \setminus A)^\circ \supseteq (X \setminus B)^\circ \\
			&\iff X \setminus (X \setminus A)^\circ \subseteq X \setminus (X \setminus B)^\circ \\
			&\iff \overline{X \setminus (X \setminus A)} \subseteq \overline{X \setminus (X \setminus B)} \\
			&\iff \overline A \subseteq \overline B.
		\end{aligned}
		$$
	\end{proof}
\end{proposition}
%--------------------------------


%--------------------------------
\begin{proposition}
	If $A$ is closed, then $A \supseteq B$ if and only if $A \supseteq \overline B$.
	
	\begin{proof}
		$A$ is closed, so $X \setminus A$ is open. $A \supseteq B$ if and only if $X \setminus A \subseteq X \setminus B$. These conditions hold if and only if $X \setminus A \subseteq (X \setminus B)^\circ$. This holds if and only if
		$$
		\begin{aligned}
		X \setminus A \subseteq (X \setminus B)^\circ &\iff X \setminus (X \setminus A) \supseteq X \setminus (X \setminus B)^\circ \\
		&\iff A \supseteq \overline B.
		\end{aligned}
		$$
	\end{proof}
\end{proposition}
%--------------------------------


%================================
\section{Boundary Sets}
%================================

Let $(X, \mathcal T)$ be a topological space, and let $A, B \subseteq X$.

%--------------------------------
\begin{definition}
	The \textit{boundary} of $A$, denoted $\partial A$, is defined to be the complement of the interior of $A$ in the closure of $A$; i.e.,
	$$
	\partial A = \overline A \setminus A^\circ.
	$$
\end{definition}
%--------------------------------


%--------------------------------
\begin{note}
	In Euclidean $n$-space, the volume of the boundary of any set must zero. For the topological spaces with countable elements, this property still holds, for the volume of any set in such spaces is zero. But for the topological spaces with uncountable elements, it is not necessarily the case.
	
	For example, let $(\mathbb R^n, \mathcal T)$ be a topological space where $\mathcal T$ is generated by $\{ B_\rho (\vec 0, 1) \}$ where $\rho$ is the Euclidean metric. In this case, $\overline{B_\rho (\vec 0, 1)} = \mathbb R^n$, and the volume of its boundary is infinite.
\end{note}
%--------------------------------


%--------------------------------
\begin{proposition}
	$\partial A$ is closed.
	
	\begin{proof}
		$\overline A$ is closed, and so is $X \setminus A^\circ$. Then we have
		$$
		\partial A = \overline A \setminus A^\circ = \overline A \cap (X \setminus A^\circ).
		$$
	\end{proof}
\end{proposition}
%--------------------------------


%--------------------------------
\begin{proposition}
	$A^\circ \cap \partial A = \emptyset$.
	
	\begin{proof}
		$$
		\begin{aligned}
			\partial A = \overline A \setminus A^\circ &\iff A^\circ \cap \partial A = \overline A \setminus A^\circ \cap A^\circ = \overline A \cap \emptyset = \emptyset.
		\end{aligned}
		$$
	\end{proof}
\end{proposition}
%--------------------------------


%--------------------------------
\begin{proposition}
	$\overline A = A^\circ \cup \partial A$.
	
	\begin{proof}
		$$
		\begin{aligned}
			\partial A = \overline A \setminus A^\circ &\iff \partial A \cup A^\circ = \overline A \setminus A^\circ \cup A^\circ = \overline A \cap (X \setminus A^\circ \cup A^\circ) \\
			&\iff \partial A \cup A^\circ = \overline A \cap X = \overline A.
		\end{aligned}
		$$
	\end{proof}
\end{proposition}
%--------------------------------


%--------------------------------
\begin{proposition}
	$A$ is closed if and only if $\partial A \subseteq A$.
	
	\begin{proof}
		$A$ is closed if and only if $A = \overline A = A^\circ \cap \partial A$.
	\end{proof}
\end{proposition}
%--------------------------------


%================================
\section{Limit Points and Isolated Points}
%================================


Let $(X, \mathcal T)$ be a topological space, and let $A \subseteq X$. Also, let $u: I \to X$ with $I \subseteq \mathbb N$ and $|I| = \aleph_0$.


%--------------------------------
\begin{definition}
	A point $x \in X$ is a \textit{limit point} (or \textit{cluster point} or \textit{accumulation point}) of $A$, denoted $L(A)$, if and only if for any neighbourhood $N$ of $x$, $N \setminus \{x\} \cap A \ne \emptyset$.
\end{definition}
%--------------------------------

%--------------------------------
\begin{definition}
	A point $x \in X$ is called a \textit{limit} of $u$ if and only if for any neighbourhood $N$ of $x$,
	$$
	|u[I] \setminus N| \in \mathbb N.
	$$
	In words, any such $N$ contains all but finite element of $u[I]$.
	
	In this case, $u$ is said to be convergent in $X$, and we say that $x$ converges to $x$.
\end{definition}
%--------------------------------


%--------------------------------
\begin{note}
	In this definition, $x$ is a limit point of $u[I]$ only if $u$ is not constant. If $u$ is constant, then $u[I]$ contains only one element, $x$. As $u[I] = \{x\}$, any set $(N \setminus \{x\}) \cap \{x\} = \emptyset$.
\end{note}
%--------------------------------


%--------------------------------
\begin{note}
	Being a limit point of the image of a sequence does not means that the sequence converges itself to this point. For example, let $u$ defined by
	$$
	u_n = \sin(n),
	$$
	and let $\mathcal T$ be an Euclidean topology on $X$. Any $x \in \mathbb R_{[0,1]}$ is a limit point of $u[I]$. But, clearly, $u$ is not convergent in $X$.
\end{note}
%--------------------------------


%--------------------------------
\begin{note}
	If $(X, \mathcal T)$ is a Hausdorff space, such as Euclidean space, then $x$ is the unique limit of $u$; in this case, write
	$$
	\lim_{n \to \infty} u_n = x.
	$$
	But, by the definition of limits of sequences, in some space, a sequence may have many limits. For example, let $\mathcal T$ be an indiscrete topology on $X$, then any sequence $u$ in the space convergent, which means any $x \in X$ is a limit of $u$.
\end{note}
%--------------------------------


%--------------------------------
\begin{definition}
	A point $x \in A$ is an \textit{isolated point} of $A$, denoted $I(A)$, if and only if there exists neighbourhood $N$ of $x$ such that $N \setminus \{x\} \cap A = \emptyset$.
\end{definition}
%--------------------------------


%--------------------------------
\begin{example}
	If $\mathcal T$ is a discrete topology on $X$, then any $x \in A$ is isolated. $X$ is discrete if and only if for any $x \in X$, there exists neighbourhood $N$ of $x$, such that $N \setminus \{x\} = \emptyset$, which means $X$ contains only isolated points. It is easy to show that $A = I(A)$ also.
\end{example}
%--------------------------------


%--------------------------------
\begin{proposition}
	The closure of $A$ is a disjoint union of its limit points and isolated points; i.e.,

	$$
	\overline A = L(A) \sqcup I(A).
	$$
	
	\begin{proof}
		$x \in \overline A$ if and only if for any neighbourhood $N$ of $x$, $N \cap A \ne \emptyset$.
		
		% todo: check if the property above is proved.
		
		If $x \in \overline A \setminus I(A)$, then, for any neighbourhood $N$ of $x$, $N \setminus \{x\} \cap A \ne \emptyset$. This is precisely the definition of limit points, so $x \in L(A)$.
		
		On the other hand, if $x \in \overline A \setminus L(A)$, then there exists some neighbourhood $N$ of $x$, such that $N \setminus \{x\} \cap A = \emptyset$. This is precisely the definition of isolated points, so $x \in I(A)$.
		
		Above all, any elements in $\overline A$ is either an element of $I(A)$ or an element of $L(A)$, but not both. So $\overline A = L(A) \sqcup I(A)$.
	\end{proof}
\end{proposition}
%--------------------------------


%--------------------------------
\begin{proposition}
	$A$ is closed if and only if it contains all its limit points.
	
	\begin{proof}
		$A$ is closed if and only if $A = \overline A$. As we have proved, in this case, $\overline A = L(A) \sqcup I(A)$. So, $L(A) \subseteq \overline A$.
		
		On the other hand, let $x \in A \setminus L(A)$, then there exists neighbourhood $N$ of $x$, such that $N \setminus \{x\} \cap A = \emptyset$. Thus such $x$ must be an isolated points of $A$. If $A = L(A) \sqcup I(A)$, then $\overline A = A$, in which case $A$ is closed. Then we have
		$$
		\begin{aligned}
			X = L(X) \sqcup I(X) & \iff X = L(X) \sqcup X \\
			&\iff L(X) = \emptyset.
		\end{aligned}
		$$
		It is easy to prove that $L(A) \subseteq L(X)$. Thus $L(A) = \emptyset$.
	\end{proof}
\end{proposition}
%--------------------------------


%================================
\section{Dense Sets}
%================================


Let $(X, \mathcal T)$ be a topological space, and let $A, B \subseteq X$.


%--------------------------------
\begin{definition}
	$A$ is \textit{dense} in $B$ if and only if for any $x \in B$, $x \in A$ or $x$ is a limit point of $A$.
\end{definition}
%--------------------------------


%--------------------------------
\begin{example}
	Let $X = \mathbb R^n$, let $\mathcal T$ be induced by Euclidean metric $\rho$ on $X$, let
	$$
	A = B(\vec 0, 1) \sqcup \{\vec x \in \mathbb R^n \setminus \mathbb Q^n\} \setminus \mathbb Q^n,
	$$
	and let
	$$
	B = A \cup \mathbb Q^n.
	$$
	$A$ is dense in $B$, although $x \notin L(A)$ and $x \in B$.
\end{example}
%--------------------------------


%--------------------------------
\begin{example}
	For any topological space $(X, \mathcal T)$, $X$ is dense in itself, even if $X = \emptyset$ (vacuously true).
\end{example}
%--------------------------------


%--------------------------------
\begin{proposition}
	$A$ is dense in $B$ if and only if $B \subseteq \overline{A}$.
	
	\begin{proof}
		$A$ is dense in $B$, so for any $x \in B$, $x \in L(A)$. Now, we have $B \subseteq L(A)$. Thus, $B \subseteq \overline A$.
		
		On the other hand, let $B \subseteq \overline A$. $\overline A = L(A) \sqcup I(A)$, so any $x \in B$ is an element of $A$ or a limit point of $A$. Thus $A$ is dense in $B$.
	\end{proof}
\end{proposition}
%--------------------------------


%--------------------------------
\begin{note}
	Naturally, $A$ is dense in $X$ if and only if $\overline A = X$. For example, $\mathbb Q$ is dense in $\mathbb R$.
\end{note}
%--------------------------------


%--------------------------------
\begin{definition}
	$A$ is said to be \textit{nowhere dense} in $X$ if and only if for any $U \in \mathcal T$, $A$ is not dense in $U$.
\end{definition}
%--------------------------------


%--------------------------------
\begin{proposition}
	$A$ is said to be \textit{nowhere dense} in $X$ if and only if the interior of its closure is empty; i.e.,
	$$
	(\overline A)^\circ = \emptyset.
	$$
	
	\begin{proof}
		Suppose $A$ is nowhere dense, but $(\overline A)^\circ \ne \emptyset$. There exists $U \in \mathcal T$ with $U \subseteq \overline A$. As $U \in \mathcal T$, any $x \in \overline A$ is also an element of $\overline A$; and as $\overline A = L(A) \sqcup I(A)$, such an $x$ is either an element of $A$ or a limit point of $A$. So $A$ is dense in $U$, which is contradicted to the condition that $A$ is nowhere dense.
		
		On the other hand, $(\overline A)^\circ = \emptyset$ implies that $\emptyset$ is the only open subset of $\overline A$. Thus, for any $U \subseteq \mathcal T$, $U \setminus \overline A \ne \emptyset$. As $U \setminus \overline A$ is also open, for any $x \in U \setminus \overline A$, there exists neighbourhood $N$ of $x$, $N \cap \overline A = \emptyset$, in this case, naturally, $N \cap A = \emptyset$. Thus $A$ is not dense in $U$.
	\end{proof}
\end{proposition}
%--------------------------------


%--------------------------------
\begin{example}
	Let $f:(\mathbb R^n, \rho) \to (\mathbb R^{m}, \rho)$ be a function continuous over $A$, where $\rho$ is the Euclidean metric, and $n < m$. $f[A]$ is nowhere dense in $\mathbb R^{m}$.
\end{example}
%--------------------------------


%--------------------------------
\begin{proposition}
	$A$ is nowhere dense if and only if $\mathcal P(\overline A) \cap \mathcal T = \emptyset$.
	
	\begin{proof}
		$A$ is nowhere dense if and only if $(\overline A)^\circ = \emptyset$. Naturally, there exists no non-empty subset in $\overline A$.
		
		On the other hand, suppose there exists non-empty $U \in \mathcal T$ with $\overline A \supseteq U$, then we have
		$$
		\begin{aligned}
			\left. U \subseteq \overline A \right|_{U \in \mathcal T} &\implies \left. U \subseteq (\overline A)^\circ \right|_{U \in \mathcal T} \\
		\end{aligned}
		$$
		But $U$ is not empty and, as $U$ is open, for any $x \in U$, there exists neighbourhood $N$ of $x$ such that $N \cap \overline A \ne \emptyset$. Clearly, such $x$ can not be isolated in $A$, so $x \in L(A)$. This implies $A$ is dense in $U$, which is contradicted to the condition.
	\end{proof}
\end{proposition}
%--------------------------------






















%