%================================
\section{Interior}
%================================


%--------------------------------
\begin{definition}
	Let $(X, \mathcal T)$ be any topological space, and let $A \subseteq X$. The \textit{interior} of $A$, denoted $A^\circ$, is defined to be the union of all open subsets of $A$; i.e.,
	$$
	A^\circ = \bigcup \mathcal U, \quad \mathcal U = \mathcal P(X) \cap \mathcal T.
	$$
\end{definition}
%--------------------------------


%--------------------------------
\begin{proposition}
	$A^\circ \subseteq A$.
	
	\begin{proof}
		Let $\mathcal U = \mathcal P(X) \cap \mathcal T$. Clearly
		$$
		A^\circ = \bigcup \mathcal U \subseteq A.
		$$
	\end{proof}
\end{proposition}
%--------------------------------


%--------------------------------
\begin{proposition}
	$A \in \mathcal T$ if and only if $A = A^\circ$.
	
	\begin{proof}
		Let $\mathcal U = \mathcal P(X) \cap \mathcal T$.
		
		$\mathcal U$ is closed under arbitrary union, so if $A = A^\circ = \bigcup \mathcal U$, then $A \in \mathcal T$.
		
		On the other hand, suppose $A \in \mathcal T$ but $A \ne A^\circ$, then $A^\circ \subsetneq A$. As $A \in \mathcal T$, there exists $U \in \mathcal P(A) \cap \mathcal T$ with $U \ni x$. But as $x \in X \setminus A^\circ$, $U$ could not be a subset of $A^\circ$. Then we have $U \in \mathcal T$ but $U \not \subseteq \bigcup \mathcal U$, which is contradicted to the assumption.
	\end{proof}
\end{proposition}
%--------------------------------


%--------------------------------
\begin{proposition}
	$(A \cap B)^\circ = A^\circ \cap B^\circ$.
	
	\begin{proof}
		$(A \cap B)^\circ \in \mathcal T$, so there exists $\mathcal U \subseteq \mathcal T$ such that 
		$$
		\bigcup \mathcal U = (A \cap B)^\circ.
		$$
		Clearly, $\mathcal U = \mathcal P(A \cap B) \cap \mathcal T$.
		
		Let $\mathcal I = \mathcal P(A) \cap \mathcal T$ and $\mathcal J = \mathcal P(B) \cap \mathcal T$, then $\mathcal I = A^\circ$ and $\mathcal J = B^\circ$. 
		
		$$
		\begin{aligned}
			\mathcal I \cap \mathcal J &= \mathcal P(A) \cap \mathcal T \cap \mathcal P(B) \cap \mathcal T \\
			&= \mathcal P(A \cap B) \cap \mathcal T \\
			&= \mathcal U.
		\end{aligned}
		$$
		Then we have
		$$
		(A \cap B)^\circ = \bigcup \mathcal U = \bigcup \mathcal I \cup \bigcup \mathcal J = A^\circ \cap B^\circ.
		$$
	\end{proof}
\end{proposition}
%--------------------------------


%--------------------------------
\begin{proposition}
	If $A \subseteq B$, then $A^\circ \subseteq B^\circ$.
	
	\begin{proof}
		Let $\mathcal U = \mathcal P(A) \cap \mathcal T$, and let $\mathcal V = \mathcal P(B) \cap \mathcal T$
	
		$A^\circ \subseteq A$, so $A \subseteq B$ implies $A^\circ \subseteq B$, so $\mathcal U \subseteq \mathcal V$. Thus,
		$$
		A^\circ = \bigcup \mathcal{U} \subseteq \bigcup \mathcal V = B^\circ.
		$$
	\end{proof}
\end{proposition}
%--------------------------------



%--------------------------------
\begin{proposition}
	Let $\mathcal T$ be induced by a metric $\rho$ on $X$. The interior of $A^\circ$ is the union of all open balls in $A$; i.e., there exists $\varepsilon \in \mathbb R_{>0}$ such that
	$$
	A^\circ = \bigcup_{x \in A} B(x, \varepsilon).
	$$
	
	\begin{proof}
		Let $\mathcal U = \mathcal P(A) \cap \mathcal T$, then any $U \in \mathcal U$,
		$$
		U = \bigcup_{x \in U} B(x, \varepsilon).
		$$
		
		??????????
	
		For any open subsets $U \in \mathcal U$ and for any $x \in U$, there exists $\varepsilon \in \mathbb R_{> 0}$ such that
		$$
		A^\circ = \bigcup \mathcal U = \bigcup_{U \in \mathcal U} \bigcup_{x \in U} B(x, \varepsilon) = \bigcup_{x \in A} B(x, \varepsilon).
		$$
	\end{proof}
\end{proposition}
%--------------------------------




































%