%================================
\section{Topological Spaces}
%================================



%--------------------------------
\begin{definition}
	[topology]
	\label{def: topology}
	Let $X$ be a set, and let a family $\mathcal T \subseteq \mathcal P(X)$. $\mathcal T$ is called a topology on $X$ iff
	\begin{enumerate}[(i)]
		\item $\emptyset, X \in \mathcal T$;
		\item $\mathcal T$ is closed under arbitrary union;
		\item $\mathcal T$ is closed under finite intersection.
	\end{enumerate}
\end{definition}
%--------------------------------


%--------------------------------
\begin{definition}
	[topological spaces]
	\label{def: topological spaces}
	Let $X$ be any set, and let $\mathcal T$ be a topology on $X$, then the pair $(X, \mathcal T)$ is called a \textit{topological space}. All subsets of $X$ in $\mathcal T$ are called \textit{open sets} in $(X, \mathcal T)$.
\end{definition}
%--------------------------------


%--------------------------------
\begin{definition}
	[closed sets]
	\label{def: closed sets}
	Let $(X, \mathcal T)$ be a topological space. A subset $V$ of $X$ is said to be \textit{closed} iff there is an open set $U$ in $X$ such that
	$$
	V = X \setminus V.
	$$
\end{definition}
%--------------------------------


%--------------------------------
\begin{proposition}
	\label{prop: dark side of topology}
	Let $(X, \mathcal T)$ be a topological space, and let $\mathcal C$ be the family of all closed sets in $X$. Then
	\begin{enumerate}[(i)]
		\item
		$\emptyset, X \in \mathcal C$;
		
		\item
		$\mathcal C$ is closed under arbitrary intersection;
		
		\item
		$\mathcal C$ is closed under finite union.
	\end{enumerate}
	
	\begin{proof}
		\
		\begin{enumerate}[(i)]
			\item
			$X \in \mathcal T$ implies $X \setminus X = \emptyset \in \mathcal C$; and $\emptyset \in \mathcal T$ implies $X \setminus \emptyset = X \in \mathcal C$;
			
			\item
			As $\mathcal T$ is closed under arbitrary union, then by Definition \ref{def: closed sets} and De Morgan's Law, $\mathcal C$ is closed under arbitrary intersection.
			
			\item
			As $\mathcal T$ is closed under finite intersection, then by Definition \ref{def: closed sets} and De Morgan's Law, $\mathcal C$ is closed under finite union.
		\end{enumerate}
	\end{proof}
\end{proposition}
%--------------------------------


%--------------------------------
\begin{definition}
	[finer and coarser topology]
	\label{def: finer and coarser topology} Let $X$ be any set, and let $\mathcal T, \mathcal T'$ be topologies on $X$. $\mathcal T$ is said to be \textit{finer} than $\mathcal T'$ iff $\mathcal T \supseteq \mathcal T'$; respectively, $\mathcal T$ is said to be \textit{coarser} than $\mathcal T'$ iff $\mathcal T \subseteq \mathcal T'$.
\end{definition}
%--------------------------------


%--------------------------------
\begin{definition}
	[neighbourhood]
	\label{def: neighbourhood}
	Given $(X, \mathcal T)$ as a topological space and a point $x \in X$, a subset $N \subseteq X$ is called a \textit{neighbourhood} iff it contains an open set $U$ containing $x$.
\end{definition}
%--------------------------------


%--------------------------------
\begin{proposition}
	\label{prop: alt-def of open sets by neighbourhoods}
	Given $(X, \mathcal T)$ as a topological space and $U \subseteq X$, $U$ is open iff for all $x \in U$, there is a neighbourhood $N$ of $x$ contained in $U$.
	
	\begin{proof}
		If $U$ is open, then $U$ itself is a neighbourhood of $x$ contained in $U$.
		
		Conversely, if for all $x \in U$, there is a neighbourhood $N_x$ of $x$ contained in $U$, then there is a open neighbourhood $U_x \ni x$ contained in $N_x$. Then we have
		$$
		U \supseteq \bigcup_{x \in U} U_x.
		$$
		Suppose $U$ is not open, then $U$ is a proper superset in the relation above. Then there exists $y \in U$ which is not in any $U_x$. This implies that such a $y$ does not have any neighbourhood $N_y$ in $U$, for such an $N_y$ must contains an open $U_y \ni y$. For if it does, then there must be a $U_x$ contains $y$. This is a contradiction. Thus,
		$$
		U = \bigcup_{x \in U} U_x
		$$
		is open.
	\end{proof}
\end{proposition}
%--------------------------------