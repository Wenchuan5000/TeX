%================================
\section{More on Separation Axioms}
%================================


%--------------------------------
\begin{definition}
	[saperated sets]
	Let $(X, \mathcal T)$ be a topological space, and let $A, B \in \mathcal P(X)$.
	
	\begin{enumerate}[(i)]
		\item $A$ and $B$ are said to be \textit{separated} iff each is disjoint from other's closure.
		\item $A$ and $B$ are said to be \textit{separated by neighbourhoods} iff there are neighbourhoods $N_A$ of $A$ and $N_B$ of $B$ such that $N_A$ and $N_B$ are disjoint.
		\item $A$ and $B$ are said to be \textit{separated by closed neighbourhoods} iff there are closed neighbourhoods $\overline N_A$ of $A$ and $\overline N_B$ of $B$ such that $\overline N_A$ and $\overline N_B$ are disjoint.
		\item $A$ and $B$ are said to be \textit{separated by a continuous function} iff there is a continuous function $f: X \to \mathbb R$, such that $f[A] = \{0\}$ and $f[B] = \{1\}$.
		\item $A$ and $B$ are said to be \textit{precisely separated by a continuous function} iff there is a continuous function $f: X \to \mathbb R$, such that $f^{-1}[\{0\}] = A$ and $f^{-1}[\{1\}] = B$
	\end{enumerate}
\end{definition}
%--------------------------------


%--------------------------------
\begin{definition}
	[$T_{2 \nicefrac{1}{2}}$ spaces]
	\label{def: T_2.5 spaces}
	A topological space $(X, \mathcal T)$ is said to be $T_{2 \nicefrac{1}{2}}$ or \textit{Urysohn} iff two distinct points in $X$ are separated by closed neighbourhoods.
\end{definition}
%--------------------------------


%--------------------------------
\begin{example}
	[$T_2$ but not $T_{2 \nicefrac{1}{2}}$]
	\footnote{
		See \href{https://planetmath.org/hausdorffspacenotcompletelyhausdorff}{MathPlanet}.
	}
	(Remained as a problem)
	% todo: Remained as a problem!
\end{example}
%--------------------------------


%--------------------------------
\begin{definition}
	[$T_3$ spaces]
	\label{def: T_3 spaces}
	A topological space $(X, \mathcal T)$ is said to be $T_3$ or \textit{regular} iff it is $T_0$ and given any point $x \in (X, \mathcal T)$ and closed set $V \subseteq X$ with $x \notin V$ are separated by neighbourhoods.
\end{definition}
%--------------------------------


%--------------------------------
\begin{definition}
	[$T_{3\nicefrac{1}{2}}$ spaces]
	\label{def: T_3.5 spaces}
	A topological space $(X, \mathcal T)$ is said to be $T_{3 \nicefrac{1}{2}}$, or \textit{Tychonoff} or, \textit{completely $T_3$}, or \textit{completely regular}, iff it is $T_0$ and given any point $x$ and closed set $V \subseteq X$ with $x \notin V$, they are separated by a continuous function.
\end{definition}
%--------------------------------


%--------------------------------
\begin{definition}
	[$T_4$ spaces]
	\label{def: T_4 spaces}
	A topological space $(X, \mathcal T)$ is said to be $T_4$ or \textit{normal} iff it is Hausdorff and any tow disjoint closed subsets of $X$ are separated by neighbourhoods.
\end{definition}
%--------------------------------


%--------------------------------
\begin{proposition}
	[Urysohn's lemma]
	\label{prop: urysohn's lemma}
	A topological space is normal iff any two disjoint closed sets are separated by a continuous function.
\end{proposition}
%--------------------------------


%--------------------------------
\begin{definition}
	[$T_5$ spaces]
	\label{def: T_5 spaces}
	A topological space $(X, \mathcal T)$ is said to be $T_5$ or \textit{completely $T_4$} iff it is $T_1$ any two separated sets are separated by neighbourhoods.
\end{definition}
%--------------------------------


%--------------------------------
\begin{proposition}
	Every subspace of a $T_5$ space is normal.
\end{proposition}
%--------------------------------


%--------------------------------
\begin{definition}
	[$T_6$ spaces]
	\label{def: T_6 spaces}
	A topological space $(X, \mathcal T)$ is said to be $T_6$, or \textit{perfectly $T_4$} or \textit{perfectly normal} iff it is $T_1$ and any two disjoint closed sets are precisely separated by a continuous function.
\end{definition}
%--------------------------------



%--------------------------------
\begin{proposition}
	[Tietze extension theorem]
	\label{prop: Tietze extension theorem}
	Let $(X, \mathcal T)$ be normal topological space, and let $f: A \to (\mathbb R, \mathcal T')$ be a continuous map where $A$ is a closed subset of $X$ and $\mathcal T'$ is the standard topology (induced by Euclidean metric). Then there exists a continuous map
	$$
	F: (X, \mathcal T) \to (\mathbb R, \mathcal T'),
	$$
	such that
	$$
	\forall x \in A: f(x) = g(x).
	$$
\end{proposition}
%--------------------------------