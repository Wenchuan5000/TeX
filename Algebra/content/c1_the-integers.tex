%================================
%::::::::::::::::::::::::::::::::
\chapter{The Integers}
%::::::::::::::::::::::::::::::::
%================================


%================================
\section{Terminology of Sets}
%================================


%================================
\section{Basic Properties}
%================================


\begin{theorem}
	[Induction: First Form]
	Suppose that for each integer $n \ge 1$ we are given an assertion $A(n)$, and that we can prove the following two properties:
	\begin{enumerate}[(1)]
		\item The assertion $A(1)$ is true.
		\item For each integer $n \ge 1$, if $A(n)$ is true, then $A(n + 1)$ is true.
	\end{enumerate}
	Then for all integers $n \ge 1$, the assertion $A(n)$ is true.
\end{theorem}


\begin{theorem}
	[Induction: Second Form]
	Suppose that for each integer $n \ge 0$ we are given an assertion $A(n)$, and that we can prove the following two properties:
	\begin{enumerate}[(i')]
		\item The assertion $A(0)$ is true;
		\item For each integer $n > 0$, if $A(k)$ is true for every integer $k$ with $0 \le k < n$, then $A(n)$ is true.
	\end{enumerate}
	Then the assertion $A(n)$ is true for all integers $n \ge 0$.
\end{theorem}


\begin{theorem}
	[Euclidean Algorithm]
	Let $m,n$ be integers and $m > 0$. Then there exists integers $q,r$ with $0\le r < m$ such that
	$$
	n = qm + r.
	$$
	The integers $q,r$ are uniquely detemined by these conditions.
\end{theorem}


\begin{proof}
	For $m = n$, then $q = 1$ and $r = 0$ are unique.
	
	For $m < n$, there is a greatest integer $q$ such that
	$$
	0 \le n - qm < m.
	$$
	Because if $q$ is not the greatest, then there must be $q + 1$ such that the inequality holds. But
	$$
	0 \le n - (q + 1)m \iff m \le n - qm,
	$$
	which is impossible. Thus $q$ must be the greatest one.
	
	Secondly, there is a smallest integer $q$ such that
	$$
	0 \le n - qm < m.
	$$
	Because if it is not, then $q - 1$ makes the inequality holds. But
	$$
	n - (q - 1) m < m \iff n - qm < 0,
	$$
	which is impossible. Thus $q$ must be the smallest one.
	
	As $q$ is the greatest as well as the smallest one, then $q$ is unique.
	
	Suppose $r$ is not unique, then there must be $s \in \mathbb Z_{[0,m)}$ with $s \ne r$ such that
	$$
	\begin{aligned}
		& n = qm + r, \text{ and}\\
		& n = qm + s.
	\end{aligned}
	$$
	then, we have
	$$
	0 = r - s \ne 0,
	$$
	a contradiction. So $r$ is unique.\end{proof}

%::::::::::::::::::::::::::::::::
\subsection*{Exercises}
%::::::::::::::::::::::::::::::::

\begin{enumerate}[1.]
	%--------------------------------
	\item If $m,n$ are integers $\ge 1$ and $n \ge m$, define the \textbf{binomial coefficient}
	$$
	{n \choose m} = \frac{n!}{m!(n - m)!}.
	$$
	As usual, $n! = n \cdot (n - 1) \cdots 1$ is the product of the first $n$ integers. We define $0! = 1$ and ${n \choose 0} = 1$. Prove that
	$$
	{n \choose m - 1} + {n \choose m} = {n + 1 \choose m}.
	$$
	\begin{proof}
		This one can be straightly proved by the definition of binomial coefficient as following.
		$$
		\begin{aligned}
			{n \choose m - 1} + {n \choose m} &= \frac{n!}{(m - 1)! (n - m + 1)!} + \frac{n!}{m! (n - m)!} \\
			&= \frac{n! m}{m!(n - m + 1)!} + \frac{n!(n - m + 1)}{m!(n - m + 1)!} \\
			&= \frac{n!}{m!(n - m + 1)!}(m + n - m + 1) \\
			&= \frac{(n + 1)!}{m!(n + 1 - m)!} \\
			&= {n + 1 \choose m}.
		\end{aligned}
		$$
	\end{proof}
	
	
	%--------------------------------
	\item Prove by induction that for any integers $x,y$ we have
	$$
	(x + y)^n = \sum_{i = 1}^n {n \choose i} x^i y^{n-i} = y^n + {n \choose 1}xy^{n-1} + {n \choose 2}x^2 y^{n-2} + \cdots + x^n.
	$$
	\begin{proof}
		The equation holds for $n = 1$, because
		$$
		(x + y)^1 = x + y.
		$$
		Assume the equation holds for any integer $n \ge 1$, then
		$$
		\begin{aligned}
			(x + y)^{n + 1} &= (x + y) \sum_{i = 0}^n {n \choose i}x^iy^{n - i} \\
			& = \sum_{i = 0}^n \left[ {n \choose i} x^iy^{n + i} + {n \choose i} x^{i + 1}y^{n - i - 1} \right].
		\end{aligned}
		$$
		By Exercise 1, it is easy to prove that
		$$
		{n \choose k} = {n + 1 \choose k + 1} - {n \choose k + 1}.
		$$
		Then the equation is
		$$
		\begin{aligned}
			&\sum_{i = 0}^{n + 1} \left[ {n + 1 \choose i} x^i y^{n + 1 - i} - {n \choose i} x^iy^{n + 1 - i} + {n \choose i} x^i y^{n + 1 - i} \right]\\
			=& \left. \sum_{i = 0}^{n + 1}{n + 1 \choose i}x^i y^{n + 1 - i} \right|_{\text{let $k = n + 1$}} \\
			=& \sum_{i = 0}^k {k \choose i} x^i y^{k - i}.
		\end{aligned}
		$$
	\end{proof}
	
	%--------------------------------
	\item Prove the following statements for all positive integers:
	\begin{enumerate}[(a)]
		\item $1 + 3 + 5 + \cdots + (2n - 1) = n^2$;
		\item $1^2 + 2^2 + \cdots + n^2 = n(n + 1)(2n + 1)/6$;
		\item $1^3 + 2^3 + 3^3 + \cdots + n^3 = [n (n + 1)/2]^2$.
	\end{enumerate}
	
	\begin{proof}
		\begin{enumerate}[(a)]
			\item Clearly the equation holds for $n = 1$. Suppose it holds for all integer $n \ge 1$, then we have
			$$
			\sum_{i = 1}^{n+1}(2n - 1) = n^2 + 2n + 1 = (n + 1)^2
			$$
			
			\item Clearly the equation holds for $n = 1$. Suppose it holds for all integer $n \ge 1$, then we have
			$$
			\begin{aligned}
				\sum_{i = 1}^{n + 1}i^2 &= \frac{n(n + 1)(2n + 1)}{6} + (n + 1)^2 \\
				&= \frac{n(n + 1)(2n + 1) + 6(n + 1)^2}{6} \\
				&= \frac{(n + 1)(2n^2 + 7n + 6)}{6} \\
				&= \left. \frac{(n + 1)(n + 2)(2n + 3)}{6} \right|_\text{let $k = n + 1$} \\
				&= \frac{k(k + 1)(2k + 1)}{6}.
			\end{aligned}
			$$
			
			\item Clearly the equation holds for $n = 1$. Suppose it holds for all integer $n \ge 1$, then we have
			$$
			\begin{aligned}
				\sum_{i = 1}^{n + 1} i^3 &= \left( \frac{n(n + 1)}{2} \right)^3 + (n + 1)^3 \\
				&= \frac{n^2 (n + 1)^2 + 4(n + 1)^3}{4} \\
				&= \frac{(n + 1)^2 (n + 2)^2}{4} \\
				&= \left. \left( \frac{(n + 1)(n + 2)}{2} \right)^2 \right|_\text{let $k = n + 1$}\\
				&= \left( \frac{k (k + 1)}{2} \right)^2
			\end{aligned}
			$$
		\end{enumerate}
	\end{proof}


	%--------------------------------
	\item Prove that
	$$
	\left( 1 + \frac{1}{1} \right)^1 \left( 1 + \frac{1}{2} \right)^2 \cdots \left( 1 + \frac{1}{n - 1} \right)^{n-1} = \frac{n^{n-1}}{(n + 1)!}
	$$
	\begin{proof}
		The equiation holds for $n = 2$, because
		$$
		\left( 1 + \frac{1}{1} \right)^1 = 2 = \frac{2}{1!}.
		$$
		Assume the equation holds for any integer $n \ge 2$, then
		$$
		\begin{aligned}
			\prod_{i=1}^{n}\left( 1 + \frac{1}{i} \right)^{i} &= \frac{n^{n-1}}{(n-1)!}\left( 1 + \frac{1}{n} \right)^n \\
			&= \frac{n^{n-1}}{(n-1)!}\frac{(n + 1)^n}{n^n} \\
			&= \frac{n^{n-1}(n+ 1)^n}{n! n^{n-1}} \\
			&= \left. \frac{(n + 1)^n}{n!} \right|_{\text{let $k = n + 1$}} \\
			&= \frac{k^{k - 1}}{(k -1)!}.
		\end{aligned}
		$$
	\end{proof}
	
	%--------------------------------
	\item Let $x$ be a real number. Prove that there exists an integer $q$ and a real number $s$ with $0 \le s < 1$ such that $x = q + s$, and that $q, s$ are uniquely detemined. Can you deduce the Euclidean algorithm from this result without using induction?
	\begin{proof}
		This is just a straight corollary of Euclidean algorithm.
	\end{proof}
\end{enumerate}


%================================
\section{Greatest Common Divisor}
%================================

%--------------------------------
\begin{definition}
	Given $n,d \in \mathbb Z \setminus \{ 0 \}$, we shall say that \textit{$d$ divides $n$}, or $d$ is a \textit{divisor} of $n$, denoted $d|n$, iff
	$$
	\exists q \in \mathbb Z, \quad n = dq.
	$$
\end{definition}

The divisors of $n$ is a set
$$
\mathrm{div}(n) = \{ d \in \mathbb Z \setminus \{0\} : d|n \}.
$$
For example,
$$
\begin{aligned}
	\mathrm{div}(8) &= \{ \pm 1, \pm 2, \pm 4, \pm 8 \}, \\
	\mathrm{div}(-24) &= \{ \pm 1, \pm 2, \pm 3, \pm 4, \pm 6, \pm 8,  \pm 12 \}, \\
	\mathrm{div}(35) &= \{ \pm 1, \pm 5, \pm 7, \pm 36 \} .
\end{aligned}

Clearly, for all $n \in \mathbb Z \setminus \{0\}$, for all $x \in \mathrm{div}(n) \setminus \{\pm n\}$
$$
|x| \le \frac{|n|}{2}.
$$

%--------------------------------
\begin{definition}
	Given $m, n \in \mathbb Z \setminus \{0\}$, the \textit{common divisor} is defined to be the set
	$$
	\mathrm{cd}(m,n) = \left\{ d \in \mathbb Z_{> 0} : d|m \land d|n \right\}.
	$$
	Thus,
	$$
	\mathrm{cd}(m,n) = \mathrm{div}(m)_{> 0} \cap \mathrm{div}(n)_{>0}
	$$
\end{definition}


For example,
$$
\begin{aligned}
	\mathrm{cd}(18, 12) &= \{ 2, 3, 6 \}, \\
	\mathrm{cd}(-18, 12) &= \{ 2, 3, 6 \}, \\
	\mathrm{cd}(24, -20) &= \{ 2, 4 \},
\end{aligned}
$$


























%